\documentclass{book}


% Set paper and layout
%
\usepackage[landscape, top=1cm, left=1cm, right=1cm, bottom=1cm]{geometry}
\setlength\parskip{-0.05cm} % no extra vertical space between paragraphs
\setlength\parindent{0cm} % no indent on first lines of paragraphs
\linespread{1.75} % vertical line-spacing and vertical paragraph spacing


% Use Paracol package to do two-column format
%
\usepackage[english]{babel} % English as "system" language, paracol demands a specification
\usepackage{paracol} % \begin{paracol}{...} ... \switchcolumn[...] ... \end{paracol}
\setlength\columnsep{1.5cm} % separation between columns
\newcommand{\s}{\switchcolumn} % a shorter command to replace \switchcolumn[...] from the paracol-package


% Use Endnotes packages, so ``footnotes'' go at end of document rather than the foot of each page
%
\usepackage{endnotes} % \endnote{...}
\addtoendnotes{\vspace*{-1em}} % no extra vertical space between the endnotes-header and the endnotes
%%% hyperendnotes.sty
\makeatletter
\newif\ifenotelinks
\newcounter{Hendnote}
% Redefining portions of endnotes-package:
\let\savedhref\href
\let\savedurl\url
\def\endnotemark{%
\@ifnextchar[\@xendnotemark{%
\stepcounter{endnote}%
\protected@xdef\@theenmark{\theendnote}%
\protected@xdef\@theenvalue{\number\c@endnote}%
\@endnotemark
}%
}%
\def\@xendnotemark[#1]{%
\begingroup\c@endnote#1\relax
\unrestored@protected@xdef\@theenmark{\theendnote}%
\unrestored@protected@xdef\@theenvalue{\number\c@endnote}%
\endgroup
\@endnotemark
}%
\def\endnotetext{%
\@ifnextchar[\@xendnotenext{%
\protected@xdef\@theenmark{\theendnote}%
\protected@xdef\@theenvalue{\number\c@endnote}%
\@endnotetext
}%
}%
\def\@xendnotenext[#1]{%
\begingroup
\c@endnote=#1\relax
\unrestored@protected@xdef\@theenmark{\theendnote}%
\unrestored@protected@xdef\@theenvalue{\number\c@endnote}%
\endgroup
\@endnotetext
}%
\def\endnote{%
\@ifnextchar[\@xendnote{%
\stepcounter{endnote}%
\protected@xdef\@theenmark{\theendnote}%
\protected@xdef\@theenvalue{\number\c@endnote}%
\@endnotemark\@endnotetext
}%
}%
\def\@xendnote[#1]{%
\begingroup
\c@endnote=#1\relax
\unrestored@protected@xdef\@theenmark{\theendnote}%
\unrestored@protected@xdef\@theenvalue{\number\c@endnote}%
\show\@theenvalue
\endgroup
\@endnotemark\@endnotetext
}%
\def\@endnotemark{%
\leavevmode
\ifhmode
\edef\@x@sf{\the\spacefactor}\nobreak
\fi
\ifenotelinks
\expandafter\@firstofone
\else
\expandafter\@gobble
\fi
{%
\Hy@raisedlink{%
\hyper@@anchor{Hendnotepage.\@theenvalue}{\empty}%
}%
}%
\hyper@linkstart{link}{Hendnote.\@theenvalue}%
\makeenmark
\hyper@linkend
\ifhmode
\spacefactor\@x@sf
\fi
\relax
}%
\long\def\@endnotetext#1{%
\if@enotesopen
\else
\@openenotes
\fi
\immediate\write\@enotes{%
\@doanenote{\@theenmark}{\@theenvalue}%
}%
\begingroup
\def\next{#1}%
\newlinechar='40
\immediate\write\@enotes{\meaning\next}%
\endgroup
\immediate\write\@enotes{%
\@endanenote
}%
}%
\def\theendnotes{%
\immediate\closeout\@enotes
\global\@enotesopenfalse
\begingroup
\makeatletter
\edef\@tempa{`\string>}%
\ifnum\catcode\@tempa=12
\let\@ResetGT\relax
\else
\edef\@ResetGT{\noexpand\catcode\@tempa=\the\catcode\@tempa}%
\@makeother\>%
\fi
\def\@doanenote##1##2##3>{%
\def\@theenmark{##1}%
\def\@theenvalue{##2}%
\par
\smallskip %<-small vertical gap between endnotes
\begingroup
\def\href{\expandafter\savedhref}%
\def\url{\expandafter\savedurl}%
\@ResetGT
\edef\@currentlabel{\csname p@endnote\endcsname\@theenmark}%
\enoteformat
}%
\def\@endanenote{%
\par\endgroup
}%
% Redefine, how numbers are formatted in the endnotes-section:
\renewcommand*\@makeenmark{%
\hbox{\normalfont\@theenmark~}%
}%
% header of endnotes-section
\enoteheading
% font-size of endnotes
\enotesize
\input{\jobname.ent}%
\endgroup
}%
\def\enoteformat{%
\rightskip\z@
\leftskip1.8em
\parindent\z@
\leavevmode\llap{%
\setcounter{Hendnote}{\@theenvalue}%
\addtocounter{Hendnote}{-1}%
\refstepcounter{Hendnote}%
\ifenotelinks
\expandafter\@secondoftwo
\else
\expandafter\@firstoftwo
\fi
{\@firstofone}%
{\hyperlink{Hendnotepage.\@theenvalue}}%
{\makeenmark}%
}%
}%
% stop redefining portions of endnotes-package:
\makeatother
% Toggle switch in order to turn on/off back-links in the
% endnote-section:
\enotelinkstrue
%\enotelinksfalse % enables links between endnotes and citing locations
\def\enoteheading{} % to have endnotes without the usual heading


% Calc package is necessary.  Does some math on lengths?  
%
\usepackage{calc}


% Use Graphicx package to rotate or flip text
% (eg. \reflectbox{C} --- \rotatebox[origin=c]{180}{V} --- \reflectbox{D})
%
\usepackage{graphicx} 


% Use math packages for fonts, symbols, etc.
%
\usepackage{upgreek} % certain upright greek characters
\usepackage{amsmath} % for $\text{cl.}^{\text{mus}}$
\usepackage{amssymb} % to have standard symbols for special sets --- to have $\nless$

\newcommand{\dittoMarkLatin}{\guillemotright}
\newcommand{\dittoMarkEnglish}{''}

% Peano's reverse C, that is used for implication.
\newcommand{\C}{\mathop{\sbox0{$\displaystyle$$\reflectbox{C}$} % reflect box mirrors upside-down and left-to-right
\raisebox{-\height+\ht0-0.05cm}[\ht0][\dp0]{\scalebox{0.8}[0.8]{\copy0}}}\displaylimits} %

% Peano's one dot, used to indicate precidence in equations
\newcommand{\p}{\mathop{\sbox0{$\displaystyle$$.$}
\raisebox{-\height+\ht0}[\ht0][\dp0]{\scalebox{1}[1]{\copy0}}}\displaylimits} % creates a "math"-point to have equal spacing
% Peano's 2 dots is just a colon :, and is used to indicate precidence in equations
% Peano's 3 dots, used to indicate precidence in equations
%    I tried using \therefore but the dots were too far separated
\newcommand{\pppNoSpace}{\leavevmode\lower0ex\hbox{.}\kern-0.1em\raise0.7ex\hbox{.}\kern-0.1em\lower0ex\hbox{.}} % arranging three points in a triangle
\newcommand{\ppp}{\thinspace \pppNoSpace\thinspace} % arranging three points in a triangle

%TODO: Some of these might be better off with mathrel or mathbin instead of mathop
\newcommand{\no}{\mathop{\thinspace \scalebox{1.5}{-}}}
\newcommand{\abs}{\mathop{\rotatebox[origin=c]{180}{V}}}
\newcommand{\mini}{\mathop{\rotatebox[origin=c]{180}{M}}}
\newcommand{\larger}{\mathop{\thinspace > \thinspace}}
\newcommand{\smaller}{\mathop{\thinspace < \thinspace}}
\newcommand{\such}{\thinspace \rotatebox[origin=c]{180}{$\epsilon$}}
\newcommand{\mult}{\mathop{\reflectbox{D}}}
\newcommand{\smallIn}{\ensuremath{\mathrel{\epsilon}}}
\newcommand{\primeWith}{\ensuremath{\mathbin{\uppi}}}

\newcommand{\I}{\mathop{\thinspace \textbf{I} \thinspace}}
\newcommand{\E}{\mathop{\thinspace \textbf{E} \thinspace}}
\newcommand{\Lfat}{\mathop{\thinspace \textbf{L} \thinspace}}
\newcommand{\D}{\mathop{\thinspace \text{D} \thinspace}}
\newcommand{\K}{\mathop{\thinspace \text{K} \thinspace}}
\newcommand{\R}{\mathop{\thinspace \text{R} \thinspace}}
\newcommand{\M}{\mathop{\thinspace \text{M} \thinspace}}
\newcommand{\N}{\mathop{\thinspace \text{N} \thinspace}}
\newcommand{\Np}{\mathop{\thinspace \text{Np} \thinspace}}
\newcommand{\Q}{\mathop{\thinspace \text{Q} \thinspace}}
\newcommand{\T}{\mathop{\thinspace \text{T} \thinspace}}

\newcommand{\II}{\mathop{\thinspace \textbf{I\hspace{0.02cm}I} \thinspace}}
\newcommand{\EE}{\mathop{\thinspace \textbf{E\hspace{0.02cm}E} \thinspace}}
\newcommand{\LfatLfat}{\mathop{\thinspace \textbf{L\hspace{0.02cm}L} \thinspace}}

\newcommand{\setOfSets}{\mathord{\text{{\footnotesize \textbf{SET}}}}}
\newcommand{\prop}{\mathord{\text{{\footnotesize \textbf{PROP}}}}}


\DeclareMathOperator{\interior}{int}
\DeclareMathOperator{\exterior}{ext}
\DeclareMathOperator{\boundary}{bd}



% Use Dashrule package for dashed lines (using \hdashrule)
%
\usepackage{dashrule} 


% Use Xcolor package to grey out unimportant text, etc.
%
\usepackage{xcolor}

% na for not applicable
% (these pieces of text are not important for mathematical
%  understand when reading the english text)
\newcommand\irrelavent[1]{\textcolor{gray}{#1}} 
\newcommand\commentary[1]{\textcolor{red}{COMMENTARY: #1}} 
\newcommand\todo{\textcolor{brown}{To do.}}
\newcommand\notPossible{\textcolor{brown}{N/A}}


% Use Hyperref package for URLs
%   ShareLaTeX.com said ``Usually [hyperref] has to be the last package to be imported''
%
\usepackage[colorlinks=true,linkcolor=red, urlcolor=red]{hyperref} % color links and url's --- hyperref for hyperlinks in endnotes
\urlstyle{same} % standard font for url's


% Heading commands
%    - assumes it is already in a centered format.
%
\newcommand\peanoHeadingSmall[1]{ \vspace{0.75cm} \textit{#1} \nopagebreak[4]

\vspace{0.25cm} \nopagebreak[1]}
\newcommand\peanoHeadingMedium[1]{ \vspace{1cm} {\Large #1} \nopagebreak[4]

  \noindent\rule{1cm}{0.4pt} \nopagebreak[1]}
\newcommand\peanoHeadingLarge[1]{ \vspace{2cm} {\Large \textbf{\uppercase{#1}} \nopagebreak[4]}

  \noindent\rule{2cm}{0.4pt} \nopagebreak[1]}
\newcommand\eendnoteHeading[1]{ {\Large #1} \nopagebreak[1]}


% Define commands for document structure
%
% Command for the end of each page of Peano's original book
\newcommand{\peanoPage}[1]{\vspace{1ex}

  \columnratio{0.475, 0.05, 0.475} \begin{paracol}{3} \centering \hdashrule{\columnwidth}{0.1mm}{0.1mm 1mm} \s #1 \s \hdashrule{\columnwidth}{0.1mm}{0.1mm 1mm} \end{paracol}

\vspace{1ex}}
%
% FANCY LaTeX!  Create translation environment and commands \LAT and \ENG that only work inside it.
\newenvironment{translateTwoCol}
               { % when beginning the environment
                 \columnratio{0.5, 0.5} \begin{paracol}{2}
                 \newcommand{\LAT}{\switchcolumn[0]*}
                 \newcommand{\ENG}{\switchcolumn[1]}
               }
               { % when ending the environment
                 \let\ENG\undefined
                 \let\LAT\undefined
                 \end{paracol}
               }

\newenvironment{translateSixCol}[6]
               { % when beginning the environment
                 % This complex line is need if you don't want \columnsep
                 % space between each column.
                 \setcolumnwidth{#1\fill/1em, #2\fill/1em, #3\fill/\columnsep, #4\fill/1em, #5\fill/1em, #6\fill}
                 \begin{paracol}{6}
               }
               { % when ending the environment
                 \end{paracol}
               }
               

\newenvironment{translateEightCol}[8]
               { % when beginning the environment
                 % This complex line is need if you don't want \columnsep
                 % space between each column.
                 \setcolumnwidth{#1\fill/1em, #2\fill/1em, #3\fill/1em, #4\fill/\columnsep, #5\fill/1em, #6\fill/1em, #7\fill/1em, #8\fill}
                 \begin{paracol}{8}
               }
               { % when ending the environment
                 \end{paracol}
               }
               


               
%%%%%%%%%%%%%%%%%%%%%%%%%%%%%%%%%%%%%%%%%%%%%%
\begin{document} 
%%%%%%%%%%%%%%%%%%%%%%%%%%%%%%%%%%%%%%%%%%%%%%


% Disable page numbering.
%   With tight margins, the numbers were in weird locations.
%   Probably want to re-enable this later, if we can.
\pagenumbering{gobble}


% Title Page
%
% TODO: fix spacing
{ \centering

  \vspace*{1cm}
  
  {\Large Giuseppe Peano's}

  {\normalsize Classic Mathematical Text}

  \vspace{1cm}

  {\Huge Arithmetices principia, nova methodo exposita}

  \vspace{0.5cm}

  {\Large OR}

  \vspace{0.5cm}

  {\Huge The Principles of Arithmetic, Presented by a New Method}

  \vspace{1cm}

  Presented

  {\Large in the original Latin}

  AND

  {\Large in parallel English Translation}

  \vspace{1cm}

  Original Translation By:

  {\Large Vincent Verheyen}

  \vspace{1cm}

  Contributions By:

  % ADD YOUR NAME HERE
  Michael Nahas

  \vfill

  % I also wanted to include the git revision, but it either required a Makefile
  % or it required enabling LaTeX to run shell commands, which requires an extra
  % flag on the command-line.  I thought that would be too confusing.
  
  \normalsize This document was compiled \today.

  \normalsize This document is licensed under Creative Commons Attribution-ShareAlike 4.0

  \normalsize This document is hosted at \url{https://github.com/mdnahas/Peano_Book}

} % end of centering, textsize changes
\newpage  


% About the translation/book page
%
% Center text
\columnratio{0.15, 0.7, 0.15}
\begin{paracol}{3}
  \switchcolumn[1] % use center column

  { \Large Note from Original Translation}

   \vspace{0.25cm}
  
%  \everypar{\hangindent0.5cm}
   Below is Giuseppe Peano's \emph{Arithmetices principia} as first published\footnote{H. Kennedy, \emph{Peano. Life and Works of Giuseppe Peano}, San Francisco: Peremptory Publications, 2002, p. 41.}, i.e. as ``\emph{Arithmetices principia, nova methodo exposita}"\footnote{G. Peano, \emph{Arithmetices principia, nova methodo exposita}, Bocca, Torino, 1889.} , which appeared translated to English\footnote{These English translations listed are the only ones (to my knowledge) and all the English translations listed in: \\
   I. Grattan-Guiness (ed.), \emph{Landmark Writings in Western Mathematics 1640-1940}, Amsterdam: Elsevier, 2005, p. 614.} in 1967 as ``\emph{The principles of arithmetic, presented by a new method}"\footnote{G. Peano, (1889), ``The principles of arithmetic presented by a new method" in: J. van Heijenoort (ed.), \emph{From Frege to G\"odel. A source book in mathematical logic. 1879-1931}, Cambridge: Harvard University Press, 1967, p. 83-97.}, as well as in 1973\footnote{G. Peano, \emph{Selected works of Giuseppe Peano}, H. Kennedy (ed.), London: George Allen \& Unwin, 1973, p. 101-134.}.
   This present document\footnote{Written by Vincent Verheyen. Last updated on 17/8/2015. I encourage you to use your reason for good. If you want my support, please contact me via \url{http://vincentverheyen.com/contact}. It is possible to contribute to the flourishing of knowledge, even when you have an intelligence like mine. Thank you and good luck studying. \\
   I would like to thank Mauro Allegranzo and acknowledge his support of this work and his various comments during its creation.} is the only (to my knowledge) side-by-side Latin-English translation of the Latin original.
   The mathematical notation (in the English, right, column) got updated to currently canonically-used or easy-to-decrypt symbols in the international and/or English mathematical community; which is also a feature currently unseen in any reprint.

   \vspace{1cm}
   
   \textcolor{red}{Red text} is mathematical commentary.

   \irrelavent{Gray text} is irrelavant for modern mathematical notation.

   Dashed lines (\hdashrule{2cm}{0.1mm}{0.1mm 1mm}) indicate pages in the original treatise.

   \vspace{1cm}
\end{paracol}
\newpage  



\peanoPage{I} % page-number I

\vspace{1em}

\begin{translateTwoCol}
\centering
Arithmetices principia
\ENG
The principles of arithmetic
\LAT
Nova methodo exposita
\ENG
Presented by a new method
\LAT
a
\ENG
by
\LAT
Ioseph Peano
\vspace{1em}
\ENG
Giuseppe Peano
\vspace{1em}
\LAT
in R. Academia militari professore
\ENG
\irrelavent{professor at the Royal Military Academy}
\LAT
Analysin infinitorum in R. Taurinensi Athen\ae o docente.
\ENG
\irrelavent{teaching Analysis of the infinite at the Royal Turin Athenaeum.}
\LAT
\vspace{1em}
Labor et honor
\vspace{1em}
\ENG 
\vspace{1em}
\irrelavent{Work and honor}
\vspace{1em}
\LAT
Augustae Taurinorum
\ENG
\irrelavent{At Turin}
\LAT
Ediderunt Fratres Bocca
\ENG
\irrelavent{Published by Libreria Bocca}
\LAT
Regis bibliopolae
\vspace{1em}
\ENG
...
\vspace{1em}
\end{translateTwoCol}

\columnratio{0.25, 0.25, 0.25, 0.25}
\begin{paracol}{4}
\centering
Romae \s Florentiae 
\s 
\irrelavent{At Rome} \s \irrelavent{At Florence}
\s*
Via del Corse, 216-217. \s Via Oerretani, 8.
\s
\irrelavent{Via del Corso, 216-217.} \s \irrelavent{Via Oerretani, 8.}
\end{paracol}

\begin{translateTwoCol}
\centering
1889
\ENG
\irrelavent{1889}
\end{translateTwoCol}

\peanoPage{II} % page-number II

\vspace{1em}
\begin{translateTwoCol}
\centering
Iuribus servatis
\ENG
\irrelavent{Respecting rights}
\LAT
\vspace{1em}
Augustae Taurinorum - Typis Vincentii Bona.
\ENG
\vspace{1em}
\irrelavent{At Turin, printed by Vincent Bona}
\end{translateTwoCol}

\pagebreak

\vspace{1em}
\peanoPage{III} % page-number III

\begin{translateTwoCol}
\centering
\phantomsection
\addcontentsline{toc}{chapter}{Preface}
\peanoHeadingLarge{Praefatio}
\ENG
\peanoHeadingLarge{Preface}
\end{translateTwoCol}

\begin{translateTwoCol}
Quaestiones, quae ad mathematicae fundamenta pertinent, etsi hisce temporibus a multis tractatae, satisfacienti solutione et adhuc carent. Hic difficultas maxime ex sermonis ambiguitate oritur.
\ENG Questions pertaining to the foundations of mathematics, although treated by many these days, still lack a satisfactory solution. The difficulty arises principally from the ambiguity of ordinary language.
\LAT
Quare summi interest verba ipsa, quibus utimur attente perpendere. Hoc examen mihi proposui, atque mei studii resultatus, et arithmeticae applicationes in hoc scripto expono.
\ENG For this reason it is of the greatest concern to consider attentively the words we use. I resolved to do this, and am presenting in this paper the results of my study with applications to arithmetic.
\LAT
Ideas omnes quae in arithmeticae principiis occurrunt, signis indicavi, ita ut quaelibet propositio his tantum signis enuncietur.
\ENG I have indicated by symbols all the idea which occur in the fundamentals of arithmetic, so that every proposition is stated with just these symbols.
\LAT
Signa aut ad logicam pertinent, aut proprie ad arithmeticam. Logicae signa quae hic occurrunt, sunt numero ad decem, quamvis non omnia necessaria. Horum signorum usus et proprietas nonnullae in priore parte communi sermone explicantur. Ipsorum theoriam fusius hic exponere nolui. Arithmeticae signa, ubi occurrunt, explicantur.
\ENG The symbols pertain either to logic or to arithmetic. The symbols of logic that occur here are about ten in number, although not all are necessary. The use of these symbols and several of their properties are explained in ordinary language in the first part. I did not wish to present their theory more fully here. The symbols of arithmetic are explained as they occur.
\LAT
His notationibus quaelibet propositio formam assumit atque praecisionem, qua in algebra aequationes gaudent, et a propositionibus ita scriptis aliea deducuntur, idque processis qui aequationum resolutioni assimilantur. Hoc caput totius scripti.
\ENG With this notation every proposition assumes the form and precision equations enjoy in algebra, and from propositions so written others may be deduced, by a process which resembles the solution of algebraic equations. That is the chief reason for writing this paper.
\LAT
Sique, confectis signis quibus arithmeticae propositiones scribere possim, in earum tractatione usus sum methodo, quam quia et in aliis studiis sequenda foret, breviter exponam.
\ENG Having made up the symbols with which I can write arithmetical propositions, in treating them I have used a method which, because it is to be followed in later studies, I shall present briefly.
\LAT
Ex arithmeticae signis quae caeteris, una cum logicae signis exprimere licet, ideas significant quas definire possumus. Ita omnia definivi signa, si quatuor excipias, quae in explicationibus \S 1 continentur. Si, ut puto, haec ulterius reduci nequeunt, ideas ipsis expressas, ideis quae prius notae supponuntur, definire non licet.
\ENG Those arithmetical symbols which may be expressed by using others along with symbols of logic represent the ideas we can define. Thus I have defined every symbol, if you except the four which are contained in the explanations of \S 1. If, as I believe, these cannot be reduced further, then the ideas expressed by them may not be defined by ideas already supposed to be known.
\end{translateTwoCol}

\peanoPage{IV} % page-number IV

\begin{translateTwoCol}
Propositiones, quae logicae operationibus a caeteris deducuntur, sunt \emph{theoremata}; quae vero non, \emph{axiomata} vocavi. Axiomata hic sunt novem (\S 1), et signorum, quae definitione carent, proprietates fundamentales exprimunt.
\ENG
Propositions which are deduced from others by the operations of logic are theorems; those for which this is not true I have called axioms. There are nine axioms here (See \S 1), and they express fundamental properties of the undefined symbols.
\LAT
In \S 1-6 numerorum proprietates communes demonstravi; brevitatis causa, demonstrationes praecedentibus similes omisi; demonstrationum communem formam immutare oportet ut logicae signis exprimantur; haec transformatio interdum difficilior est, tamen inde demonstrationis natura clarissime patet.
\ENG
In \S 1-6 I have proved the ordinary properties of numbers; for the sake of brevity, I have omitted proofs which are similar to preceding ones. The ordinary form of proofs has had to be altered in order that they may be expressed with the symbols of logic. This transformation is sometimes rather difficult but the nature of the proof then becomes quite clear. 
\LAT
In sequentibus \S \ varia tractavi, ut huius methodi potentia magis videatur.
\ENG
In the following sections I have treated various things so that the power of the method is better seen.
\LAT
In \S 7 nonnulla theoremata, quae ad numerorum theoriam pertinent, continentur. In \S 8 et 9 rationalium et irrationalium definitiones inveniuntur.
\ENG
In \S 7 are several theorems pertaining to the theory of numbers. In \S 8 and 9 are found the definitions of rationals and irrationals.
\LAT
Denique, in \S 10, theoremata exposui nonnulla, quae nova esse puto, ad entium theoriam pertinentia, quae $\text{cl.}^{\text{mus}}$ Cantor \emph{Punktmenge (ensemble de points)} vocavit.
\ENG
Finally, in \S 10 I have given several theorems, which I believe to be new, pertaining to the theory of those entities which Professor Cantor has called \emph{Punktmenge (ensemble de points)}.
\LAT
In hoc scripto aliorum studiis usus sum. Logicae notationes et propositiones quae in num. II, III et IV continentur, si nonnullas excipias, ad multorum opera, inter quae Boole praecipue, referenda sunt.\endnote{\textbf{Giuseppe Peano's footnote (original):} \\
Boole: \\
\quad \emph{The mathematical analysis of logic ...}, Cambridge, 1847. \\
\quad \emph{The calculus of logic}, Camb. and Dublin Math. Journal, 1848. \\
\quad \emph{An investigation of the laws of thought ...}, London, 1854. \\
E. Schr\"oder: \\
\quad \emph{Der Operationskreis des Logikkalculus}, Leipzig, 1877.

Ipse iam nonnulla quae ad logicam pertinent tractavit in praecedenti opera. \\

\quad \emph{Lehrbuch der Arithmetik und Algebra ...}, Leipzig, 1873.

Boole e Schr\"oder theorias brevissime exposui in meo libro \emph{Calcolo geometrico ...}, Torino, 1888. \\
Vide: \\
\quad C. S. Pierce, \emph{On the Algebra of logic}; American Journal, III, 15; VII, 180. \\
\quad Jevons, \emph{The principles of science}, London, 1883. \\
\quad Mc.Coll., \emph{The calculus of equivalent statements}, Proceedings of the London Math. Society, 1878, Vol. IX, 9. Vol X, 16.}
\ENG
In this paper I have used the research of others. The notations and propositions of logic which are contained in numbers II, III, and IV, with some exceptions, represent the work of many, among them Boole especially.\endnote{\textbf{Giuseppe Peano's footnote (translated):} \\
Boole: \\
\quad \emph{The mathematical analysis of logic ... (Cambridge, 1847.)} \\
\quad `The calculus of logic,' \emph{Camb. and Dublin Math. J.}, 3 (1848), 193-98. \\
\quad \emph{An investigation of the laws of thought}  ... (London, 1854). \\
E. Schr\"oder: \\
\quad \emph{Der Operationskreis des Logikkalculus} (Leipzig, 1877).

He had already treated several matters pertaining to logic in a preceding work. \\

\quad \emph{Lehrbuch der Arithmetik und Algebra} ... (Leipzig, 1873).

I gave a very brief presentation of the theories of Boole and Schr\"oder in my book \emph{Calcolo geometric} etc. (Torino, 1888). \\
Cf: \\

\quad C. S. Pierce, `On the Algebra of logic,' \emph{American J. Math.}, 3 (1880), 15-57; 7 (1885), 180-202. \\
\quad Jevons, \emph{The principles of science} (London, 1883). \\
\quad Mc.Coll., `The calculus of equivalent statements,' \emph{Proc. London Math. Soc.}, 9 (1878), 9-20; 10 (1878), 16-28.}
\end{translateTwoCol}

\peanoPage{V} % page-number V

\begin{translateTwoCol}
Signum $\smallIn$, quod cum signo $\C$ confundere non licet, inversionis in logica applicationes, et paucas alias institui conventiones, ut ad exprimendam quamlibet propositionem pervenirem.
\ENG
The symbol $\in$, which must not be confused with the symbol $\subset$, applications of the inverse in logic, and a few other conventions, I have adopted so that I could express any proposition whatever.
\LAT
In arithmeticae demonstrationibus usus sum libro: H. Grassmann, \emph{Lehrbuch der Arithmetik}, Berlin 1861.
\ENG
In the proofs of arithmetic I used the book H. Grassmann, \emph{Lehrbuch der Arithmetik} (Berlin 1861).
\LAT
Utilius quoque mihi fuit recens scriptum: R. Dedekind, \emph{Was sind und was sollen die Zahlen}; Braunschweig, 1888, in quo quaestiones, quae ad numerorum fundamenta pertinent, acute examinantur.
\ENG
Also quite useful to me was the recent work by R. Dedekind, \emph{Was sind und was sollen die Zahlen} (Braunschweig, 1888), in which questions pertaining to the foundations of numbers are acutely examined.
\LAT
Hic meus libellus ut novae methodi specimen habendus est. Hisce notationibus innumeras alias propositiones, ut quae ad rationales et irrationales pertinent, enunciare et demonstrare possumus. Sed, ut aliae theoriae tractentur, nova signa, quae nova indicant entia, instituere necesse est. Puto vero his tantum logicae signis propositiones cuiuslibet scientiae exprimi posse, dummodo adiungantur signa quae entia huius scientiae representant.
\ENG
My booklet should be taken as a sample of this new method. With these notations we can state and prove innumerable other propositions, such as those which pertain to rationals and irrationals. But in order to treat other theories, it is necessary to adopt new symbols to indicate new entities. I believe, however, that with only these symbols of logic the propositions of any science can be expressed, so long as the symbols which represent the entities of the science are added.
\end{translateTwoCol}

\peanoPage{VI} % page-number VI

\begin{translateTwoCol}
\ENG
\commentary{Peano's mathematics has been translated to modern symbols and language.  This is easiest for the casual reader.  We caution expert readers that terms like ``set'' were new in Peano's era and that Peano's usage may differ in subtle ways from modern usage.  E.g., Peano use the concept of a power set, but without a fully developed theory for it.}  
\LAT  
\centering
\phantomsection
\addcontentsline{toc}{chapter}{Table of symbols}
\peanoHeadingMedium{Signorum tabula}
\ENG
\peanoHeadingMedium{Table of symbols}
\end{translateTwoCol}

\begin{translateTwoCol}
\centering
\textbf{Logicam signa}
\ENG
\textbf{Symbols of logic}
\end{translateTwoCol}

\begin{translateSixCol}{0.05}{0.35}{0.1}{0.05}{0.35}{0.1}
\raggedright
Signum \s Significatio \s Pag.
\s Symbol \s Meaning \s Pag.
\end{translateSixCol}

\begin{translateSixCol}{0.06}{0.34}{0.1}{0.06}{0.34}{0.1}
\raggedright
P \s propositio \s VII
\s $\prop$ \emph{or} P\# \s proposition \s VII
\s* $\text{K}$ \s classis \s X
\s $\setOfSets$ \s set \s X
\s* $\cap$ \s et \s VII, X
\s $\wedge$ \s and \s VII, X
\s* $\cup$ \s vel \s VIII, X, XI
\s $\vee$ \s or \s VIII, X, XI
\s* $-$ \s non \s VIII, X
\s $\neg$ \s not \s VIII, X
\s* $\abs$ \s absurdum \emph{aut} nihil \s VIII, XI
\s $\bot$ \\ $\varnothing$ \s false \emph{or} \\ \quad nothing \s VIII, XI
\s* $\C$ \s deducitur \emph{aut} continetur \s VIII, XI
\s $\rightarrow$ \\ $\subset$ \s one deduces \emph{or} \\ \quad is contained in \s VIII, XI
\s* = \s est aequalis \s VIII
\s = \s equals \s VIII
\s* $\smallIn$ \s est \s X
\s $\in$ \s is a \s X   %Original translation included ``or, is an element of''
\s* $[ \ ]$ \s inversionis signum \s XI
\s $\{ x | \ldots \}$ \s notation of the inverse \s XI  % Chose to translate ``signum'' as ``notation'' rather than ``symbol''
\s* \such \s qui \emph{vel} $[ \smallIn ]$ \s XII
\s $\ni$ \s such that \emph{or} $[ \in ]$ \s XII
\s* Th \s Theorema \s XVI
\s  Th \s Theorem \s XVI
\s* Hp \s Hypothesis \s
\s  Hp \s Hypothesis \s
\s* Ts \s Thesis \s
\s  Ts \s Thesis \s
\s* L \s Logica \s
\s  L\# \s Logic \s
\end{translateSixCol}

\vspace{1em}
\begin{translateTwoCol}
\centering
\textbf{Arithmeticae signa}
\ENG
\textbf{Symbols of arithmetic}
\end{translateTwoCol}

\begin{translateSixCol}{0.05}{0.35}{0.1}{0.05}{0.35}{0.1}
\raggedright
Signum \s Significatio \s Pag.
\s Symbol \s Meaning \s Pag.
\end{translateSixCol}

\begin{translateTwoCol}
\raggedright
Signa 1, 2, ..., =, $>$, $<$, +, -, $\times$ vulgarem habent significationem. Divisionis signum est /.
\ENG
The symbols 1, 2, ..., =, $>$, $<$, +, -, $*$ have their usual meaning. The symbol of division is /.
\end{translateTwoCol}

\begin{translateSixCol}{0.06}{0.34}{0.1}{0.06}{0.34}{0.1}
\raggedright
$N$ \s numerus integer positivus \s 1
\s $\mathbb{N}$ \s positive integers \s 1
\s* $R$ \s num. rationalis positivus \s 12
\s $\mathbb{Q}^+$ \s postive rational numbers \s 12
\s* $Q$ \s quantitas, \emph{sive} numerus realis positivus \s 16
\s $\mathbb{R}^+$ \s quantity \emph{or} postive real numbers \s 16
\s* $\text{N}p$ \s numerus primus \s 9
\s $\mathbb{P}$ \s prime number \s 9
\s* $\text{M}$ \s maximus \s 6
\s $\max()$ \s maximum \s 6
\s* $\mini$ \s minimus \s 6
\s $\min()$ \s minimum \s 6
\s* $\text{T}$ \s terminus, \emph{vel} limes summus \s 15
\s $\sup()$ \s boundary \emph{or} upper limit \s 15
\s* $\text{D}$ \s dividit \s 9
\s $|$ \s divides \s 9
\s* \reflectbox{D} \s est multiplex \s 9
\s \reflectbox{D} \s is divisible \s 9
\s* $\primeWith$ \s est primus cum \s 6
\s $\primeWith$ \s is prime with \s 6
\end{translateSixCol}

\vspace{1em}
\begin{translateTwoCol}
\centering
\textbf{Signa composita}
\ENG
\textbf{Composite symbols}
\end{translateTwoCol}

\begin{translateSixCol}{0.06}{0.34}{0.1}{0.06}{0.34}{0.1}
\raggedright
$-<$ \s non est minor \s
\s $\nless$ \s is not less than \s
\s* $=\cup>$ \s est aequalis aut maior \s
\s $\geq$ \s is equal to or greater than \s
\s* $\such \D$ \s divisor \s
\s $\such \D$ \s is a divisor \s
\s* $\text{M}\such\D$ \s maximus divisor \s
\s $\text{M}\such\D$ \s is the greatest divisor \s
\end{translateSixCol}

\peanoPage{VII} % page-number VII

\begin{translateTwoCol}
\centering
\phantomsection
\addcontentsline{toc}{chapter}{Notations of logic}
\peanoHeadingMedium{Logicae notationes.}
\ENG
\peanoHeadingMedium{Notations of logic.}
\end{translateTwoCol}

\begin{translateTwoCol}
\centering
\phantomsection
\addcontentsline{toc}{section}{I. Punctuation}
\peanoHeadingSmall{I. De punctuatione.}
\ENG
\peanoHeadingSmall{I. Punctuation.}
\end{translateTwoCol}

\begin{translateTwoCol}
Litteris $a,b,... x,y,... x', y',...$ entia indicamus indeterminata quaecumque. Entia vero determinata signis, sive litteris $\text{P}, \K, \N,...$ indicamus.
\ENG
By the letters $a,b,... A,B,... x,y,... x', y',...$ we indicate any variables. Constants are, however, indicated by the symbols, or rather by the letters, $\mathbb{P}, \mathbb{N},...$  % dropped \K
\LAT
Signa plerumque in eadem linea scribemus. Ut ordo pateat quo ea coniungere oporteat, \emph{parenthesibus} ut in algebra, sive \emph{punctis} $. : \pppNoSpace :\hspace{0.1em}:$ etc. utimur.
\ENG
\irrelavent{Generally we write symbols on the same line. So that it will be clear how they are to be joined, we use \emph{parentheses}, as in algebra, or rather \emph{points} $. : \pppNoSpace :\hspace{0.1em}:$ etc.}
\LAT
Ut formula punctis divisa, intelligatur, primum signa quae nullo puncto seperantur colligenda sunt, postea quae uno puncto, deinde quae duobus punctis, etc.
\ENG
\irrelavent{So that a formula divided by points may be understood, first the symbols which are not separated by points are taken together, then those separated by one point, then those by two points, etc.}
\LAT
Ex. g. sint $a,b,c,...$ signa quaecumque. Tunc $ab$ $.$ $cd$ significat $(ab)(cd)$; et $ab$ $.$ $cd:ef$ $.$ $gh$ $\pppNoSpace$ $k$ significat $(((ab)(cd))((ef)(gh)))k$.
\ENG
\irrelavent{For example, let $a,b,c,...$ be any symbols. Then $ab$ $.$ $cd$ means $(ab)(cd)$; and $ab$ $.$ $cd:ef$ $.$ $gh$ $\pppNoSpace$ $k$ means $\{[(ab)(cd)][(ef)(gh)]\}k$.}
\LAT Punctuationis signa omittere licet si formulae quae diversa punctuatione existerent eundem habeant sensum; vel si una tantum formula, et ipsa quam scribere volumus, sensum habeat.
\ENG \irrelavent{The symbols of punctuation may be omitted if formulas having different punctuation have the same meaning, or if just one formula, that being the one we wish to write, has meaning.}
\LAT Ut ambiguitatis periculum absit, aritmeticae operationum signis $. :$ nunquam utimur.
\ENG \irrelavent{To avoid the danger of ambiguity, we never use $. :$ as symbols of arithmetical operations.}
\LAT Parenthesum figura una est $( \ )$; si in eadem formula, parentheses et puncta occurant, primum quae parenthesibus continentur, colligantur.
\ENG \irrelavent{The figure of parentheses is one $( \ )$; if parentheses and points occur in the same formula, whatever is contained in parentheses is to be gathered first.}
\end{translateTwoCol}

\begin{translateTwoCol}
\centering
\phantomsection
\addcontentsline{toc}{section}{II. Propositions}
\peanoHeadingSmall{II. De propositionibus.}
\ENG
\peanoHeadingSmall{II. Propositions.}
\end{translateTwoCol}

\begin{translateTwoCol}
Signo $\text{P}$ significatur \emph{propositio}.
\ENG
The symbol $\prop$ means \emph{proposition}.
\LAT
Signum $\cap$ legitur \emph{et}. Sint $a, b$ propositiones; tunc $a \cap b$ est simultanea affirmatio propositionum $a, b$. Brevitatis causa, loco $a \cap b$ vulgo scribemus $a$ $b$.
\ENG
The symbol $\wedge$ is read \emph{and}. Let $a, b$, be propositions; then $a \wedge b$ is the simultaneous affirmation of the propositions $a, b$. For the sake of brevity, instead of $a \wedge b$, we ordinarily write $a b$.
\end{translateTwoCol}

\peanoPage{VIII} % page-number VIII

\begin{translateTwoCol}
Signum $-$ legitur \emph{non}. Sit $a$ quaedam $\text{P}$; tunc $-a$ est negatio propositionis $a$.
\ENG The symbol $\neg$ is read \emph{not}. Let $a$ be a $\prop$; then $\neg a$ is the negation of the proposition $a$.
\LAT Signo $\cup$ legitur \emph{vel}. Sint $a,b$ propositiones; tunc $a \cup b$ idem est ac $-:-a$ $.$ $-b$.
\ENG The symbol $\vee$ is read \emph{or}. Let $a,b$ be propositions; then $a \vee b$ is the same as $\neg[(\neg a) \wedge (\neg b)]$.
\LAT \text{[}Signo $V$ significatur \emph{verum}, sive \emph{identitas}; sed hoc signo numquam utimur\text{]}.  % TODO: The math V here should be upright, not slanted
\ENG \irrelavent{The symbol $\top$  means \emph{true}, or \emph{identity}, but we never use this symbol.}
\LAT Signum  $\abs$ significat \emph{falsum}, sive \emph{absurdum}.
\ENG The symbol $\bot$ means \emph{false}, or \emph{absurd}.
\LAT \text{[}Signum C significat \emph{est consequentia}; ita $b$ C $a$ legitur $b$ \emph{est consequentia propositionis} $a$. Sed hoc signo nunquam utimur\text{]}.
\ENG \irrelavent{The symbol $\leftarrow$ means \emph{is a consequence of}. Thus $b \leftarrow a$ is read $b$ \emph{is a consequence of the proposition} $a$. But we never use this symbol.}
\LAT Signum $\C$ significat \emph{deducitur}; ita $a$ $\C$ $b$ significat quod $b$ C $a$. Si propositiones $a$, $b$ entia indeterminata continent $x$, $y$,..., scilicet sunt inter ipsa entia conditiones, tunc $a$ $\C_{x,y,...}$ $b$ significat: quaecumque sunt $x,y,...,$ a propositione $a$ deducitur $b$. Si vero ambiguitatis periculum absit, loco $\C_{x,y,...}$, scribemus solum $\C$.
\ENG The symbol $\rightarrow$ means \emph{one deduces}; thus $a \rightarrow b$ means the same as $b \leftarrow a$. If the propositions $a,b$ contain the variables $x,y,...$, that is, express conditions on these objects, then $a \xrightarrow[\forall x,y,...]{} b$ means: whatever the $x$, $y$,..., from propositions $a$ one deduces $b$. If indeed there is no danger of ambiguity, instead of $\xrightarrow[\forall x,y,...]{}$, we write only $\rightarrow$.
\LAT Signum = significat \emph{est aequalis}. Sint $a,b$ propositiones; tunc $a=b$ idem significat quod $a$ $\C$ $b$ $.$ $b$ $\C$ $a$; propositio $a=_{x,y,...} b$ idem significat quod $a$ $ \C_{x,y,...} b$ $.$ $b$ $\C_{x,y,...} a$.
\ENG The symbol = means \emph{equals}. Let $a,b$ be propositions; then $a=b$ means the same as $(a \rightarrow b) \wedge (b \rightarrow a)$; proposition $a\underset{\forall x,y,...}=b$ means the same as ($a \xrightarrow[\forall x,y,...]{} b) \wedge (b \xrightarrow[\forall x,y,...]{} a)$.
\end{translateTwoCol}

\begin{translateTwoCol}
\centering
\phantomsection
\addcontentsline{toc}{section}{III. Propositions of logic}
\peanoHeadingSmall{III. Logicae propositiones.}
\ENG
\peanoHeadingSmall{III. Propositions of logic.}
\end{translateTwoCol}

\begin{translateTwoCol}
Sint $a,b,c,...$ propositiones. Tunc erit:
\ENG
Let $a,b,c,...$ be propositions. We have:
\LAT
1. \hspace{0.6cm} $a$ $\C$ $a$
\ENG
1. \hspace{0.6cm} $a \rightarrow a$
\LAT
2. \hspace{0.6cm} $a$ $\C$ $b$ . $b$ $\C$ $c$ : $\C$ : $a$ $\C$ $c$
\ENG
2. \hspace{0.6cm} $[(a \rightarrow b) \wedge (b \rightarrow c)] \rightarrow (a \rightarrow c)$
\LAT
3. \hspace{0.6cm} $a$ = $b$ . = : $a$ $\C$ $b$ . $b$ $\C$ $a$.
\ENG
3. \hspace{0.6cm} $(a=b)=[(a \rightarrow b) \wedge (b \rightarrow a)]$
\LAT
4. \hspace{0.6cm} $a$ = $a$
\ENG
4. \hspace{0.6cm} $a=a$
\LAT
5. \hspace{0.6cm} $a$ = $b$ . = . $b$ = $a$
\ENG
5. \hspace{0.6cm} $(a=b)=(b=a)$
\LAT
6. \hspace{0.6cm} $a$ = $b$ . $b$ $\C$ $c$ : $\C$ . $a$ $\C$ $c$
\ENG
6. \hspace{0.6cm} $[(a=b) \wedge (b \rightarrow c)] \rightarrow (a \rightarrow c)$
\LAT
7. \hspace{0.6cm} $a$ $\C$ $b$ . $b$ = $c$ : $\C$ . $a$ $\C$ $c$
\ENG
7. \hspace{0.6cm} $[(a \rightarrow b) \wedge (b=c)] \rightarrow (a \rightarrow c)$
\LAT
8. \hspace{0.6cm} $a$ = $b$ . $b$ = $c$ : $\C$ . $a$ = $c$
\ENG
8. \hspace{0.6cm} $[(a=b) \wedge (b=c) ] \rightarrow (a=c)$
\LAT
9. \hspace{0.6cm} $a$ = $b$ . $\C$ . $a$ $\C$ $b$
\ENG
9. \hspace{0.6cm} $(a=b) \rightarrow (a \rightarrow b)$
\LAT
10. \hspace{0.67cm} $a$ = $b$ . $\C$ . $b$ $\C$ $a$
\ENG
10. \hspace{0.67cm} $(a=b) \rightarrow (b \rightarrow a)$
\LAT
\hfill
\ENG
\hfill
\LAT
11. \hspace{0.67cm} $ab$ $\C$ $a$
\ENG
11. \hspace{0.67cm} $(a \wedge b) \rightarrow a$
\LAT
12. \hspace{0.67cm} $ab$ = $ba$
\ENG
12. \hspace{0.67cm} $(a \wedge b)=(b \wedge a)$
\LAT
13. \hspace{0.67cm} $a$\thinspace $(bc)$ = $(ab)$\thinspace $c$ = $abc$
\ENG
13. \hspace{0.67cm} $(a \wedge (b \wedge c))=((a \wedge b) \wedge c)=(a \wedge b \wedge c)$
\end{translateTwoCol}

\peanoPage{IX} % page-number IX

\begin{translateTwoCol}
14. \hspace{0.67cm} $aa$ = $a$
\ENG
14. \hspace{0.67cm} $(a \wedge a)=a$
\LAT
15. \hspace{0.67cm} $a$ = $b$ . $\C$ . $ac$ = $bc$
\ENG
15. \hspace{0.67cm} $(a=b) \rightarrow [(a \wedge c)=(b \wedge c)]$
\LAT
16. \hspace{0.67cm} $a$ $\C$ $b$ . $\C$ . $ac$ $\C$ $bc$
\ENG
16. \hspace{0.67cm} $(a \rightarrow b) \rightarrow [(a \wedge c) \rightarrow (b \wedge c)]$
\LAT
17. \hspace{0.67cm} $a$ $\C$ $b$ . $c$ $\C$ $d$ : $\C$ . $ac$ $\C$ $bd$
\ENG
17. \hspace{0.67cm} $[(a \rightarrow b) \wedge (c \rightarrow d)] \rightarrow [(a \wedge c) \rightarrow (b \wedge d)]$
\LAT
18. \hspace{0.67cm} $a$ $\C$ $b$ . $a$ $\C$ $c$ : = . $a$ $\C$ $bc$
\ENG
18. \hspace{0.67cm} $[(a \rightarrow b) \wedge (a \rightarrow c)]=[(a \rightarrow (b \wedge c)]$
\LAT
19. \hspace{0.67cm} $a$ = $b$ . $c$ = $d$ : $\C$ . $ac$ = $bd$
\ENG
19. \hspace{0.67cm} $[(a=b) \wedge (c=d)] \rightarrow [(a \wedge c)=(b \wedge d)]$
\LAT
\hfill
\ENG
\hfill
\LAT
20. \hspace{0.67cm} $-(-a)$ = $a$
\ENG
20. \hspace{0.67cm} $\neg(\neg a) = a$
\LAT
21. \hspace{0.67cm} $a$ = $b$ . = . $-a$ = $-b$.
\ENG
21. \hspace{0.67cm} $(a=b)=[(\neg a)=(\neg b)]$
\LAT
22. \hspace{0.67cm} $a$ $\C$ $b$ . = . $-b$ $\C$ $-a$ =
\ENG
22. \hspace{0.67cm} $(a \rightarrow b)=[(\neg b) \rightarrow (\neg a)]$
\LAT
\hfill
\ENG
\hfill
\LAT
23. \hspace{0.67cm} $a$ $\cup$ $b$ . = $\pppNoSpace$ $-$ : $-a$ . $-b$
\ENG
23. \hspace{0.67cm} $(a \vee b) = \neg[(\neg a) \wedge (\neg b) $
\LAT
24. \hspace{0.67cm} $-(ab)$ = $(-a)$ $\cup$ $(-b)$
\ENG
24. \hspace{0.67cm} $[\neg(a \wedge b)] = [(\neg a) \vee (\neg b)]$
\LAT
25. \hspace{0.67cm} $-(a$ $\cup$ $b)$ = $(-a)$  $(-b)$
\ENG
25. \hspace{0.67cm} $[\neg(a \vee b)] = [(\neg a) \vee (\neg b)]$
\LAT
26. \hspace{0.67cm} $a$ $\C$ . $a$ $\cup$ $b$
\ENG
26. \hspace{0.67cm} $a \rightarrow (a \vee b)$
\LAT
27. \hspace{0.67cm} $a$ $\cup$ $b$ = $b$ $\cup$ $a$
\ENG
27. \hspace{0.67cm} $(a \vee b) = (b \vee a)$
\LAT
28. \hspace{0.67cm} $a$ $\cup$ $(b$ $\cup$ $c)$ = $(a$ $\cup$ $b)$ $\cup$ $c$ = $a$ $\cup$ $b$ $\cup$ $c$
\ENG
28. \hspace{0.67cm} $[a \vee (b \vee c)] = [(a \vee b) \vee c] = (a \vee b \vee c)$
\LAT
29. \hspace{0.67cm} $a$ $\cup$ $a$ = $a$
\ENG
29. \hspace{0.67cm} $(a \vee a)=a$
\LAT
30. \hspace{0.67cm} $a$ $(b$ $\cup$ $c)$ = $ab$ $\cup$ $ac$
\ENG
30. \hspace{0.67cm} $[a \wedge (b \vee c)]=[(a \wedge b) \vee (a \wedge c)]$
\LAT
31. \hspace{0.67cm} $a$ = $b$ . $\C$ . $a$ $\cup$ $c$ = $b$ $\cup$ $c$
\ENG
31. \hspace{0.67cm} $(a=b) \rightarrow [(a \vee c) = (b \vee c)]$
\LAT
32. \hspace{0.67cm} $a$ $\C$ $b$ . $\C$ . $a$ $\cup$ $c$ $\C$ $b$ $\cup$ $c$
\ENG
32. \hspace{0.67cm} $(a \rightarrow b) \rightarrow [(a \vee c) \rightarrow (b \vee c)]$
\LAT
33. \hspace{0.67cm} $a$ $\C$ $b$ . $c$ $\C$ $d$ : $\C$ : $a$ $\cup$ $c$ . $\C$ . $b$ $\cup$ $d$
\ENG
33. \hspace{0.67cm} $[(a \rightarrow b) \wedge (c \rightarrow d)] \rightarrow [(a \vee c) \rightarrow (b \vee d)]$
\LAT
34. \hspace{0.67cm} $b$ $\C$ $a$ . $c$ $\C$ $a$ : = . $b$ $\cup$ $c$ $\C$ $a$
\ENG
34. \hspace{0.67cm} $[(b \rightarrow a) \wedge (c \rightarrow a)] = [(b \vee c) \rightarrow a]$
\LAT
\hfill
\ENG
\hfill
\LAT
35. \hspace{0.67cm} $a-a$ = $\abs$
\ENG
35. \hspace{0.67cm} $[a \wedge \neg a] = \bot$
\LAT
36. \hspace{0.67cm} $a$ $\abs$ = $\abs$
\ENG
36. \hspace{0.67cm} $(a \wedge \bot) = \bot$
\LAT
37. \hspace{0.67cm} $a$ $\cup$ $\abs$ = a
\ENG
37. \hspace{0.67cm} $(a \vee \bot)=a$
\LAT
38. \hspace{0.67cm} $a$ $\C$ $\abs$ . = . $a$ = $\abs$
\ENG
38. \hspace{0.67cm} $(a \rightarrow \bot) = (a = \bot)$
\LAT
39. \hspace{0.67cm} $a$ $\C$ $b$ . = . $a-b$ = $\abs$
\ENG
39. \hspace{0.67cm} $(a \rightarrow b) = [(a \wedge \neg b) = \bot]$
\LAT
40. \hspace{0.67cm} $\abs$ $\C$ $a$
\ENG
40. \hspace{0.67cm} $\bot \rightarrow a$
\LAT
41. \hspace{0.67cm} $a$ $\cup$ $b$ = $\abs$ . = : $a$ = $\abs$ . $b$ = $\abs$
\ENG
41. \hspace{0.67cm} $[(a \vee b)= \bot]=[(a= \bot) \wedge (b= \bot)]$
\LAT
\hfill
\ENG
\hfill
\LAT
42. \hspace{0.67cm} $a$ $\C$ . $b$ $\C$ $c$ : = : $ab$ $\C$ $c$
\ENG
42. \hspace{0.67cm} $[a \rightarrow (b \rightarrow c)] = [(a \wedge b) \rightarrow c]$
\LAT
43. \hspace{0.67cm} $a$ $\C$ . $b$ = $c$ : = . $ab$ = $ac$
\ENG
43. \hspace{0.67cm} $[a \rightarrow (b=c)]=[(a \wedge b)=(a \wedge c)]$
\end{translateTwoCol}

\peanoPage{X} % page-number X

\begin{translateTwoCol}
\quad Sit $\alpha$ quoddam relationis signum (ex. gr. $=$, $\C$), ita ut $a$ $\alpha$ $b$ sit quaedam propositio. Tunc loco $-$ . $a$ $\alpha$ $b$ scribemus $a$ $-$ $\alpha$ $b$; scilicet:
\ENG
\quad Let $\alpha$ be the symbol of some relation (eg., $=, \rightarrow$) so that $a$ $\alpha$ $b$ is a proposition. Then instead of $\neg(a$ $\alpha$ $b)$, we write $a$ $\not\alpha$ $b$. Thus:
\LAT
\hspace{1.06cm} $a$ $-$ = $b$ . = : $-$ . $a$ = $b$
\ENG
\hspace{1.06cm} $(a \not= b) = \neg (a=b)$
\LAT
\hspace{1.06cm} $a$ $-$ $\C$ $b$ . = : $-$ . $a$ $\C$ $b$
\ENG
\hspace{1.06cm} $ (a \not\rightarrow b) = \neg (a \rightarrow b)$
\LAT
\ENG
\commentary{Peano's $=$\scalebox{0.7}{$x$} operator contains a ``for all $x$'', while the $-=$\scalebox{0.7}{$x$} operator contains an ``exists $x$''.}
\LAT
\quad Ita signum $-$ $=$ significat \emph{non est aequalis}. Si propositio $a$ indeterminatum continet $x$, $a$ $-=$\scalebox{0.7}{$x$}\thinspace $\abs$ significat: sunt $x$ quae conditioni $a$ satisfaciunt. Signum $-$ $\C$ significat \emph{non deducitur}.
\ENG
Thus the symbol $\not=$ means \emph{is not equal to}. If the proposition $a$ contains the variable $x, a\underset{\exists x}\neq \bot$ means: there is an $x$ which satisfies condition $a$. The symbol $\not\rightarrow$ means \emph{one does not deduce}.
\LAT
\quad Similter, si $\alpha$ et $\beta$ sunt relationis signa, loco $a$ $\alpha$ $b$, et $a$ $\alpha$ $b$ . $\cup$ . $a$ $\beta$ $b$ scribere possumus $a$ . $\alpha$ $\beta$ . $b$ et $a$ . $\alpha$ $\cup$ $\beta$ . $b$. Ita, si $a$ et $b$ sunt propositiones, formula $a$ . $\C$ $-$ = . $b$ dicit: ab $a$ \emph{deducitur} $b$, sed non vice versa.
\ENG
\quad Similarly, if $\alpha$ and $\beta$ are symbols of relations, instead of $(a$ $\alpha$ $b) \wedge (a$ $\beta$ $b)$, and $(a$ $\alpha$ $b) \vee (a$ $\beta$ $b)$ we may write $a$ $(\alpha \wedge \beta)$ $b$ and $a$ $(\alpha \vee \beta)$ $b$. Thus, if $a$ and $b$ are propositions, the formula $a$ $(\rightarrow \wedge \not=)$ $b$ says: from $a$ \emph{one deduces} $b$, but not vice versa.
\LAT
\hspace{1.06cm} $a$ . $\C$ $-$ = . $b$ : = : $a$ $\C$ $b$ . $b$ $-$ $\C$ $a$
\ENG
\hspace{1.06cm} $[a$ $(\rightarrow \wedge \not=)$ $b] = [(a \rightarrow b) \wedge (b \not\rightarrow a)]$
\LAT
\quad Formulae:
\ENG
\quad Formulas:
\LAT
\hspace{1.06cm} $a$ $\C$ $b$ . $b$ $\C$ $c$ . $a$ $-$ $\C$ $c$ : = $\abs$
\ENG
\hspace{1.06cm} $[(a \rightarrow b) \wedge (b \rightarrow c) \wedge (a \not\rightarrow c)] = \bot$
\LAT
\hspace{1.06cm} $a$ = $b$ . $b$ = $c$ . $a$ $-$ = $c$ : = $\abs$
\ENG
\hspace{1.06cm} $[(a=b) \wedge (b=c) \wedge (a \not= c)] = \bot$
\LAT
\hspace{1.06cm} $a$ $\C$ $b$ . $b$ $\C$ $-$ = $c$ : $\C$ . $a$ $\C$ $-$ = $c$
\ENG
\hspace{1.06cm} $\{(a \rightarrow b) \wedge [b$ $(\rightarrow \wedge \not=)$ $c]\} \rightarrow [a$ $(\rightarrow \wedge \not=)$ $c]$
\LAT
\hspace{1.06cm} $a$ $\C$ $-$ = $b$ . $b$ $\C$ $c$ : $\C$ . $a$ $\C$ $-$ = $c$
\ENG
\hspace{1.06cm} $[a$ $(\rightarrow \wedge \not=)$ $b] \wedge (b \rightarrow c)\} \rightarrow [a$ $(\rightarrow \wedge \not=)$ $c]$
\LAT
\quad Sed his notationibus raro utimur.
\ENG
\quad But we shall rarely use these notations.
\end{translateTwoCol}

\begin{translateTwoCol}
\centering
\phantomsection
\addcontentsline{toc}{section}{IV. Sets}
\peanoHeadingSmall{IV. De classibus.}
\ENG
\peanoHeadingSmall{IV. Sets.}
\end{translateTwoCol}

\begin{translateTwoCol}
\quad Signo $\K$ significatur \emph{classis}, sive entium aggregatio.
\ENG
\quad The symbol $\setOfSets$ means a \emph{set}, or aggregate of entities.
\LAT
\quad Signum $\smallIn$ significat \emph{est}. Ita $a \smallIn b$ legitur $a$ \emph{est quoddam} $b$; $a \smallIn \K$ significat $a$ \emph{est quaedam classis}; $a \smallIn \text{P}$ significat $a$ \emph{est quaedam propositio}.
\ENG
\quad The symbol $\in$ means \emph{is}. Thus $a \in B$ is read $a$ \emph{is (an element of)} $B$; $A \in \setOfSets$ means \emph{$A$ is a set}; $a \in \prop$ means \emph{$a$ is a proposition}.
\LAT
\quad Loco $-(a \smallIn b)$ scribemus $a$ $- \smallIn b$; signum $-\smallIn$ significat \emph{non est}; scilicet:
\ENG
\quad Instead of $\neg(a \in b)$ we shall write $a \not\in b$. The symbol $\not\in$ means \emph{is not}; thus:
\LAT
44. \hspace{0.67cm} $a$ $- \smallIn b$ . = : $-$ . $a \smallIn b$
\ENG
44. \hspace{0.67cm} $(a \not\in b) = [\neg(a \in b)]$
\LAT
\quad Signum $a$, $b$, $c \smallIn m$ significat: $a$, $b$ et $c$ sunt $m$; scilicet:
\ENG
\quad The notation $a, b, c \in m$ means: $a$, $b$, and $c$ are in $m$; thus:  % Chose to translate signum as ``notation'' rather than ``symbol''
\LAT
45. \hspace{0.67cm} $a$, $b$, $c \smallIn m$ . = : $a \smallIn m$ . $b \smallIn m$ . c \smallIn $m$
\ENG
45. \hspace{0.67cm} $[a, b, c \in m] = [(a \in m) \wedge (b \in m) \wedge (c \in m)]$
\LAT
\quad Sit $a$ classis; tunc $-a$ significatur classis indiviuis constituta quae non sunt $a$.
\ENG
\quad Let $A$ be a set. Then $\overline{A}$ means that set made up of individuals that are not in $A$. 
\LAT
46. \hspace{0.67cm} $a \smallIn \K$ . $\C$ : $x \smallIn -$ $a$ . = . $x$ $- \smallIn a$
\ENG
46. \hspace{0.67cm} $A \in \setOfSets \rightarrow [(x \in \overline{A}) = (x \not\in A)]$
\LAT
\quad Sint $a$, $b$ classes; $a$ \scalebox{0.8}[0.6]{$\cap$} $b$, sive $a$ $b$, est classis individuis constituta
\ENG
\quad Let $A, B$ be sets. Then $A \cap B$\irrelavent{, or $A B$,} is the set composed of individuals
\end{translateTwoCol}

\peanoPage{XI} % page-number XI

\begin{translateTwoCol}
quae eodem tempore sunt $a$ et $b$; $a$ $\cup$ $b$ est classis individuis constituta qui sunt $a$ vel $b$.
\ENG
which are at the same time in $A$ and $B$; $A \cup B$ is the set composed of individuals which are in $A$ or $B$.
\LAT
47. \hspace{0.67cm} $a$, $b \smallIn \K$ . $\C$ $\pppNoSpace$ $a$ $x$ \smallIn . $a$ $b$ : = : $x \smallIn a$ . $x \smallIn b$
\ENG
47. \hspace{0.67cm} $(A, B \in \setOfSets) \rightarrow \{[(x \in (A \cap B)] = [(x \in A) \wedge (x \in B)]\}$
\LAT
48. \hspace{0.67cm} $a$, $b \smallIn \K$ . $\C$ $\pppNoSpace$ $a$ $\cup$ $x$ \smallIn . $a$ $\cup$ $b$ : = : $x \smallIn a$ . $\cup$ . $x \smallIn b$
\ENG
48. \hspace{0.67cm} $(A, B \in \setOfSets) \rightarrow \{[(x \in (A \cup B)] = [(x \in A) \vee (x \in B)]\}$
\LAT
\quad Signum $\abs$ indicat classem quae nullum continet individuum. Ita:
\ENG
\quad The symbol $\varnothing$ indicates the set which contains no individuals. Thus:
\LAT
49. \hspace{0.67cm} $a \smallIn \K$ . $\C$ $\pppNoSpace$ $a$ = $\abs$ : = : $x \smallIn a$ . =\scalebox{0.7}{$x$}\thinspace  $\abs$
\ENG
49. \hspace{0.67cm} $A \in \setOfSets \rightarrow \{(A= \varnothing) = [(x \in A) \underset{\forall x}= \bot]\}$ 
\LAT
\quad [Signo V, quod classem ex omnibus individuis constitutam, de quibus quaestio est, indieat, non utimur].
\ENG
\quad \irrelavent{[We shall not use the symbol $\mathbb{U}$, which indicates the set composed of all individuals being considered].}
\LAT
\quad Signum $\C$ significat \emph{continetur}. Ita $a$ $\C$ $b$ significat \emph{classis a continetur in classi b}.
\ENG
\quad The symbol $\subset$ means \emph{is contained}. Thus $A \subset B$ means \emph{the set A is contained in the set B}.
\LAT
50. \hspace{0.67cm} $a$, $b \smallIn \K$ . $\C$ $\pppNoSpace$ $a$ $\C$ $b$ : = : $x \smallIn a$ . $\C$\scalebox{0.7}{$x$}\thinspace . $x \smallIn b$
\ENG
50. \hspace{0.67cm} $(A, B \in \setOfSets) \rightarrow \{(A \subset B)=[(x\in A) \xrightarrow[\forall x]{} (x \in B)]\}$
\LAT
\quad [Formula $b$ \scalebox{0.8}{C} $a$ significare potest \emph{classis b continet classem a}; at signo  \scalebox{0.8}{C} non utiumur].
\ENG
\quad \irrelavent{[The formula $B \supset A$ could mean \emph{the set B contains the set A}, but we shall not use the symbol $\supset$].}
\LAT
\ENG
\commentary{The following paragraph describes ambiguity in Peano's notation.  In translating the expressions to modern notation, the ambiguity has been removed.  (Except for the equal sign, which is shared by sets and propositions.) }
\LAT
Hic signa $\abs$ et $\C$ significationem habent quae paullo a praecedenti differt; sed nulla orietur ambiguitas. Nam si de propositionibus agatur, haec signa legantur \emph{absurdum} et \emph{deducitur}; si vero de classibus, \emph{nihil} et \emph{continetur}.
\ENG
\irrelavent{The symbols $(\bot / \varnothing)$ and $(\rightarrow / \subset)$ have meanings here which are slightly different from the preceding, but no ambiguity will arise, for if propositions are being considered, the symbols are read \emph{absurd} $(\bot)$ and \emph{one deduces} $(\rightarrow)$, but if sets are being considered, they are read \emph{empty} $(\varnothing)$ and \emph{is contained} $(\subset)$.}
\LAT
\quad Formula $a$ = $b$, si $a$ et $b$ sint classes, significat $a$ $\C$ $b$ . $b$ $\C$ $a$. Itaque
\ENG
\raggedright
\quad The formula $A = B$, if $A$ and $B$ are sets, means $(A \subset B) \wedge (B \subset A)$. Thus
\LAT
51. \hspace{0.67cm} $a$, $b \smallIn \K$ . $\C$ $\pppNoSpace$ $a$ = $b$ : = : $x \smallIn a$ . =\scalebox{0.7}{$x$}\thinspace . $x \smallIn b$
\ENG
51. \hspace{0.67cm} $(A, B \in \setOfSets) \rightarrow \{(A = B) = [(x \in A) \underset{\forall x}=  (x \in B)]\}$
\LAT
Propositiones 1...41 quoque subsistunt, si $a$, $b$... classes indicant; praeterea est:
\ENG
Propositions 1-41 also hold if $a,b,...$ indicate sets. In addition, we have:
\LAT
52. \hspace{0.67cm} $a\smallIn b$ . $\C$ . $b \smallIn \K$
\ENG
52. \hspace{0.67cm} $(a \in B) \rightarrow B \in \setOfSets$
\LAT
53. \hspace{0.67cm} $a\smallIn b$ . $\C$ . $b$ $-$ = $\abs$
\ENG
53. \hspace{0.67cm} $(a \in B) \rightarrow (B \not= \varnothing)$
\LAT
54. \hspace{0.67cm} $a \smallIn b$ . $b$ = $c$ : $\C$ . $a \smallIn c$
\ENG
54. \hspace{0.67cm} $[(a \in B) \wedge (B = C)] \rightarrow (a \in C)$
\LAT
55. \hspace{0.67cm} $a \smallIn b$ . $b$ $\C$ $c$ : $\C$ . $a \smallIn c$
\ENG
55. \hspace{0.67cm} $[(a \in B) \wedge (B \subset C)] \rightarrow (a \in C)$
\LAT
\ENG
\commentary{Below, Peano defines a subsets that each contain a single element.}
\LAT
\quad Sit $s$ classis, et $k$ classis quae in $s$ contineatur; tunc dicimus $k$ esse individuum classis $s$, si $k$ ex uno tantum constat individuo. Itaque:
\ENG
\quad Let $A$ be a set, and $B$ be a set which is contained in $A$; then we say that $B$ is an individual of the set $A$, if $B$ consists of only one individual. That is:
\LAT
56. \hspace{0.67cm} $s \smallIn \K$ . $k$ $\C$ $s$ : $\C$ :: $k \smallIn s$ . = $\pppNoSpace$ $k$ $-$ $=$ $\abs$ : $x$, $y \smallIn k$ . $\C$\scalebox{0.7}{$x, y$}\thinspace . $x$ = $y$
\ENG
56. \hspace{0.67cm} $[A \in \setOfSets$ $\wedge$ $( B \subset A)] \rightarrow \Big\{( B \in A) = \big\{(B \not= \varnothing) \wedge \{[(x,y) \in B] \xrightarrow[\forall x,y]{} (x=y)\}\big\}\Big\}$
\end{translateTwoCol}

\begin{translateTwoCol}
\centering
\phantomsection
\addcontentsline{toc}{section}{V. The inverse}
\peanoHeadingSmall{V. De inversione.}
\ENG
\peanoHeadingSmall{V. The inverse.}
\end{translateTwoCol}

\begin{translateTwoCol}
\ENG
\commentary{By inversion, Peano means going backwards from a proposition to a set.  It is not a mathematical inverse; it is ``set-builder notation''. }
\LAT  
\quad Inversionis signum est $\scalebox{1.5}[0.8]{[\scalebox{0.2}{ \ }]}$, eiusque usum in sequenti numero explicabimus. Hic tantum casus particulares exponimus.
\ENG
\quad The notation of the inverse is $\{x | \ldots\}$, and we shall explain its use in the following section. Here we give some particular examples.  % Chose to translate ``signum'' as ``notation'' instead of ``symbol''
\LAT
\quad 1. Sit $a$ propositio, indeterminatum continens $x$; tunc scriptura $[x] \smallIn a$, quae legitur \emph{ea x quibus a}, sive \emph{solutiones}, vel \emph{radices} conditionis $a$, classem significat individuis constitutam, quae conditioni $a$ satisfaciunt. Itaque:
\ENG
\quad 1. Let $a$ be a proposition containing the variable $x$; then the expression $\{ x | a \}$, which is read \emph{those x such that a}, or \emph{solutions}, or \emph{roots} of the condition $a$, indicates the set consisting of individuals which satisfy the condition $a$. That is:
\LAT
57. \hspace{0.67cm} $a \smallIn \text{P}$ . $\C$ : $[x\smallIn]$ $a$ . $\smallIn \K$
\ENG
57. \hspace{0.67cm} $a \in \prop \rightarrow ( \{x | a \} \in \setOfSets )$ 
\LAT
58. \hspace{0.67cm} $a \smallIn \K$ . $\C$ $\pppNoSpace$ $[x\smallIn]$ . $x \smallIn a$ : = $a$
\ENG
58. \hspace{0.67cm} $A \in \setOfSets \rightarrow \{ x | x \in A \} = A $
\LAT
59. \hspace{0.67cm} $a \smallIn \text{P}$ . $\C$ $\pppNoSpace$ $x$ \smallIn . $[x\smallIn]$ $a$ : = $a$
\ENG
59. \hspace{0.67cm}  $a \in \prop \rightarrow (x \in \{x | a \}) = a$
\LAT
\quad Sint $\alpha$, $\beta$ propositiones indeterminatum continentes $x$; erit:
\ENG
\quad Let $\alpha, \beta$ be propositions containing the variable $x$. We will have:
\LAT
60. \hspace{0.67cm} $[x\smallIn]$ ($\alpha$ $\beta$) = ($[x\smallIn]$ $\alpha)$ $([x\smallIn]$ $\beta$)
\ENG
60. \hspace{0.67cm} $\{x | \alpha \wedge \beta\}=\{x|\alpha\} \cap \{x | \beta\}$
\LAT
61. \hspace{0.67cm} $[x\smallIn]$ $-$ $\alpha$ = $-$ $[x\smallIn]$ $\alpha$
\ENG
61. \hspace{0.67cm} $\{x | \neg \alpha \} = \overline{\{x | \alpha\}}$
\LAT
62. \hspace{0.67cm} $[x\smallIn]$ ($\alpha$ $\cup$ $\beta$) = $[x\smallIn]$ $\alpha$ $\cup$ $[x\smallIn]$ $\beta$
\ENG
62. \hspace{0.67cm} $\{ x | \alpha \vee \beta \} = \{ x | \alpha \} \cup \{ x | \beta \}$
\LAT
63. \hspace{0.67cm} $\alpha$ $\C$\scalebox{0.7}{$x$} $\beta$ . = . $[x\smallIn]$ $\alpha$  $\C$ $[x\smallIn]$ $\beta$
\ENG
63. \hspace{0.67cm} $\alpha \xrightarrow[\forall x]{} \beta = \{x | \alpha\} \subset \{x | \beta \}$
\LAT
64. \hspace{0.67cm} $\alpha$ =\scalebox{0.7}{$x$} $\beta$ . = . $[x\smallIn]$ $\alpha$ = $[x\smallIn]$ $\beta$
\ENG
64. \hspace{0.67cm} $(\alpha \underset{\forall x}= \beta) = (\{ x |  \alpha\} = \{x | \beta \} )$
\LAT
\quad 2. Sint $x,y$ entia quacumque; system ex ente $x$ et ex ente $y$ compositum ut novum ens consideramus, et signo $(x,y)$ indicamus; similiterque si entium numerus maior fit. Sit $\alpha$ propositio indeterminata continens $x,y$; tunc $[(x,y)\smallIn]$ $\alpha$ significat classem entibus $(x,y)$ constitutam, quae conditioni $\alpha$ satisfaciunt. Erit:
\ENG
\quad 2. Let $x,y$ be any entities. We shall consider the system composed of the entity $x$ and the entity $y$ as a new entity, and indicate it by the notation $(x,y)$; and similarly if the number of entities becomes larger. Let $\alpha$ be a proposition containing the variables $x,y$; then $\{(x,y) | \alpha \}$ indicates the set of entities $(x,y)$ which satisfy the condition $\alpha$. We have:   % chose to translate ``signum'' as ``notation'' rather than ``symbol''
\LAT
65. \hspace{0.67cm} $\alpha$ $\C$\scalebox{0.7}{$x, y$} $\beta$ . = . $[(x, y)\smallIn]$ $\alpha$  $\C$ $[(x, y)\smallIn]$ $\beta$
\ENG
65. \hspace{0.67cm} $\alpha \xrightarrow[\forall x,y]{} \beta = \{(x,y) | \alpha\} \subset \{(x,y) | \beta \}$
\LAT
66. \hspace{0.67cm} $[(x, y)\smallIn]$ $\alpha$ $-$ = $\abs$ . = $\pppNoSpace$ $[x\smallIn]$ . $[y\smallIn]$ $\alpha$ $-$ = $\abs$ : $-$ = $\abs$
\ENG
66. \hspace{0.67cm} $( \{ (x,y) | \alpha \} \neq \varnothing ) = ( \{ x | \{ y | \alpha \} \neq \varnothing \} \} \neq \varnothing )$   %% TODO: Does this need a \forall x under the middle \neq?
\LAT
\quad 3. Sit $x$ $\alpha$ $y$ relatio inter indeterminata $x$ et $y$ (ex. g. in logica relationes $x = y$, $x$ $- = y$, $x$ $\C$ $y$; in arithmetica $x < y$, $x > y$, etc). Tunc signo $[\smallIn$ $\alpha]$ $y$ ea $x$ indicamus, quae relationi $x$ $\alpha$ $y$ satisfaciunt. Commoditatis causa, loco $[\smallIn]$, signo $\such$ utimur. Ita $\such$ $\alpha$ $y$ . = : $[x$ $\smallIn ]$ . $x$ $\alpha$ $y$, et signum $\such$ legitur \emph{qui}, vel \emph{quae}. Ex. gr. sit $y$ numerus; tunc $\such$ $< y$ classem indicat numeris $x$ compositam qui conditioni $x<y$ satisfaciunt, scilicet, \emph{qui sunt minores} $y$, vel simpliciter \emph{minores} $y$. Similiter, quum signum $\D$ significet \emph{dividit}, vel \emph{est divisor}, formula $\such$D significat \emph{qui dividunt} vel {divisores}. Deducitur $x$ $\smallIn$ $\such$ $\alpha$ $y$ $=$ $x$ $\alpha$ $y$.
\ENG
\quad 3. Let $x$ $\alpha$ $y$ be a relation between the variables $x$ and $y$ (eg. in logic, the relations $x = y$, $x \not= y$, $x \rightarrow y$; in arithmetic, $x < y$, $x > y$, ...). Then the notation $[\in$ $\alpha]$ $y$ denotes the $x$ that satisfy the relation $x \mathbin{\alpha} y$. For the sake of convenience, we use the symbol $\ni$ instead of the notation $[\in]$. Thus, $\mathord{\ni} \mathbin{\alpha} y = \{ x | x \mathbin{\alpha} y \}$, and the symbol $\ni$ is read \emph{the objects that}. For example, let $y$ be a number; then $\ni < y$ denotes the set formed by the numbers $x$ that satisfy the condition $x<y$, that is, \emph{the objects that are smaller than} $y$, or simply \emph{the objects smaller than} $y$. Similarly, if the symbol $|$ means \emph{divides} or \emph{is a divisor of}, the formula $\ni|$ means \emph{the objects that divide} or \emph{the divisors}. It follows that $x \in (\mathord{\ni} \mathbin{\alpha} y) = x \mathbin{\alpha} y$.   % Chose in multiple places to translate ``signum'' as ``notation'' instead of ``symbol''
\LAT
\quad 4. Sit $\alpha$ formula indeterminate continens $x$. Tunc scriptura $x' [x] \alpha$, quae legitur $x'$ \emph{loco} $x$ \emph{in} $\alpha$ \emph{substituto}, formulam indicat quae obtinetur si in $\alpha$, loco $x$, $x'$ legimus. Deducitur $x [x] \alpha = \alpha$.
\ENG
\quad 4. Let $\alpha$ be a formula containing the variable $x$. Then the expression $\alpha [ x := x' ]$, which is read $x'$ \emph{being substituted for} $x$ \emph{in} $\alpha$, denotes the formula obtained if, in $\alpha$, we read $x'$ for $x$. It follows that $\alpha[x := x] = \alpha$.
\LAT
\quad 5. Sit $\alpha$ formula, quae indeterminata $x,y,...$ continet. Tunc
\ENG
\quad 5. Let $\alpha$ be a formula that contains the variables $x,y,...$ Then
\LAT
\hspace{1.06cm} $(x',y',...)$ $[x,y,...]$ $\alpha$,
\ENG
\hspace{1.06cm} $\alpha[x:=x', y:=y', \ldots]$,
\end{translateTwoCol}

\peanoPage{XIII} % page-number XIII

\begin{translateTwoCol}
quae legitur $x'y',...$ \emph{loco} $x,y,...$ \emph{in} $\alpha$ \emph{substitutis}, formulam indicat quae obtinetur si in $\alpha$ loco $x,y,...$, litterae $x'y',...$ scribantur. Deducitur $(x,y)$ $[x,y]$ $\alpha = \alpha$.
\ENG
which is read $x',y',...$ \emph{being substituted for} $x,y,...$ in $\alpha$, denotes the formula obtained if, in $\alpha$, the letters $x',y',...$ are written for $x,y,...$ It follows that $\alpha[x:=x,y:=y] = \alpha$.
\end{translateTwoCol}

\begin{translateTwoCol}
\centering
\phantomsection
\addcontentsline{toc}{section}{VI. Functions}
\peanoHeadingSmall{VI. De functionibus.}
\ENG
\peanoHeadingSmall{VI. Functions.}
\end{translateTwoCol}

\begin{translateTwoCol}
\ENG
\commentary{Peano has a very structural or syntactical version of a function.  There is no parameter.  A presymbol function (``functionis praesignum'') is text like ``2 + `` where if you append a number to the end, you have a valid expression.   A postsymbol function would be ``+ 2''.} 
\LAT  
Logicae notationes quae praecedunt exprimendae cuilibet arithmeticae propositioni sufficiunt, iisdemque tantum utimur. Hic notationes alias nonnullas breviter explicamus, quae utiles fieri possunt.
\ENG
The symbols of logic introduced above suffice to express any proposition of arithmetic, and we shall only these.  We explain here briefly some other symbols that may be useful.
\LAT
Sit $s$ quaedam classis; supponimus aequalitatem inter entia systematis $s$ definitam, quae conditionibus satisfaciat:
\ENG
Let $S$ be a set; we assume that equality is defined between the elements of the system $S$ so as to satisfy the conditions:
\LAT
\hspace{1.06cm} $a=a$
\ENG
\hspace{1.06cm} $a=a$.
\LAT
\hspace{1.06cm} $a=b$ $.$ $=$ $.$ $b=a$
\ENG
\hspace{1.06cm} $(a=b)$ $=$ $(b=a)$
\LAT
\hspace{1.06cm} $a=b$ . $b=c$ $:$ $\C$ $.$ $a=c$
\ENG
\hspace{1.06cm} $[(a=b)$ $\wedge$ $(b=c)]$ $\rightarrow$ $a=c$
\LAT
Sit $\upvarphi$ signum, sive signorum aggregatus, ita ut si $x$ est ens classis $s$, scriptura $\upvarphi$ $x$ novum indicet ens; supponimus quoque aequalitatem inter entia $\upvarphi$ $x$ definitam; et si $x$ et $y$ sunt entia classis $s$, et est $x=y$, supponimus deduci posse $\upvarphi$ $x$ $=$ $\upvarphi$ $y$. Tunc signum $\upvarphi$ dicitur esse \emph{functionis praesignum in classi} $s$, et scribemus $\upvarphi$ $\such$ $F'$ $s$.
\ENG
Let $\upvarphi$ be a symbol or an aggregate of symbols such that, if $x$ is an element of the set $S$, the expression $\upvarphi$ $x$ denotes a new object; we assume also that equality is defined between the objects $\upvarphi$ $x$; further, if $x$ and $y$ are elements of the set $S$ and if $x = y$, we assume it is possible to deduce $\upvarphi$ $x = \upvarphi$ $y$. Then the symbol $\upvarphi$ is said to be a \emph{function presymbol in the set S}, and we write $\upvarphi$ $\ni$ $F'$ $S$.
\LAT
\hspace{1.06cm} $s \smallIn \K$ $.$ $\C$ $:\hspace{0.1em}:$ $\upvarphi$ $F'$ $s$ $.$ $=$ $\pppNoSpace$ $x,y$ $\smallIn$ $s$ $.$ $x$ $=$ $y$ $:$ $\C_{x,y}$ $.$ $\upvarphi x$ $=$ $\upvarphi y$
\ENG
\todo
\LAT
Verum si, cum sit $x$ quodlibet ens classis $s$, scriptura $x\upvarphi$ novum indicet ens, et, ex, $x$ $=$ $y$ deducitur $x\upvarphi$ $=$ $y\upvarphi$, tunc dicimus $\upvarphi$ esse \emph{functionis postsignum in classi} $s$ et scribemus $\upvarphi$ $\smallIn$ $s$ $'F$.
\ENG
If, $x$ being any element of the set $S$, the expression $x\upvarphi$ denotes a new object and $x\upvarphi = y\upvarphi$ follows from $x=y$, then we say that $\upvarphi$ is a \emph{function postsymbol in the set S}, and we write $\upvarphi$ $\in$ $S$ $'F$.
\LAT
\hspace{1.06cm} $s \smallIn \K$ $.$ $\C$ $:\hspace{0.1em}:$ $\upvarphi$ $s$ $'F$ $.$ $=$ $\pppNoSpace$ $x,y$ $\smallIn$ $s$ $.$ $x$ $=$ $y$ $:$ $\C_{x,y}$ $.$ $x\upvarphi$ $=$ $y\upvarphi$
\ENG
\todo
\LAT
\emph{Exempla}. Sit $a$ numerus; tunc $a$ $+$ est functionis praesignum in numerorum classe, et $+$ $a$ est functionis postsignum; quicumque enim est numerus $x$, formulae $a$ $+$ $x$ et $x$ $+$ $a$ novos indicant numeros, et ex $x$ $=$ $y$ deducitur $a$ $+$ $x$ $=$ $a$ $+$ $y$, et $x$ $+$ $a$ $=$ $y$ $+$ $a$. Itaque
\ENG
\emph{Examples}. Let $a$ be a number; then $a+$ is a function presymbol in the set of numbers, and $+a$ is a function postsymbol; for any number $x$, formulas $a+x$ and $x+a$ denote new numbers; $a+x=a+y$ and $x+a=y+a$ follow from $x=y$. Thus
\LAT
\hspace{1.06cm} $a$ $\smallIn \N . \C : a + . \smallIn .$ $F' \N$
\ENG
\todo
\LAT
\hspace{1.06cm} $a$ $\smallIn \N .$ $\C$ $:$ $+$ $a$ $.$ $\smallIn$ $.$ $N$ $'F$
\ENG
\todo
\LAT
\hfill
\ENG
\hfill
\LAT
Sit $\upvarphi$ functionis praesignum in classe $s$. Tunc $[\upvarphi] y$ classem significat iis $x$ constitutam, quae conditioni $\upvarphi x$ $=$ $y$ satisfaciunt; scilicet:
\ENG
Let $\upvarphi$ be a function presymbol in the set $S$. Then $[\upvarphi] y$ denotes the set composed of $x$ that satisfy the condition $\upvarphi x = y$; that is,
\LAT
\emph{Def}. \hspace{0.25cm} $s \smallIn \K$ $.$ $\upvarphi$ $\smallIn$ $F'$ $s$ $:$ $\C$ $:$ $[\upvarphi]y$ $.$ $=$ $.$ $[x \smallIn]$ $(\upvarphi x$ $=$ $y)$
\ENG
\emph{Def}. \hspace{0.25cm} \todo
\end{translateTwoCol}

\peanoPage{XIV} % page-number XIV

\begin{translateTwoCol}
Classis $[\upvarphi] y$ vel unum vel plura, vel etiam nullum individuum continere potest. Erit:
\ENG
The set $[\upvarphi] y$ may contain one or several individuals, or none at all. We have:
\LAT
\hspace{1.06cm} $s \smallIn \K$ $.$ $\upvarphi$ $\smallIn$ $F'$ $s$ $:$ $\C$ $:$ $y$ $=$ $\upvarphi$ $x$ $.$ $=$ $.$ $x$ $\smallIn$ $[\upvarphi] y$
\ENG
\todo
\LAT
Si vero $\upvarphi y$ uno tantum constat individuo, erit $y$ $=$ $\upvarphi x$ $.$ $=$ $.$ $x$ $=$ $[\upvarphi] y$
\ENG
But if $\upvarphi y$ consists of just one individual, we have $(y$ $=$ $\upvarphi x)$ $=$ $(x$ $=$ $[\upvarphi] y)$
\LAT
Sit $\upvarphi$ functions postsignum; similiter ponimus:
\ENG
Let $\upvarphi$ be a function postsymbol; we write similarly:
\LAT
\hspace{1.06cm} $s \smallIn \K$ $.$ $\upvarphi$ $\smallIn$ $s$ $'F$ $:$ $\C$ $\pppNoSpace$ $y$ $| \upvarphi |$ $=$ $| x\smallIn |$ $(x \upvarphi = y)$.
\ENG
\todo
\LAT
Signum $[$ $]$ dicitur \emph{inversionis signum}, eiusque usus nonullos in logica iam exposuimus. Nam si $\alpha$ est propositio indeterminatum continens $x$, atque $A$ est classis individuis $x$ composita quae conditioni $\alpha$ satisfaciunt, erit $x$ $\smallIn$ $a$ $.$ $=$ $\alpha$, tunc $a$ $=$ $[x$\smallIn$]$ $\alpha$, ut in V, i.
\ENG
The notation $\{ x | \ldots \}$ is called \emph{notation of the inverse}, and we have already presented some of its uses in logic. If $\alpha$ is a proposition containing the variable $x$ and $A$ is a set composed of the individuals $x$ that satisfy the condition $\alpha$, we have $x$ $\in$ $A$ $=$ $\alpha$, and then $A$ $=$ $\{ x | \alpha \}$, as in V, i. % chose to translate ``signum'' as ``notation'' rather than ``symbol''
\LAT
\ENG
\commentary{In the following, Peano relies on the syntax of his substitution operator to create a presymbol function.  The modern substitution operator has a different syntax and therefore doesn't work properly.}
\LAT
Sit $\alpha$ formula indeterminate continens $x$, sitque $\upvarphi$ functionis praesignum, quod litterae $x$ praepositum, formulam $\alpha$ gignat; scilicet sit $\alpha$ $=$ $\upvarphi$ $x$; tunc erit $\upvarphi$ $=$ $\alpha [x]$, et si $x'$ est novum ens, erit $\upvarphi x'$ $=$ $\alpha [x] x'$, scilicet, si $\alpha$ est formula indeterminatum continens $x$, tunc $\alpha [x] x'$ significat id quod obtinetur si in $\alpha$, loco $x, x'$ ponatur.
\ENG
Let $\alpha$ be a formula containing the variable $x$ and let $\upvarphi$ be a function presymbol that yields the formula $\alpha$ when written before the letter $x$; that is, let $\alpha = \upvarphi x$; then we have $\upvarphi$ $=$ $\alpha [x := ? ]$, and if $x'$ is a new object, we have $\upvarphi x'$ $=$ $\alpha [x := x']$, that is, if $\alpha$ is a formula containing the variable $x$, then $\alpha[x := x']$ means what is obtained when, in $\alpha$, we put $x'$ for $x$.
\LAT
\ENG
\commentary{It is unclear how Peano's substitution syntax works in the following statements.} 
\LAT
Similiter, sit $\alpha$ formula indeterminate continens $x$, sitque $\upvarphi$ functionis postsignum, ut $x \upvarphi = \alpha$; deducitur $\upvarphi = [x] \alpha$; tunc, si $x'$ est novum ens, erit $x' \upvarphi = x' [x] \alpha$, scilicet $x' [x] \alpha$ rursum indicat id quod obtinetur si in $\alpha$, loco $x, x'$ legatur, ut in V, 4.
\ENG
Similarly, let $\alpha$ be a formula containing the variable $x$ and let $\upvarphi$ be a function postsymbol, such that $x\upvarphi = \alpha$; it follows that $\upvarphi = \alpha[x := ?]$. Then, if $x'$ is a new object, have $x' \upvarphi = \alpha[x := x']$; that is, $x'[x := \alpha]$ again denotes what is obtained, when, in $\alpha$, read $x'$ for $x$, as in V, 4..
\LAT
Alios quoque usus in logica signum $[ ]$ habere potest, quos breviter esponimus, quum ipsis non utamur. Sint $a$ et $b$ duae classes; tunc $[a \cap ]b$ sive $b[ \cap a]$ classes indicat $x$, quae conditioni $b = a \cap x$, sive $b = x \cap a$ satisfaciunt. Si $b$ in $a$ non continetur, nulla classis huic conditioni satifacit; si $b$ in $a$ continetur, signum $b [ \cap a ]$ omnes indicat classes quae $b$ continent atque in $b \cup - a$ continentur.
\ENG
\irrelavent{The symbol $[ ]$ can have other uses in logic, which we present only briefly, since we shall not use it in these ways. Let $A$ and $B$ be two sets; then $[A \cap ]B$ or $B[ \cap A]$ denotes the sets $X$ that satisfy the condition $B = A \cap X$, or $B = X \cap A$. If $B$ is not contained in $A$, no set satisfies this condition; if $B$ is contained in $A$, the notation $B [ \cap A ]$ denotes all sets that contain $B$ and are contained in $B \cup  \overline{A}$.} % Chose to translate second ``signum'' as ``notation'' rather than ``symbol''.  May also want to use it on first one.
\LAT
In Arithmetica, sint $a, b$ numeri; tunc $[b + a]$ sive $[a +] b$ numerum indicat $x$, qui conditioni $b = x + a$, sive $b = a + x$ satisfacit, nempe $b - a$. Similiter erit $b [ \times a ] = [a \times ] b = b / a$. Et in analysi hoc signum usuvenire potest; itaque
\ENG
\irrelavent{In arithmetic, let $a$ and $b$ be numbers; then $[b + a]$ or $[a +] b$ denotes the number $x$ that satisfies the condition $b = x + a$, or $b = a + x$, that is $b - a$. Similarly we have $b [ \times a ] = [a \times ] b = b / a$. This notation can even find a use in analysis; thus} % Chose to translate ``signum'' as ``notation'' rather than ``symbol''.
\end{translateTwoCol}

\columnratio{0.30, 0.20, 0.30, 0.20}
\begin{paracol}{4}
\hspace{0.54cm} $y = \sin$ $x$ $.$ = $.$ $x \smallIn$ $[\sin]$ $y$
\s
(loco $x = \mbox{arc}$ $\sin$ $y).$
\s
\todo
\s
\todo
\s
\hspace{0.54cm} $d F (x) = f(x) dx$ $.$ $=$ $.$ $F$ $(x)\smallIn$ $[d]$ $f(x)dx$
\s
(loco $F(x) = \int{f(x)dx}).$ 
\s
\todo
\s
\todo
\end{paracol}

\begin{translateTwoCol}
\hfill
\ENG
\hfill
\LAT
Sit rursum $\upvarphi$ functionis praesignum in classi $s$, sitque $k$ classis
\ENG
Let $\upvarphi$ again be a function presymbol in a set $S$ and let $C$ be a set
\end{translateTwoCol}

\peanoPage{XV} % page-number XV

\begin{translateTwoCol}
in $s$ contenta; tunc $\upvarphi k$ classem indicat omnibus $\upvarphi x$ compositam, ubi $x$ sunt entia classis $k$; scilicet
\ENG
contained in $S$; then $\upvarphi C$ denotes the set consisting of all $\upvarphi x$, where the $x$ are the elements of set $C$; that is
\LAT
\emph{Def.} \hspace{0.25cm} $s \smallIn \K$ $.$ $k \smallIn \K$ $.$ $k$ $\C$ $s$ $.$ $\upvarphi$ $\smallIn$ $F'$ $s$ $:$ $\C$ $.$ $\upvarphi$ $k$ $=$ $[y $\smallIn$]$ $(k$ $.$ $[\upvarphi]y$ $:$ $-$ $=$ $\abs$)
\ENG
\emph{Def}. \hspace{0.25cm} \todo
\LAT
Sive \hspace{0.25cm} $s \smallIn \K$ $.$ $k \smallIn \K$ $.$ $k$ $\C$ $s$ $.$ $\upvarphi$ $\smallIn$ $F'$ $s$ $:$ $\C$ $.$ $\upvarphi$ $k$ $=$ $[y $\smallIn$]$ $([x\smallIn]$ $:$ $x$ $\smallIn$ $k$ $.$ $[\upvarphi]x$ $=$ $y$ $\pppNoSpace$ $-$ $=$ $\abs$)
\ENG
Or \hspace{0.25cm} \todo
\LAT
\emph{Def.} \hspace{0.25cm} $s \smallIn \K$ $.$ $k \smallIn \K$ $.$ $k$ $\C$ $s$ $.$ $\upvarphi$ $\smallIn$ $s$ $'F$ $:$ $\C$ $.$ $k$ $\upvarphi$ $=$ $[y $\smallIn$]$ $(k$ $.$ $y$ $[\upvarphi]$ $:$ $-$ $=$ $\abs$)
\ENG
\emph{Def.} \hspace{0.25cm} \todo
\LAT
Itaque, si $\upvarphi$ $\smallIn$ $F' s$, tunc $\upvarphi$ $s$ classem indicat omnibus $\upvarphi$ $x$ constitutam, ubi $x$ sint entia classis $s$. Erit:
\ENG
Thus, if $\upvarphi$ $\in$ $F' S$, then $\upvarphi$ $S$ denotes the set composed of all $\upvarphi$ $x$, where the $x$ are elements of the set $S$. We have:
\LAT
\hspace{1.06cm} $s \smallIn \K$ . $\upvarphi \smallIn F' s$ . $y \smallIn \upvarphi s : \C : \upvarphi [\upvarphi] y = y$
\ENG
\todo
\LAT
\hspace{1.06cm} $s \smallIn \K$ . $a, b \smallIn \K$ . $a$ $\C$ $s$ . $b$ $\C$ $s$ . $\upvarphi \smallIn F' s : \C$ . $\upvarphi (a \cup b) = (\upvarphi a) \cup (\upvarphi b)$
\ENG
\todo
\LAT
\hspace{1.06cm} $s \smallIn \K$ . $\upvarphi \smallIn F' s : \C$ . $\upvarphi \abs = \abs$
\ENG
\todo
\LAT
\hspace{1.06cm} $s \smallIn \K$ . $a,b \smallIn \K$ . $b$ $\C$ $s$ . $a$ $\C$ $b$ . $\upvarphi \smallIn F' s : \C$ . $\upvarphi a$ $\C$ $\upvarphi b$
\ENG
\todo
\LAT
\hspace{1.06cm} $s \smallIn \K$ . $a,b \smallIn \K$ . $a$ $\C$ $s$ . $b$ $\C$ $s$ . $\upvarphi \smallIn F' s : \C$ . $\upvarphi (ab)$ $\C$ $(\upvarphi a)(\upvarphi b)$
\ENG
\todo
\LAT
Sit $a$ quaedam classis; tunc $a \cap \K$, sive $\K \cap a$, sive $\K a$, classes omnes indicat formae $a \cap x$, sive $x \cap a$, $x a$, ubi $x$ est classis quacumque; scilicet $\K a$ indicat classes quae in $a$ continentur. Formula $x \smallIn \K a$ idem significat quod $x \smallIn \K$ . $x$ $\C$ $a$. Hac conventione quandoque utimur; ita $\K N$ isgnificat \emph{numerorum classem}.
\ENG
Let $A$ be a set; then $A \cap \setOfSets$, or $\setOfSets \cap A$, or $\setOfSets A$, denotes all sets of the form $A \cap X$, or $X \cap A$,\irrelavent{ $X A$,} where $X$ is any set; that is $\setOfSets A$ denotes the sets that are contained in $A$. The formula $X \in \setOfSets A$ means the same as  $X \in \setOfSets \wedge X \subset A$. We shall sometimes use this convention; thus $\setOfSets \mathbb{N}$ means \emph{a set of numbers}.
\LAT
\ENG
\commentary{Peano's natural numbers start at 1. This is why in the following text, $a + \mathbb{N}$ is the set of numbers ``greater than $a$'', rather than ``greater than or equal to $a$''.}
\LAT
Similiter, si $a$ est classis, $\K \cup a$ indicat classes quae $a$ continent. Sit $a$ numerus; tunc $a + N$, sive $N + a$, \emph{numeros} indicat \emph{numero a maiores}; $a \times N$, sive $N \times a$, sive $N a$ indicat \emph{multiplices numeri a}; $a^N$ indicat \emph{potestas numeri} $a$; $N^2$, $N^3$, ... indicat \emph{numeros quadratos}, vel \emph{numeros cubos},...
\ENG
Similarly, if $A$ is a set, $\setOfSets \cup A$ indicates the sets that contain $A$. Let $a$ be a number; then $a + \mathbb{N}$, or $\mathbb{N} + a$, denotes \emph{the numbers greater than the number} $a$; $a \times \mathbb{N}$, or $\mathbb{N} \times a$, or $\mathbb{N} a$ denotes \emph{the multiples of the number} $a$; $a^\mathbb{N}$ denotes \emph{the powers of the number} $a$; $\mathbb{N}^2$, $\mathbb{N}^3$, ... denote \emph{the squares}, \emph{the cubes},...
\LAT
\hfill
\ENG
\hfill
\LAT
Functional signorum aequalitatem, productum, potestas, ita definire licet:
\ENG
Equality, product and powers can be defined thus for function symbols:
\LAT
\emph{Def.} \hspace{0.25cm} $s \smallIn \K$ . $\varphi, \psi \smallIn F' s :$ $\C$ $\pppNoSpace$ $\varphi = \psi$ : $=$ : $x \smallIn s$ . $\C$ . $\varphi x = \psi x$
\ENG
\emph{Def.} \hspace{0.25cm} \todo
\LAT
\emph{Def.} \hspace{0.25cm} $s \smallIn \K$ . $\varphi \smallIn F' s$ $.$ $\psi \smallIn F' \varphi s$ . $x \smallIn s$ : $\C$ . $\psi \varphi x = \psi (\varphi x)$
\ENG
\emph{Def.} \hspace{0.25cm} \todo
\LAT
Itaque, in definitionis hypothesi, erit $\psi \varphi$ novum functionis praesignum; idque \emph{productum signorum} $\psi$ \emph{et} $\varphi$ vocatur.
\ENG
Thus, if we assume this definition, we have the new function presymbol $\psi \varphi$; it is called the \emph{product of the symbols} $\psi$ and $\varphi$.
\LAT
Similiterque, si $\varphi$, $\psi$ sunt functionis postsigna.
\ENG
Similarly if $\varphi$, $\psi$ are function postsymbols.
\LAT
Haec valet propositio:
\ENG
The following proposition holds:
\LAT
\hspace{1.06cm} $s \smallIn \K$ . $\varphi \smallIn F' s$ . $\varphi s$ $\C$ $s : \C : \varphi \varphi s$ $\C$ $s$ . $\varphi \varphi \varphi s$ $\C$ $s$. etc.
\ENG
\todo
\LAT
Funcitones $\varphi \varphi, \varphi \varphi \varphi,...$ \emph{iteraiae} vocantur, et communiter signis $\varphi^2, \varphi^3,...$ indicantur, ut operationis $\varphi$ potestates.
\ENG
The functions $\varphi \varphi, \varphi \varphi \varphi,...$ are said to be \emph{iterated} and are generally denoted by the symbols $\varphi^2, \varphi^3,...$ as powers of the operation $\varphi$.
\end{translateTwoCol}

\peanoPage{XVI} % page-number XVI

\begin{translateTwoCol}
Si vero $\varphi$ est functionis postsignum, ha faciliori notatione, absque ambiguitate, uti licet:
\ENG
But if $\varphi$ is a function postsymbol, we can use the following more convenient notation without ambiguity:
\LAT
\emph{Def.} \hspace{0.25cm} $s \smallIn \K$ . $\varphi \smallIn s 'F$ . $s \varphi$ $\C$ $s : \C : \varphi 1 = \varphi$ . $\varphi 2 = \varphi \varphi$ . $\varphi 3 = \varphi \varphi \varphi$. etc.
\ENG
\emph{Def.} \hspace{0.25cm} \todo
\LAT
In definitionis hypothesi, si $m, n \smallIn N$, erit $\varphi$ $(m+n)$ $=$ $(\varphi m)(\varphi n)$; $(\varphi m)n$ $=$ $\varphi (m n)$
\ENG
Assuming this definition, if $m, n \in \mathbb{N}$, we have $\varphi(m+n)$ $=$ $(\varphi m)(\varphi n)$; $(\varphi m)n$ $=$ $\varphi (m n)$
\LAT
Si hac definitione in Arithmetica utimur, haec invenimus. Numerum qui sequitur numerum $a$ signo faciliori $a +$ indicare possumus; tunc $a + 1, a+2,...$ et, si $b$ est numerus, $a + b$, sensum habent $a +, a + +,...$ quod a definitione in \S 1 patet. Propositionem 6 in \S 1 scribere possumus $N +$ $\C$ $N$. Si $a, b, c$ sunt numeri, tunc $a : + b$ . $c$ significat $a + b c$, et $a : \times b$ . $c$ significat $a b^c$.
\ENG
If we use this definition in arithmetic, we obtain the following. We can denote the number that follows the number $a$ by the more convenient notation $a+$; then $a + 1$, $a + 2$,..., and, if $b$ is a number, $a + b$, have the meaning of $a +$, $a ++$,..., which is clear from the definition in \S 1 below. Proposition 6 in \S 1 can be written $\mathbb{N} +$ $\subset$ $\mathbb{N}$. \irrelavent{If $a, b, c$ are numbers, then $a : + b$ . $c$ means $a + b c$, and $a : \times b$ . $c$ means $a b^c$.} % Chose to translate second ``signum'' as ``notation'' rather than ``symbol''.  May also want to use it on first one.
\LAT
Multi aliis proprietatibus gaudent functionem signa, praesertim si conditioni satisfaciunt: $\varphi x = \varphi y$ . $\C$ . $x = y$. Functionis signum quod huic conditioni satisfacit vocatur a clarrissimo Dedekind \emph{simile} (\"ahnliche Abbildung).
\ENG
Function symbols possess many other properties, especially if they satisfy the condition: $(\varphi x = \varphi y) \rightarrow (x = y)$. A function symbol that satisfies this condition is called \emph{equivalent}\endnote{The 2 other translation, mentioned at the beginning of this current document, translated "\"ahnlich" literally to "similar", instead of "equivalent". However, additional information can be found in a footnote of the first translation: \\

\quad \emph{"Today "similar" has another meaning and instead we would say "equivalent"."} \\

G. Peano, (1889), ``The principles of arithmetic presented by a new method" in: J. van Heijenoort (ed.), \emph{From Frege to G\"odel. A source book in mathematical logic. 1879-1931}, Cambridge: Harvard University Press, 1967, p. 93.} by Dedekind (\"ahnliche Abbildung).
\LAT
Sed his exponendis locus deest.
\ENG
But we lack the space to present these properties.
\end{translateTwoCol}

\begin{translateTwoCol}
\centering
\phantomsection
\addcontentsline{toc}{section}{Remarks}
\peanoHeadingSmall{Declarationes.}
\ENG
\peanoHeadingSmall{Remarks.}
\end{translateTwoCol}

\begin{translateTwoCol}
\emph{Defenitio}, vel breviter \emph{Def.} est propositio formam habens $x = a$, sive $\alpha$ $\C$ . $x = a$, ubi $\alpha$ est signorum aggregatus sensum habens notum; $x$ est signum, vel signorum aggregatus significatione adhuc carnes; $\alpha$ vero est conditio sub qua definitio datur.
\ENG
A \emph{Definition}, or \emph{Def.} for short, is a proposition of the form $x=a$ or $\alpha \rightarrow (x = a)$, where $a$ is an aggregate of symbols having a known meaning, $x$ is a symbol or an aggregate of symbols, hitherto without meaning, and $\alpha$ is the condition under which the definition is given.
\LAT
\emph{Theorema}, (Theor. vel Th) est propositio quae demonstratur. Si theorema formam habet $\alpha$ $\C$ $\beta$, ubi $\alpha$ et $\beta$ sunt propositiones, tunc $\alpha$ dicitur \emph{Hypothesis} (Hyp. vel breviter Hp.), $\beta$ vero \emph{Thesis} (Thes. vel Ts.). Hyp. ac Ts. a Theorematis forma pendent; nam si loco $\alpha$ $\C$ $\beta$ scribemus $- \beta$ $\C$ $- \alpha$, erit $- \beta$ Hp, et $- \alpha$ Ts.; si vero scribemus $\alpha - \beta = \abs$, Hp. ac Ts. absunt.
\ENG
A \emph{theorem} (Theor. or Th.) is a proposition that is \emph{proved}. If a theorem has the form $\alpha$ $\C$ $\beta$, where $\alpha$ and $\beta$ are propositions, then $\alpha$ is called the \emph{hypothesis} (Hyp., or even shorter, Hp.) and $\beta$ the \emph{thesis} (Thes. or Ts.). Hyp. and Ts. depend on the form of the theorem; in fact, if we write $\alpha$ $\C$ $\beta$ instead of $- \beta$ $\C$ $- \alpha$, then $- \beta$ is the Hp., and $- \alpha$ the Ts.; if we write $\alpha - \beta = \abs$, Hp. and Ts. do not exist.
\LAT
In quolibet $\S$ signum P quod quidam numerus sequatur, propositionem indicat eiusdem $\S$ hoc numero signatam. Logicae propositiones indicantur signo L et propositiones numero.
\ENG
In any $\S$ below, the symbol P followed by a number denotes the proposition indicated by that number in the same $\S$. Propositions of logic are indicated by the symbol L and the number of the proposition.
\LAT
Formulae quae in una linea non continentur, in altera linea, nullo interposito signo, sequuntur.
\ENG
Formulas that do not fit on one line are continued on the next line without any intervening symbol.
\end{translateTwoCol}

\pagebreak

\peanoPage{1} % page-number 1

\begin{translateTwoCol}
\centering
\phantomsection
\addcontentsline{toc}{chapter}{The principles of arithmetic}
\peanoHeadingLarge{Arithmetices Principia.}
\ENG
\peanoHeadingLarge{The Principles of Arithmetic.}
\LAT
\phantomsection
\addcontentsline{toc}{section}{\S1. Numbers and addition}
\peanoHeadingMedium{\S1. De numeris et de additione.}
\ENG
\peanoHeadingMedium{\S1. Numbers and addition.}
\LAT
\phantomsection
\addcontentsline{toc}{subsection}{Explanations}
\peanoHeadingSmall{Explicationes.}
\ENG
\peanoHeadingSmall{Explanations.}

\raggedright
\commentary{Peano starts his natural numbers at 1; most modern versions start at 0.  Also, while Peano calls it ``successor'' (``sequens''), he does not use the modern convention of using ``S'' for it.}
\end{translateTwoCol}

\begin{translateEightCol}{0.1}{0.05}{0.1}{0.25}{0.1}{0.05}{0.1}{0.25}
\centering Signo \s $\N$ \s significatur \s \raggedright \emph{numerus (integer positivus)}.
\s
\centering The~symbol \s $\mathbb{N}$ \s means \s \raggedright \emph{number (positive integer)}.
\s*
\centering $\dittoMarkLatin$ \s $1$ \s $\dittoMarkLatin$ \s \raggedright \emph{unitas}.
\s
\centering \dittoMarkEnglish \s $1$ \s \dittoMarkEnglish \s \raggedright \emph{unity}.
\s*
\centering $\dittoMarkLatin$ \s $a + 1$ \s $\dittoMarkLatin$ \s \raggedright \emph{sequens} $a$, sive $a$ \emph{plus} $1$.
\s
\centering \dittoMarkEnglish \s $a + 1$ \s \dittoMarkEnglish \s \raggedright \emph{the successor of} $a$, or $a$ \emph{plus} $1$.
\s*
\centering $\dittoMarkLatin$ \s $=$  \s $\dittoMarkLatin$ \s \raggedright \emph{est aequalis}. Hoc ut novum signum considerandum est, etsi logicae signi figuram habeat.
\s
\centering \dittoMarkEnglish \s $=$ \s \dittoMarkEnglish \s \raggedright \emph{is equal to}. This must be considered as a new symbol, although it has the appearance of a symbol of logic.
\end{translateEightCol}

\begin{translateTwoCol}
\centering
\phantomsection
\addcontentsline{toc}{subsection}{Axioms}
\peanoHeadingSmall{Axiomata.}
\ENG
\peanoHeadingSmall{Axioms.}
\end{translateTwoCol}

\begin{translateTwoCol}
1. \hspace{0.6cm} $1 \smallIn \N$
\ENG
1. \hspace{0.6cm} $1 \in \mathbb{N}$
\LAT
2. \hspace{0.6cm} $a \smallIn \N \p \C \p a = a$
\ENG
2. \hspace{0.6cm} $a \in \mathbb{N} \rightarrow a = a$
\LAT
3. \hspace{0.6cm} $a,b,c \smallIn \N \p \C : a = b \p = \p b = a$
\ENG
3. \hspace{0.6cm} $a,b,c \in \mathbb{N} \rightarrow [(a = b) = (b = a)]$
\LAT
4. \hspace{0.6cm} $a,b \smallIn \N \p \C \ppp a = b \p b = c : \C \p a = c$
\ENG
4. \hspace{0.6cm} $a,b \in \mathbb{N} \rightarrow [(a = b \wedge b = c) \rightarrow a = c]$
\LAT
5. \hspace{0.6cm} $a = b \p b \smallIn \N : \C \p a \smallIn \N$
\ENG
5. \hspace{0.6cm} $(a = b \wedge b \in \mathbb{N}) \rightarrow a \in \mathbb{N}$
\LAT
6. \hspace{0.6cm} $a \smallIn \N \p \C \p a + 1 \smallIn \N$
\ENG
6. \hspace{0.6cm} $a \in \mathbb{N} \rightarrow (a + 1 \in \mathbb{N})$
\LAT
7. \hspace{0.6cm} $a, b \smallIn \N \p \C : a = b \p = \p a + 1 = b + 1$
\ENG
7. \hspace{0.6cm} $a, b \in \mathbb{N} \rightarrow [(a = b) = (a + 1 = b + 1)]$
\LAT
8. \hspace{0.6cm} $a \smallIn \N \p \C \p a + 1 \no = 1$
\ENG
8. \hspace{0.6cm} $a \in \mathbb{N} \rightarrow (a + 1 \neq 1)$
\LAT
9. \hspace{0.6cm} $k \smallIn \K \ppp 1 \smallIn k \ppp x \smallIn \N \p x \smallIn k : \C_x \p x + 1 \smallIn k : : \C \p \N \C k$
\ENG
9. \hspace{0.6cm} $\big(A \in \setOfSets \wedge 1 \in A \wedge [(x \in \mathbb{N} \wedge x \in A) \xrightarrow[\forall x]{} (x + 1 \in A)] \big) \rightarrow \mathbb{N} \subset A$
\end{translateTwoCol}

\begin{translateTwoCol}
\centering
\phantomsection
\addcontentsline{toc}{subsection}{Definitions}
\peanoHeadingSmall{Definitiones.}
\ENG
\peanoHeadingSmall{Definitions.}
\end{translateTwoCol}

\begin{translateTwoCol}
10. \hspace{0.67cm} $2 = 1 + 1;$ $3 = 2 + 1;$ $4 = 3 +1;$ etc.
\ENG
10. \hspace{0.67cm} $2 = 1 + 1;$ $3 = 2 + 1;$ $4 = 3 +1;$ etc.
\end{translateTwoCol}

\peanoPage{2} % page-number 2

\begin{translateTwoCol}
\centering
\phantomsection
\addcontentsline{toc}{subsection}{Theorems}
\peanoHeadingSmall{Theoremata.}
\ENG
\peanoHeadingSmall{Theorems.}
\end{translateTwoCol}

\begin{translateTwoCol}
11. \hspace{0.67cm} $2 \smallIn \N$.
\ENG
11. \hspace{0.67cm} $2 \in \mathbb{N}$.
\LAT
\vspace{1em}
\emph{Demonstratio:}
\vspace{1em}
\ENG
\vspace{1em}
\emph{Proof:}
\vspace{1em}
\end{translateTwoCol}

\begin{translateSixCol}{0.25}{0.17}{0.08}{0.25}{0.17}{0.08}
\raggedright
$P 1 \p \C :$ \s $1 \smallIn N$ \s (1)
\s
Axiom 1 \s $1 \in \mathbb{N}$ \s (1)
\s*
$1 [a] (P6) \p \C :$ \s $1 \smallIn \N \p \C \p 1 + 1 \smallIn N$ \s (2)
\s
Axiom 6[a:=1] \s $1 \in \mathbb{N} \rightarrow (1 + 1 \in \mathbb{N})$ \s (2)
\s*
$(1)(2) \p \C :$ \s $1 + 1 \smallIn N$ \s (3)
\s
Steps 1 and 2 \s $1 + 1 \in \mathbb{N}$ \s (3)
\s*
$P 10 \p \C :$ \s $2 = 1 + 1$ \s (4)
\s
Def. 10 \s $2 = 1 + 1$ \s (4)
\s*
$(4) \p (3) \p (2, 1+1) [a, b] (P 5) : \C :$ \s $2 \smallIn N$ \s (Theor.)
\s
Steps 4,3,Axiom 5[a:=2,b:=$1+1$] \s $2 \in \mathbb{N}$ \s (Q.E.D.)
\end{translateSixCol}

\begin{translateTwoCol}
\vspace{1em}
\commentary{Peano's proof notation uses ``$P1$'' and ``$P6$'' to refer to proposition \#1 and \#6 proved earlier.  The ``$1[a]$'' before ``$(P6)$'' is Peano's notation for substituting $1$ for $a$ in proposition \#6.   The notations ``$(1)$'' and ``$(2)$'' refer to the results of steps 1 and 2 of this proof.  Modus ponens is indicated by concatenating the antecedent with the implication, as in ``$(1)(2)$''.  Peano's style is different from a modern proof, where each line consists of a statement and a rationale.  Here, he connects the rationale to the statement with a $\C$.  So the rationale actually \emph{implies} the statement.  This makes each line into a complete universal expression of logic.  All the steps can be combined together to form a single expression, called a ``proof term''.  This is what happens below.}
\ENG
\commentary{Peano is able to convert the entire proof into a single proof term.  This is possible because of his notation.  Modern notation does not generally use proof terms.  Modern works tend to use a proof tree, which we do not show here.  It is worth noting that some research into proof theory does use proof terms.}
\LAT
\emph{Nota.} - Huius facillimae demonstrationis gradus omnes ecplicite scripsimus. Brevitatis causa ipsam ita scribemus:
\vspace{1em}
\ENG
\vspace{1em}
\emph{Note.} - We have explicitly written every step of this very easy proof. For the sake of brevity, we shall write it as follows:
\vspace{1em}
\LAT
$\hspace{1.4cm} P 1 \p 1 [a] (P 6) : \C : 1 +1 \smallIn \N \p P 10 \p (2, 1+1) [a, b] (P 5) : \C : Th$
\ENG
\notPossible
\LAT
vel
\ENG
or
\LAT
$\hspace{1.4cm} P 1 \p P 6 : \C : 1 + 1 \smallIn \N \p P 10 \p P 5 : \C : Th$
\ENG
\notPossible
\LAT
12. \hspace{0.67cm} $3, 4,... \smallIn \N$
\ENG
12. \hspace{0.67cm} $3, 4, ... \in \mathbb{N}$
\LAT
13. \hspace{0.67cm} $a,b,c,d \smallIn \N \p a = b \p b = c \p c = d : \C : a = d$
\ENG
13. \hspace{0.67cm} $(a,b,c,d \in \mathbb{N} \wedge a = b \wedge b = c \wedge c = d ) \rightarrow a = d$
\LAT
\emph{Dem.} \hspace{0.27cm} $Hyp. P 4 : \C : a,c,d \smallIn \N \p a = c \p c = d \p P4 : \C : Thes.$
\ENG
\emph{Proof} \hspace{0.27cm} \todo
\LAT
14. \hspace{0.67cm} $a,b,c \smallIn \N \p a = b \p b = c \p a \no = c : = \abs$
\ENG
14. \hspace{0.67cm} $(a,b,c \in \mathbb{N} \wedge a = b \wedge b = c \wedge a \neq c) = \bot$
\LAT
\emph{Dem.} \hspace{0.27cm} $P4 \p L 39 : \C \p Theor$.
\ENG
\emph{Proof} \hspace{0.27cm} \todo
\LAT
15. \hspace{0.67cm} $a,b,c \smallIn \N \p a = b \p b \no = c : \C \p a \no = c$
\ENG
15. \hspace{0.67cm} $(a,b,c \in \mathbb{N} \wedge a = b \wedge b \neq c) \rightarrow  a \neq c$
\LAT
16. \hspace{0.67cm} $a,b \smallIn \N \p a + 1 = b + 1 : \C \p a = b$
\ENG
16. \hspace{0.67cm} $[a,b \in \mathbb{N} \wedge (a + 1 = b + 1 )] \rightarrow a = b$
\LAT
17. \hspace{0.67cm} $a,b \smallIn \N \p \C : a \no = b \p = \p a + 1 \no = b + 1$
\ENG
17. \hspace{0.67cm} $(a,b \in \mathbb{N}) \rightarrow [(a \neq b) = (a + 1 \neq b + 1)]$
\LAT
\emph{Dem.} \hspace{0.27cm} $P7 \p L21 : \C \p Theor$
\ENG
\emph{Proof} \hspace{0.27cm} \todo
\end{translateTwoCol}

\begin{translateTwoCol}
\centering
\phantomsection
\addcontentsline{toc}{subsection}{Definition}
\peanoHeadingSmall{Definitio.}
\ENG
\peanoHeadingSmall{Definition.}
\end{translateTwoCol}

\begin{translateTwoCol}
18. \hspace{0.67cm} $a,b \smallIn \N \p \C \p a + (b + 1) = (a + b) + 1$
\ENG
18. \hspace{0.67cm} $(a,b \in \mathbb{N}) \rightarrow [a + (b + 1) = (a + b) + 1]$
\LAT
\vspace{1em}
\emph{Nota.} - Hanc definitionem ita legere oportet: si $a$ et $b$ sunt numeri, et $(a + b) + 1$ sensum habet (scilicet si $a + b$ est numerus), sed $a + (b + 1)$ nondum definitus est, tunc $a + (b + 1)$ significat numerum qui $a + b$ sequitur.
\vspace{1em}
\ENG
\vspace{1em}
\emph{Note.} - This definition should be read: if $a$ and $b$ are numbers, and $(a + b) + 1$ has meaning (that is, if $a + b$ is a number), but $a + (b + 1)$ has not yet been defined, then $a + (b + 1)$ indicates the number that follows $a + b$.
\vspace{1em}
\LAT
Ab hac definitione, et a praecedentibus deducitur:
\ENG
From this definition, and the preceding, we deduce that:
\LAT
\hspace{1.4cm} $a \smallIn \N \p \C \ppp a + 2 = a + (1 + 1) = (a + 1) + 1$
\ENG
\hspace{1.4cm} $a \in \mathbb{N} \rightarrow a + 2 = a + (1 + 1) = (a + 1) + 1$
\LAT
\hspace{1.4cm} $a \smallIn \N \p \C \ppp a + 3 = a + (2 + 1) = (a + 2) + 1$, etc.
\ENG
\hspace{1.4cm} $a \in \N \rightarrow a + 3 = a + (2 + 1) = (a + 2) + 1$, etc.
\end{translateTwoCol}

\peanoPage{3} % page-number 3

\begin{translateTwoCol}
\centering
\phantomsection
\addcontentsline{toc}{subsection}{Theorems}
\peanoHeadingSmall{Theoremata.}
\ENG
\peanoHeadingSmall{Theorems.}
\end{translateTwoCol}

\begin{translateSixCol}{0.08}{0.39}{0.02}{0.08}{0.39}{0.02}
\raggedright
19. \s $a, b \smallIn \N \p \C \p a + b \smallIn N$. \s
\s
19. \s \todo \s
\s*
\emph{Dem.} \s $a \smallIn \N \p$ P6 $: \C : a + 1 \smallIn \N : \C : 1 \smallIn [b \smallIn]$ Ts. \s (1)
\s
\emph{Proof} \s \todo \s (1)
\s*
\s $a \smallIn \N \p \C :: b \smallIn \N \p b \smallIn [b \smallIn]$ Ts $: \C : a + b \smallIn \N \p$ P6 $: \C : (a + b) + 1 \smallIn \N \p$ P18 $: \C : a + (b + 1) \smallIn \N : \C : (b + 1) \smallIn [b \smallIn]$ Ts. \s (2)
\s
\s \s (2)
\s*
\s $a \smallIn \N \p (1) \p (2) \p \C : : 1 \smallIn [b \smallIn]$ Ts $\ppp b \smallIn \N \p b \smallIn [b \smallIn]$ Ts $: \C : b + 1 \smallIn [b \smallIn]$ Ts $\ppp ([b \smallIn]$ Ts $) [k]$ P9 $: : \C : \N \C [b \smallIn]$ Ts $\p$ (L50) $:: \C : b \smallIn \N \p C$ Ts. \s (3)
\s
\s \s (3)
\s*
\s $(3) \p$ (L42) $: C : a,b \smallIn \N \p \C \p$ Thesis \s (Th.)
\s
\s \s (Th.)
\s*
20. \emph{Def.} \s $a + b + c = (a + b) + c$ \s
\s
20. \emph{Proof} \s \todo \s
\s*
21. \s $a,b,c \smallIn \N \p \C \p a + b + c \smallIn N$ \s
\s
21. \s \todo \s
\s*
22. \s $a,b,c \smallIn \N \p \C : a = b \p = \p a + c = b + c$ \s
\s
22. \s \todo \s
\s*
\emph{Dem.} \s $a,b \smallIn \N \p$ P7 $: \C \p 1 \smallIn [c \smallIn]$ Ts. \s (1)
\s
\emph{Proof} \s \todo \s (1)
\s*
\s $a,b \smallIn \N \p \C :: c \smallIn \N \p c \smallIn [c \smallIn]$ Ts $\ppp \C \ppp a = b \p = \p a + c = b + c : a + c , b + c \smallIn \N : a + c = b + c \p = \p a + c + 1 = b + c + 1 \ppp \C \ppp a = b \p = \p a + (c + 1) = b + (c + 1) \ppp \C \ppp (c + 1) \smallIn [C\smallIn]$ Ts. \s (2)
\s
\s \s (2)
\s*
\s $a,b \smallIn \N \p (1) \p (2) : \C :: 1 \smallIn [c \smallIn]$ Ts $\ppp c \smallIn [c \smallIn]$ Ts $ \p \C \p (c + 1) \smallIn [c \smallIn]$ Ts $:: \C :: c \smallIn \N \p \C \p$ Ts. \s (3)
\s
\s \s (3)
\s*
23. \s $a,b,c \smallIn \N \p \C \p a + (b + c) = a + b + c$ \s
\s
23. \s \todo \s
\s*
\emph{Dem.} \s $a,b \smallIn \N \p$ P18 $\p$ P20 $: \C \p 1 \smallIn [c \smallIn]$ Ts. \s (1)
\s
\emph{Proof} \s \todo \s (1)
\s*
\s $a,b \smallIn \N \p \C \ppp c \smallIn \N \p c \smallIn [c \smallIn]$ Ts $: \C : a + (b +c) = a + b + c \p$ P7 $: \C : a + (b + c) + 1 = a + b + c + 1 \p$ P18 $: \C : a + (b + (c + 1)) = a + b + (c + 1) : \C \p c + 1 \smallIn [c \smallIn]$ Ts. \s (2)
\s
\s \s (2)
\s*
\s $(1)(2)$ (P9) $\p \C \p$ Theor. \s
\s
\s \s
\s*
24. \s $a \smallIn \N \p \C \p 1 + a = a + 1$ \s
\s
24. \s \todo \s
\s*
\emph{Dem.} \s P2 $\p \C \p 1 \smallIn [a \smallIn]$ Ts. \s (1)
\s
\emph{Proof} \s \todo \s (1)
\s*
\s $a \smallIn \N \p a \smallIn [a \smallIn]$ Ts $: \C : 1 + 1 = a + 1 : \C : 1 + (a + 1) = (a + 1) + 1 : \C : (a + 1) \smallIn [a \smallIn]$ Ts. \s (2)
\s
\s \s (2)
\s*
\s $(1)(2) \p \C \p$ Theor. \s
\s
\s \s
\s*
24'. \s $a,b \smallIn \N \p \C \p 1 + a + b = a + 1 + b$ \s
\s
24'. \s \todo \s
\s*
\emph{Dem.} \s Hyp. P24 $: \C \C 1 + a = a + 1 \p$ P22 $: \C \p$ Thesis. \s
\s
\emph{Proof} \s \todo \s
\end{translateSixCol}

\peanoPage{4} % page-number 4

\begin{translateSixCol}{0.08}{0.39}{0.02}{0.08}{0.39}{0.02}
25. \s $a,b \smallIn \N \p \C \p a + b = b + a$. \s
\s
25. \s \todo \s
\s*
\emph{Dem.} \s $a \smallIn \N \p$ P24 $: \C : 1 \smallIn [b \smallIn]$ Ts. \s (1)
\s
\emph{Proof} \s \todo \s (1)
\s*
\s $a \smallIn \N \p \C \ppp b \smallIn \N \p b \smallIn [b \smallIn]$ Ts $: \C : a + b = b + a \p$ P7 $: \C : (a + b) + 1 = (b + a) + 1 \p (a + b) + 1 = (a + (b + 1) \p (b + a ) + 1 = 1 + (b + a) \p 1 + (b + a) = (1 + b) + a \p (1 + b) + a = (b + 1) + a : \C : a + (b + 1) = (b + 1) + a : \C : (b + 1) \smallIn [b \smallIn]$ Ts. \s (2)
\s
\s \s (2)
\s*
\s $(1)(2) \p \C \p$ Theor. \s
\s
\s \s
\s*
26. \s $a,b,c \smallIn \N \p \C : a = b \p = \p c + a = c + b$. \s
\s
26. \s \todo \s
\s*
27. \s $a,b,c \smallIn \N \p \C : a + b + c = a + c + b$. \s
\s
27. \s \todo \s
\s*
28. \s $a,b,c,d \smallIn \N \p a = b \p c = d : \C \p a + c = b + d$. \s
\s
28. \s \todo \s
\end{translateSixCol}

\begin{translateTwoCol}
\centering
\phantomsection
\addcontentsline{toc}{section}{\S2. Subtraction}
\peanoHeadingMedium{\S2. De substractione.}
\ENG
\peanoHeadingMedium{\S2. Subtraction.}
\LAT
\phantomsection
\addcontentsline{toc}{subsection}{Explanations}
\peanoHeadingSmall{Explicationes.}
\ENG
\peanoHeadingSmall{Explanations.}
\end{translateTwoCol}

\begin{translateEightCol}{0.1}{0.1}{0.1}{0.2}{0.1}{0.1}{0.1}{0.2}
\centering Signo \s $-$ \s legitur \s \raggedright \emph{minus}.
\s
\centering The~symbol \s $-$ \s is~read \s \raggedright \emph{minus}.
\s*
\centering $\dittoMarkLatin$ \s $<$ \s $\dittoMarkLatin$ \s \raggedright \emph{est minor.}
\s
\centering \dittoMarkEnglish \s $<$ \s \dittoMarkEnglish \s \raggedright \emph{is less than.}
\s*
\centering $\dittoMarkLatin$ \s $>$ \s $\dittoMarkLatin$ \s \raggedright \emph{est maior.}
\s
\centering \dittoMarkEnglish \s $>$ \s \dittoMarkEnglish \s \raggedright \emph{is greater than.}
\end{translateEightCol}

\begin{translateTwoCol}
\centering
\phantomsection
\addcontentsline{toc}{subsection}{Definitions}
\peanoHeadingSmall{Definitiones.}
\ENG
\peanoHeadingSmall{Definitions.}
\end{translateTwoCol}

\begin{translateSixCol}{0.08}{0.39}{0.02}{0.08}{0.39}{0.02}
\raggedright
1. \s $a,b \smallIn \N \p \C : b - a = \N [x \smallIn] (x + a = b)$. \s
\s
1. \s $a,b \in \mathbb{N} \rightarrow b - a = \mathbb{N} \cap \{x | x + a = b \}$. \s
\s*
2. \s $a,b \smallIn \N \p \C : a < b \p = \p b - a \no = \abs$. \s
\s
2. \s $a,b \in \mathbb{N} \rightarrow (a < b) = (b - a \neq \bot$). \s
\s*
3. \s $a,b \smallIn \N \p \C : b > a \p = \p a < b$. \s
\s
3. \s $a,b \in \mathbb{N} \rightarrow (b > a) = (a < b)$. \s
\s*
\s $a + b - c = (a + b) - c; a - b + c = (a - b) + c; a - b - c = (a - b) - c$. \s
\s
\s $a + b - c = (a + b) - c; a - b + c = (a - b) + c; a - b - c = (a - b) - c$. \s
\end{translateSixCol}

\begin{translateTwoCol}
\centering
\phantomsection
\addcontentsline{toc}{subsection}{Theorems}
\peanoHeadingSmall{Theoremata.}
\ENG
\peanoHeadingSmall{Theorems.}
\end{translateTwoCol}

\begin{translateSixCol}{0.08}{0.39}{0.02}{0.08}{0.39}{0.02}
\raggedright
4. \s $a,b,a',b' \smallIn \N \p a = a' \p b = b' : \C : b - a = b' - a'$. \s
\s
4. \s \todo... \s
\s*
\emph{Dem.} \s Hyp $\p \C : x + a = b \p = \p x + a' = b' : \C \p$ Thesis. \s
\s
\s \s
\s*
5. \s $a,b \smallIn \N \p \C : a < b \p = \p b - a \smallIn N$. \s
\s
\s \s
\s*
\emph{Dem.} \s $a,b \smallIn \N : \C \ppp x,y \smallIn b - a \p \C_{x,y} : x, y \smallIn \N \p x + a = b \p y + a = b \p$ \S1 P22 $: \C : x = y$. \s (1)
\s
\s \s
\s*
\s $a,b \smallIn \N \p a < b \p$ P2 $\p (1) : \C \ppp b - a \no = \abs : x,y \smallIn b - a \p \C \p x = y : (N, b - a) [s,k]$ (L56) $\ppp \C \ppp b - a \smallIn N$. \s (2)
\s \s
\end{translateSixCol}

\peanoPage{5} % page-number 5

\begin{translateSixCol}{0.08}{0.39}{0.02}{0.08}{0.39}{0.02}
\raggedright
\s $a,b \smallIn \N \p b - a \smallIn \N \p$ (L56) $: \C : b - a \no = \abs : \C : a < b$. \s (3)
\s
\s \s
\s*
\s $(2)(3) \p \C \p$ Theor. \s
\s
\s \s
\s*
6. \s $a,b \smallIn \N \p a < b : \C \p b - a + a = b$. \s
\s
\s \s
\s*
\emph{Dem.} \s Hyp $\p$ P5 $\p$ P1 $: \C : b - a \smallIn \N \p (b - a) \smallIn [x \smallIn] (x + a = b) : \C :$ Thes. \s
\s
\s \s
\s*
7. \s $a,b,c \smallIn \N \p \C : c = b - a \p = \p c + a = b$. \s
\s
\s \s
\s*
\emph{Dem.} \s Hyp $\p \S 1$ P22 $\p$ P6 $: \C : c = b - a \p = \p c + a = b - a + a \p = \p c + a = b$. \s
\s
\s \s
\s*
8. \s $a,b \smallIn \N \p \C \p a + b - a = b$. \s
\s
\s \s
\s*
\emph{Dem.} \s $(a + b, b)[b, c]$ P7 $\p \C \p$ Theor. \s
\s
\s \s
\s*
9. \s $a,b,c \smallIn \N \p a < b : \C : c + (b - a) = c + b - a$. \s
\s
\s \s
\s*
\emph{Dem.} \s Hyp $\p$ P6 $: \C : (b - a) + a = b : \C : c + (b - a) + a = c + b \p$ P7 $: \C :$ Thesis. \s
\s
\s \s
\s*
10. \s $a,b,c \smallIn \N \p a > b + c : \C \p a - (b + c) = a - b - c$. \s
\s
\s \s
\s*
11. \s $a,b,c \smallIn \N \p b > c \p a > b - c : \C \p a - (b - c) = a + c - b$. \s
\s
\s \s
\s*
12. \s $a,b,a',b' \smallIn \N \p a = a' \p b = b' : \C : a < b \p = \p a' < b'$. \s
\s
\s \s
\s*
\emph{Dem.} \s Hyp $\p \C \p b - a = b' - a' \p \C \p b - a \smallIn \N = b' - a' \smallIn \N \p \C \p$ Thes. \s
\s
\s \s
\s*
13. \s $a,b \smallIn \N \p \C \p a < a + b$. \s
\s
\s \s
\s*
\emph{Dem.} \s Hyp $\p P8 : \C : a + b - a = b : \C \p a + b - a \smallIn \N \p$ P5 $: \C :$ Thesis. \s
\s
\s \s
\s*
14. \s $a,b,c \smallIn \N \p a < b \p b < c : \C \p a < c$. \s
\s
\s \s
\s*
\emph{Dem.} \s Hyp $\p \C : b - a \smallIn \N \p c - b \smallIn : \C : (b - a) + (c - b) \smallIn \N : \C : c - a \smallIn \N : \C \p$ Thesis. \s
\s
\s \s
\s*
15. \s $a,b,c \smallIn \N \p \C : a < b \p = \p a + c < b + c$. \s
\s
\s \s
\s*
\emph{Dem.} \s Hyp $\p \C : a < b \p = \p b - a \smallIn \N \p = \p (b + c) - (a + c) \smallIn \N \p = \p a + c < b + c$. \s
\s
\s \s
\s*
16. \s $a,b,a',b' \smallIn \N \p a < b \p a' < b' : \C \p a + a' < b + b'$. \s
\s
\s \s
\s*
\emph{Dem.} \s Hyp $\p \C : a + a' < b + a' \p b + a' < b + b' : \C \p$ Thesis. \s
\s
\s \s
\s*
17. \s $a,b,c \smallIn \N \p a < b < c : \C \p c - a > c - b$. \s
\s
\s \s
\s*
\emph{Dem.} \s Hyp $\p \C b - a \smallIn \N \p c - b \smallIn \N \p (c - b) + (b - a) = c - a : \C \p$ Thesis. \s
\s
\s \s
\s*
18. \s $a \smallIn \N \p \C : a = 1 \p \cup \p a > 1$. \s
\s
\s \s
\s*
\emph{Dem.} \s $1 \smallIn [a \smallIn]$ Thesis. \s
\s
\s \s
\s*
\s $a \smallIn \N \p$ P13 $: \C : a + 1 > 1 : \C : a + 1 \smallIn [a \smallIn]$ Thesis. \s
\s
\s \s
\s*
\s $(1)(2) \p \C \p$ Theor. \s
\s
\s \s
\end{translateSixCol}

\peanoPage{6} % page-number 6

\begin{translateSixCol}{0.08}{0.39}{0.02}{0.08}{0.39}{0.02}
\raggedright
19. \s $a,b \smallIn \N \p \C \p a + b \no = b$.\s
\s
\s \s
\s*
\emph{Dem.} \s $a \smallIn \N \p \S1$ P8 $: \C : a + 1 \no = 1 : \C : 1 \smallIn [b \smallIn]$ Thesis. \s (1)
\s
\s \s
\s*
\s $a \smallIn \N \p b \smallIn \N \p b \smallIn [b \smallIn]$ Ts $: \C : a + b \no = b \p \S1$ P17 $: \C : a + (b + 1) \no = b + 1 : \C : b + 1 \smallIn [b \smallIn]$ Ts. \s (2)
\s
\s \s
\s*
\s $(1)(2) \p \C \p$ Theor. \s
\s
\s \s
\s*
20. \s $a,b \smallIn \N \p a < b \p a = b : = \abs$. \s
\s
\s \s
\s*
\emph{Dem.} \s Hyp $: \C : b - a \smallIn \N \p (b - a) + a = a \p$ P19 $: \C : \abs$. \s
\s
\s \s
\s*
21. \s $a,b \smallIn \N \p a > b \p a = b : = \abs$. \s
\s
\s \s
\s*
22. \s $a,b \smallIn \N \p a > b \p a < b : = \abs$. \s
\s
\s \s
\s*
23. \s $a,b \smallIn \N : \C : a < b \p \cup \p a = b \p \cup \p a > b$. \s
\s
\s \s
\s*
\emph{Dem} \s $a \smallIn \N \p$ P18 $: \C \p 1 \smallIn [b \smallIn]$ Ts. \s (1)
\s
\s \s
\s*
\s $a,b \smallIn \N \p a < b : \C \p a < b + 1$. \s (2)
\s
\s \s
\s*
\s $a,b \smallIn \N \p a = b : \C \p a < b + 1$. \s (3)
\s
\s \s
\s*
\s $a,b \smallIn \N \p a > b : \C : a - b \smallIn \N \p$ P18 $: \C : a - b = 1 \p \cup \p a - b > 1$. \s (4)
\s
\s \s
\s*
\s $a,b \smallIn \N \p a - b = 1 : \C \p a = b + 1$. \s (5)
\s
\s \s
\s*
\s $a,b \smallIn \N \p a - b > 1 : \C \p a > b + 1$. \s (6)
\s
\s \s
\s*
\s $a,b \smallIn \N \p a > b \p (4)(5)(6) : \C : a = b + 1 \p \cup \p a > b + 1 \p$ \s (7)
\s
\s \s
\s*
\s $a,b \smallIn \N : a > b \p \cup \p a = b \p \cup \p a > b : (2)(3)(7) \ppp \C \ppp a < b + 1 \p \cup \p a = b + 1 \p \cup \p a > b + 1$. \s (8)
\s
\s \s
\s*
\s $a,b \smallIn \N \p b \smallIn [b \smallIn]$ Ts $\p (8) : \C : b + 1 \smallIn [b \smallIn]$ Ts. \s (9)
\s
\s \s
\s*
\s $(1)(9) \p \C \p$ Theor. \s
\s
\s \s
\end{translateSixCol}

\begin{translateTwoCol}
\centering
\phantomsection
\addcontentsline{toc}{section}{\S3. Maxima and minima}
\peanoHeadingMedium{\S3. De maximis et minimis.}
\s
\peanoHeadingMedium{\S3. Maxima and minima.}
\s*
\phantomsection
\addcontentsline{toc}{subsection}{Explanations}
\peanoHeadingSmall{Explicationes.}
\s
\peanoHeadingSmall{Explanations.}
\s*
Sit $a \smallIn \K N$, hoc est sit $a$ quaedam numerorum classis; tunc $M a$ legatur \emph{maximus inter} $a$, et $\mini a$ legatur \emph{minimus inter a}.
\s
Let $A \smallIn \setOfSets \mathbb{N}$, that is, let $A$ be a set of numbers; then $\max(A)$ is read \emph{greatest among} $A$, and $\min(A)$ is read \emph{least among A}.
\s*
\phantomsection
\addcontentsline{toc}{subsection}{Definitions}
\peanoHeadingSmall{Definitiones.}
\s
\peanoHeadingSmall{Definitions.}
\end{translateTwoCol}

\begin{translateSixCol}{0.08}{0.39}{0.02}{0.08}{0.39}{0.02}
\raggedright
1. \s $a \smallIn \K \N \p \C : \M a = [x \smallIn] (x \smallIn a \ppp a \p \smallIn \larger x : = \abs)$.\s
\s
1. \s $A \in \setOfSets \mathbb{N} \rightarrow  \max(A) = \big\{x \big| x \in A \wedge [(A \cap \{ z | z > x \}) = \bot]\big\}$.\s
\s*
2. \s $a \smallIn \K \N \p \C : \mini a = [x \smallIn] (x \smallIn a \ppp a \p \smallIn \smaller x : = \abs)$.\s
\s
2. \s $A \in \setOfSets \mathbb{N} \rightarrow \min(A) = \big\{x \big| x \in A \wedge [(A \cap \{ z | z < x \}) = \bot]\big\}$.\s
\end{translateSixCol}

\peanoPage{7} % page-number 7

\begin{translateTwoCol}
\centering
\phantomsection
\addcontentsline{toc}{subsection}{Theorems}
\peanoHeadingSmall{Theoremata.}
\s
\peanoHeadingSmall{Theorems.}
\end{translateTwoCol}

\begin{translateSixCol}{0.08}{0.39}{0.02}{0.08}{0.39}{0.02}
\raggedright
3. \s $n \smallIn \N \p a \smallIn \K \N \p a \no = \abs \p a \such \larger n = \abs : \C \p M a \smallIn N$.\s
\s
3. \s \todo... \s
\s*
\emph{Dem}. \s $a \smallIn \K \N \p a \no = \abs \p a \such \larger 1 = \abs : \C : a = 1 : \C \p M a = 1 : \C \p M a \smallIn N$.\s (1)
\s
\s \s
\s*
\s $(1) \C : 1 \smallIn [n \smallIn]$ (Hp $\C$ Ts). \s (2)
\s
\s \s
\s*
\s $n \smallIn \N \p a \smallIn \K \N \p a \such \larger n + 1 = \abs \p n + 1 \smallIn a : \C : n + 1 = M a : \C : M a \smallIn N$. \s (3)
\s
\s \s
\s*
\s $n \smallIn \N \p a \smallIn \K \N \p a \such \larger n + 1 = \abs \p n + 1 \no \smallIn a : \C : a \such \larger n = \abs$. \s (4)
\s
\s \s
\s*
\s $n \smallIn [n \smallIn]$ (Hp $\C$ Ts) $\p a \smallIn \K \N \p a \such > n + 1 = \abs \p n + 1 \no \smallIn a : \C : M a \smallIn N$. \s (5)
\s
\s \s
\s*
\s $n \smallIn [n \smallIn]$ (Hp $\C$ Ts) $. (6) : \C \p (n+ 1) \smallIn [n \smallIn]$ (Hp $\C$ Ts). \s (7)
\s
\s \s
\s*
\s $(2)(7) \p \S1$ P9 $: \C : n \smallIn \N \p \C \p$ Hp $\C$ Ts. \s (Th.)
\s
\s \s
\s*
4. \s $a \smallIn \K \N \p a \no = \abs : \C \p \mini a \smallIn N$. \s
\s
\s \s
\s*
5. \s $a \smallIn \K \N \p \C : \mini a = M [x \smallIn] (a \such \smaller x = \abs)$. \s
\s
\s \s
\end{translateSixCol}

\begin{translateTwoCol}
\centering
\phantomsection
\addcontentsline{toc}{section}{\S4. Multiplication}
\peanoHeadingMedium{\S4. De multiplicatione.}
\s
\peanoHeadingMedium{\S4. Multiplication.}
\s*
\phantomsection
\addcontentsline{toc}{subsection}{Definitions}
\peanoHeadingSmall{Definitiones.}
\s
\peanoHeadingSmall{Definitions.}
\end{translateTwoCol}

\begin{translateSixCol}{0.08}{0.39}{0.02}{0.08}{0.39}{0.02}
\raggedright
1. \s $a \smallIn \N \p \C \p a \times 1 = a$.\s
\s
1. \s $a \in \mathbb{N} \rightarrow a \times 1 = a$.\s
\s*
2. \s $a,b \smallIn \N \p \C \p a \times (b + 1) = a \times b + a$. \s
\s
2. \s $a,b \in \mathbb{N} \rightarrow a \times (b + 1) = a \times b + a$. \s
\s*
\s $a b = a \times b; a b + c = (a b) + c; a b c = (a b) c$. \s
\s
\s $a b = a \times b; a b + c = (a b) + c; a b c = (a b) c$. \s
\end{translateSixCol}

\begin{translateTwoCol}
\centering
\phantomsection
\addcontentsline{toc}{subsection}{Theorems}
\peanoHeadingSmall{Theoremata.}
\s
\peanoHeadingSmall{Theorems.}
\end{translateTwoCol}

\begin{translateSixCol}{0.08}{0.39}{0.02}{0.08}{0.39}{0.02}
\raggedright
3. \s $a,b \smallIn \N \p \C \p ab \smallIn N$.\s
\s
3. \s \todo... \s
\s*
\emph{Dem}. \s $a,b \smallIn \N  \p$ P1 $: \C : a \times 1 \smallIn \N : \C \p 1 \smallIn [b \smallIn]$ Ts. \s
\s
\s \s
\s*
\s $a,b \smallIn \N \p b \smallIn [b \smallIn]$ Ts $: \C : a \times b \smallIn \N \p \S 1$ P19 $: \C : ab + a \smallIn \N \p$ P1 $: \C : a (b + 1) \smallIn \N : \C : b + 1 \smallIn [b \smallIn]$ Ts. \s (2)
\s
\s \s
\s*
\s $(1)(2) \p \C \p$ Theor. \s
\s
\s \s
\end{translateSixCol}

\peanoPage{8} % page-number 8

\begin{translateSixCol}{0.1}{0.3}{0.1}{0.1}{0.3}{0.1}
\raggedright
4. \s $a,b,c \smallIn \N \p \C \p (a + b) c = a c + b c$.\s
\s
\s \s
\s*
\s \s
\s
\s \commentary{Euclid's \emph{Elements} was \emph{the} mathematical text from 300 BCE up until Peano's era.  Being able to prove propositions from it was an indication that Peano's axioms matched the prevailing notions of mathematics.} \s 
\s*
\emph{Nota.} \s Haec est prop. $5^a$ Euclidis elem. libri VII. \s
\s
\emph{Note.} \s This is Proposition \#5 of Euclid's \emph{Elements}, Book VII. \s
\s*
\emph{Dem.} \s $a,b \smallIn \N \p$ P1 $: \C : 1 \smallIn [c \smallIn]$ Ts. \s (1)
\s
\s \s
\s*
\s $a,b,c \smallIn \N \p c \smallIn [c \smallIn]$ Ts $: \C : (a + b) c = a c + b c \p \S1$ P22 $: \C : (a + b) c + a + b = ac + bc + a + b \p$ P2 $: \C : (a + b) (c + 1) = a (c + 1) + b (c + 1) : \C : c + 1 \smallIn [c \smallIn]$ Ts. \s (2)
\s
\s \s
\s*
\s $(1)(2) \p \C \p$ Theor. \s
\s
\s \s
\s*
5. \s $a \smallIn \N \p \C \p 1 \times a = a \p$ \s
\s
\s \s
\s*
\emph{Dem.} \s $1 \smallIn [a \smallIn]$ Ts. \s (1)
\s
\s \s
\s*
\s $a \smallIn [a \smallIn]$ Ts $\p \C \p 1 \times a = a \p \C \p 1 \times a + 1 = a + 1 \p \times \p 1 \times (a + 1) = a + 1 \p \C \p a + 1 \smallIn [a \smallIn]$ Ts. \s (2)
\s
\s \s
\s*
\s $(1)(2) \p \C \p$ Theor. \s
\s
\s \s
\s*
6. \s $a,b \smallIn \N \p \C \p b a + a = (b + 1) a$. \s
\s
\s \s
\s*
7. \s $a,b \smallIn \N \p \C \p ab = ba$. \s (Eucl. VII, 16)
\s
\s \s
\s*
\emph{Dem.} \s $a \smallIn \N \p$ P5 $\p$ P1 $: \C \p a \times 1 = a = 1 \times a : \C : 1 \smallIn [b \smallIn]$ Ts. \s (1)
\s
\s \s
\s*
\s $a,b \smallIn \N \p b \smallIn [b \smallIn]$ Ts $: \C : ab = ba : \C : ab + a = ba + a \p$ P1 $\p $P6$ : \C : a (b + 1) = (b + 1) a : \C : b + 1 \smallIn [b \smallIn]$ Ts. \s (2)
\s
\s \s
\s*
\s $(1)(2) \p \C \p$ Theor. \s
\s
\s \s
\s*
8. \s $a,b,c \smallIn \N \p \C \p a (b + c) = ab + ac$. \s
\s
\s \s
\s*
\emph{Dem.} \s P4 $\p$ P7 $: \C \p$ Theor. \s
\s
\s \s
\s*
9. \s $a,b,c \smallIn \N \p a = b : \C : ac = bc$. \s
\s
\s \s
\s*
\emph{Dem.} \s $a,b \smallIn \N \p a = b :: \C :: 1 \smallIn [c \smallIn]$ Ts $\ppp c \smallIn [c \smallIn]$ Ts $\p \C : ac = bc \p a = b : \C : ac + a = bc + b : \C : a (c + 1) = b (c + 1) : \C : c + 1 \smallIn [c \smallIn]$ Ts $:: \C : c \smallIn \N \p \C \p$ Ts. \s
\s
\s \s
\s*
10. \s $a,b,c \smallIn \N : a \smaller b : \C \p (b-a)c = bc - ac$. \s (Eucl. VII, 7)
\s
\s \s
\s*
\emph{Dem.} \s Hyp $\p \C : b - a \smallIn \N \p (b-a) + a = b : \C : (b-a)c + ac=bc : \C : (b-a)c = bc-ac$. \s
\s
\s \s
\s*
11. \s $a,b,c \smallIn \N \p a \smaller b : \C : ac \smaller bc$. \s
\s
\s \s
\s*
\emph{Dem.} \s Hyp $\p \C : b - a \smallIn \N \p$ P3 $: \C : (b-a) c \smallIn \N \p$ P10 $: \C : bc - ac \smallIn \N : \C$ Thesis. \s
\s
\s \s
\s*
12. \s $a,b,c \smallIn \N \p \C \ppp a \smaller b \p = \p ac \smaller bc : a = b \p = \p ac = bc : a \larger b \p = \p ac \larger bc$. \s
\s
\s \s
\s*
13. \s $a,b,a',b' \smallIn \N \p a \smaller a' \p b \smaller b' : \C : ab \smaller a'b'$. \s
\s
\s \s
\s*
14. \s $a,b \smallIn \N : \C \p ab \p \larger \cup = \p a$. \s
\s
\s \s
\s*
15. \s $a,b,c \smallIn \N \p \C \p a(bc)=abc$. \s
\s
\s \s
\end{translateSixCol}

\peanoPage{9} % page-number 9

\begin{translateSixCol}{0.1}{0.3}{0.1}{0.1}{0.3}{0.1}
\raggedright
\emph{Dem.} \s $a,b \smallIn \N \p$ P1 $: \C : 1 \smallIn [c \smallIn]$ Ts. \s (1)
\s
\s \s
\s*
\s $a,b,c \smallIn \N \p c \smallIn [c \smallIn]$ Ts $: \C : a(bc) = abc : \C : a(bc) + ab = abc + ab : \C : a (bc + b) = ab (c+ 1) : \C : a (b (c +1)) = ab (c + 1) : \C : c + 1 \smallIn [c \smallIn]$ Ts. \s (2)
\s
\s \s
\s*
\s $(1)(2) \p \C \p$ Theor. \s
\s
\s \s
\end{translateSixCol}

\begin{translateTwoCol}
\centering
\phantomsection
\addcontentsline{toc}{section}{\S5. Powers}
\peanoHeadingMedium{\S5. De potestatibus.}
\s
\peanoHeadingMedium{\S5. Powers.}
\s*
\phantomsection
\addcontentsline{toc}{subsection}{Definitions}
\peanoHeadingSmall{Definitiones.}
\s
\peanoHeadingSmall{Definitions.}
\end{translateTwoCol}

\begin{translateSixCol}{0.1}{0.3}{0.1}{0.1}{0.3}{0.1}
\raggedright
1. \s $a \smallIn \N \p \C \p a^1 = a$. \s
\s
1. \s $a \in \mathbb{N} \rightarrow a^1 = a$. \s
\s*
2. \s $a,b \smallIn \N \p \C \p a^{b+1} = a^{b} a$. \s
\s
2. \s $a,b \in \mathbb{N} \rightarrow a^{b+1} = a^{b} a$. \s
\end{translateSixCol}

\begin{translateTwoCol}
\centering
\phantomsection
\addcontentsline{toc}{subsection}{Theorems}
\peanoHeadingSmall{Theoremata.}
\s
\peanoHeadingSmall{Theorems.}
\end{translateTwoCol}

\begin{translateSixCol}{0.1}{0.3}{0.1}{0.1}{0.3}{0.1}
\raggedright
3. \s $a,b \smallIn \N \p \C \p a^{b} \smallIn N$. \s
\s
3. \s \todo... \s
\s*
\emph{Dem.} \s $a \smallIn \N \p$ P1 $: \C \p 1 \smallIn [b \smallIn]$ Ts. \s (1)
\s
\s \s
\s*
\s $a,b \smallIn \N \p b \smallIn [b \smallIn]$ Ts $: \C : a^b \smallIn \N \p \S4 $P3$ : \C : a^b a \smallIn \N \p$ P1$: \C : a^{b+1} \smallIn \N : \C : b + 1 \smallIn [b \smallIn]$ Ts. \s (2)
\s
\s \s
\s*
\s $(1)(2) \p \C \p.$ Theor. \s
\s
\s \s
\s*
4. \s $a \smallIn \N \p \C \p 1^{a} = 1$. \s
\s
\s \s
\s*
5. \s $a,b,c \smallIn \N \p \C \p a^{b + c} = a^b a^c$. \s
\s
\s \s
\s*
6. \s $a,b,c \smallIn \N \p \C \p (ab)^c = a^c b^c$. \s
\s
\s \s
\s*
7. \s $a,b,c \smallIn \N \p \C \p (a^b)^c = a^{bc}$. \s
\s
\s \s
\s*
8. \s $a,b,c \smallIn \N \p \C \ppp a \smaller b \p = \p a^c \smaller b^c : a = b \p = \p a^c = b^c : a \larger b \p = \p a^c \larger b^c$. \s
\s
\s \s
\s*
9. \s $a,b,c \smallIn \N \p a \larger 1 \p \C \ppp b \smaller c \p = \p a^b \smaller a^c : b = c \p = \p a^b = a^c : b \larger c \p = \p a^b \larger a^c$.
\end{translateSixCol}

\begin{translateTwoCol}
\centering
\phantomsection
\addcontentsline{toc}{section}{\S6. Division}
\peanoHeadingMedium{\S6. De divisione.}
\s
\peanoHeadingMedium{\S6. Division.}
\s*
\phantomsection
\addcontentsline{toc}{subsection}{Explications}
\peanoHeadingSmall{Explicationes.}
\s
\peanoHeadingSmall{Explications.}
\end{translateTwoCol}

\begin{translateEightCol}{0.12}{0.06}{0.06}{0.26}{0.12}{0.06}{0.06}{0.26}
\centering Signum \s \raggedright $/$ \s \centering legatur \s \raggedright \emph{divisus per.}
\s
\centering The~symbol \s \raggedright $/$ \s \centering is~read \s \raggedright \emph{divided by.}
\s*
\centering $\dittoMarkLatin$ \s \raggedright $\D$ \s \centering $\dittoMarkLatin$ \s \raggedright \emph{dividit}, sive \emph{est divisor.}
\s
\centering \dittoMarkEnglish \s \raggedright $|$ \s \centering \dittoMarkEnglish \s \raggedright \emph{divides}, or \emph{is a divisor of.}
\s*
\centering $\dittoMarkLatin$ \s \raggedright $\mult$ \s \centering $\dittoMarkLatin$ \s \raggedright \emph{est multiplex.}
\s
\centering \dittoMarkEnglish \s \raggedright $\mult$ \s \centering \dittoMarkEnglish \s \raggedright \emph{is a multiple of.}
\s*
\centering $\dittoMarkLatin$ \s \raggedright $\Np$ \s \centering $\dittoMarkLatin$ \s \raggedright \emph{numerus primus.}
\s
\centering \dittoMarkEnglish \s \raggedright $\mathbb{P}$ \s \centering \dittoMarkEnglish \s \raggedright \emph{prime number.}
\s*
\centering $\dittoMarkLatin$ \s \raggedright $\primeWith$ \s \centering $\dittoMarkLatin$ \s \raggedright \emph{est primus cum.}
\s
\centering \dittoMarkEnglish \s \raggedright $\primeWith$ \s \centering \dittoMarkEnglish \s \raggedright \emph{is prime with.}
\end{translateEightCol}

\peanoPage{10} % page-number 10

\begin{translateTwoCol}
\centering
\phantomsection
\addcontentsline{toc}{subsection}{Definitions}
\peanoHeadingSmall{Definitiones.}
\s
\peanoHeadingSmall{Definitions.}
\end{translateTwoCol}

\begin{translateSixCol}{0.1}{0.35}{0.05}{0.1}{0.35}{0.05}  % wider than usual for definition 5
\raggedright
1. \s $a,b \smallIn \N \p \C : b / a = \N [x \smallIn] (xa=b)$. \s
\s
1. \s $a,b \in \mathbb{N} \rightarrow b / a = \mathbb{N} \cap \{x | xa=b \}$. \s
\s*
2. \s $a,b \smallIn \N \p \C : a \D b \p = \p b / a \no = \abs$. \s
\s
2. \s $a,b \in \mathbb{N} \rightarrow a | b = (b / a \neq \varnothing)$. \s
\s*
3. \s $a,b \smallIn \N \p \C : b \mult a \p = \p a \D b$. \s
\s
3. \s $a,b \in \mathbb{N} \rightarrow b \mult a = (a | b)$. \s  %%%TODO: There is no modern symbol for this, is there?
\s*
4. \s $\Np = \N [x \smallIn] (\such \D x \p \such \larger 1 \p \such \smaller x : = \abs)$. \s
\s
4. \s $\mathbb{P} = \mathbb{N} \cap \big\{x \big| \big( \{ z | (z | x) \} \cap \{z | z > 1 \} \cap \{z | z < x \} \big) = \varnothing \big\}$. \s
\s*
5. \s $a,b \smallIn \N \p \C :: a \primeWith b \ppp = \ppp \such \D a \p \such \D b \p \such \larger 1 : = \abs$. \s
\s  %%% TODO: create new command for \primeWith with \mathbin 
5. \s $a,b \in \mathbb{N} \rightarrow a \primeWith b = \big[\big(\{z | (z | a) \} \cap \{z | (z | b) \} \cap \{ z | z > 1 \} \big) = \varnothing \big]$. \s
\s*
6. \s $a,b \smallIn \N \p \C \ppp \such \D (a,b) : = : \such \D a \p \cap \p \such \D b$. \s
\s
6. \s $a,b \in \mathbb{N} \rightarrow \such \D (a,b) = \big( \{ z | ( z | a) \} \cap \{ z | (z | b) \} \big)$. \s %%%TODO: There is no modern symbol for this, is there?
\s*
7. \s $a,b \smallIn \N \p \C \ppp \such \mult (a,b) : = : \such \mult a \p \cap \p \such \mult b$. \s
\s
7. \s $a,b \in \mathbb{N} \rightarrow \such \mult (a,b) = \big( \{ z | z \mult a \} \cap \{z | z \mult b \} \big)$. \s %%%TODO: There is no modern symbol for this, is there?
\s*
\s $ab / c = (ab)/c; a/b/c = (a/b)/c; a/b \times c = (a/b)c$. \s
\s
\s $ab / c = (ab)/c; a/b/c = (a/b)/c; a/b \times c = (a/b)c$. \s
\end{translateSixCol}

\begin{translateTwoCol}
\centering
\phantomsection
\addcontentsline{toc}{subsection}{Theorems}
\peanoHeadingSmall{Theoremata.}
\s
\peanoHeadingSmall{Theorems.}
\end{translateTwoCol}

\columnratio{0.1, 0.4, 0.1, 0.4}
\begin{paracol}{4}
\raggedright
\emph{Nota.} \s Haec theoremata ut in substractione demonstrantur.
\s
\emph{Note.} \s These theorems are proved as for subtraction.
\end{paracol}

\begin{translateSixCol}{0.1}{0.3}{0.1}{0.1}{0.3}{0.1}
\raggedright
8. \s $a,b,a',b' \smallIn \N \p a = a' \p b = b' : \C \p a / b = a' / b'$. \s
\s
8. \s \todo... \s
\s*
9. \s $a,b,a',b' \smallIn \N \p a = a' \p b = b' : \C : a \D b \p = \p a' \D b'$. \s
\s
\s \s
\s*
10. \s $a,b,c \smallIn \N \p \C : ac = b \p = \p c = b / a$. \s
\s
\s \s
\s*
11. \s $a,b \smallIn \N \p \C : a \D b \p = \p b / a \smallIn N$. \s
\s
\s \s
\s*
12. \s $a \smallIn \N \p \C \p a / 1 = a$. \s
\s
\s \s
\s*
13. \s $a \smallIn \N \p \C \p a / a = 1$. \s
\s
\s \s
\s*
14. \s $a \smallIn \N \p \C \p 1 \D a$. \s
\s
\s \s
\s*
15. \s $a \smallIn \N \p \C \p a \D a$. \s
\s
\s \s
\s*
16. \s $a,b \smallIn \N \p ab / b = a$. \s
\s
\s \s
\s*
17. \s $a,b \smallIn \N \p a \D b : \C \p a (b/a) = b$. \s
\s
\s \s
\s*
18. \s $a,b,c \smallIn \N \p c \D b : \C \p a (b/c) = ab / c$. \s
\s
\s \s
\s*
19. \s $a,b,c \smallIn \N \p a \mult bc : \C : a / (bc) = a / b / c$. \s
\s
\s \s
\s*
20. \s $a,b,c \smallIn \N \p a \mult b \p b \mult c : \C \p a / (b/c) = a / b \times c$. \s
\s
\s \s
\s*
21. \s $a,m,n \smallIn \N \p m \larger n : \C \p a^m / a^n = a^{m-n}$. \s
\s
\s \s
\s*
22. \s $a,b \smallIn \N \p \C \p a \D ab$. \s
\s
\s \s
\s*
23. \s $a,b,c \smallIn \N \p a \D b \p b \D c : \C \p a \D c$. \s
\s
\s \s
\s*
24. \s $a,b,c \smallIn \N \p a \D b \D c : \C \p c / a \mult c / b$. \s
\s
\s \s
\s*
25. \s $a,b,c \smallIn \N \p c \D a \p c \D b : \C \p (a + b) / c = a / c + b / c$. \s
\s
\s \s
\s*
26. \s $a,b,c \smallIn \N \p c \D a \p c \D b \p a \larger b : \C : (a - b) / c = a / c - b / c$. \s
\s
\s \s
\s*
27. \s $a,b,c, \smallIn \N \p c \D a \p c \D b : \C \p c \D a + b$. \s
\s
\s \s
\s*
28. \s $a,b,c \smallIn \N \p c \D a \p c \D b \p a \larger b : \C \p c \D a - b$. \s
\end{translateSixCol}

\peanoPage{11} % page-number 11

\begin{translateSixCol}{0.1}{0.3}{0.1}{0.1}{0.3}{0.1}
\raggedright
29. \s $a,b,c,m,n \smallIn \N \p c \D a \p c \D b : \C \p c \D ma + nb$. \s
\s
\s \s
\s*
30. \s $a,b,c,m,n \smallIn \N \p c \D a \p c \D b \p ma \larger nb : \C \p c \D ma - nb$. \s
\s
\s \s
\s*
31. \s $a,b \smallIn \N \p a \D b : \C : a \p \smaller \cup = \p b$. \s
\s
\s \s
\s*
\emph{Dem.} \s Hyp . P11 $\p$ P17 $\p$ \S4 P14 $: \C : b / a \smallIn \N \p a (b / a) = b \p a \smaller \cup = a (b / a) : \C \p$ Thesis. \s
\s
\s \s
\s*
32. \s $a,b \smallIn \N \p a \D b \p b \D a : \C \p a = b$. \s
\s
\s \s
\s*
33. \s $a \smallIn \N \p \C \p M \such \D a = a$. \s
\s
\s \s
\s*
34. \s $a,b \smallIn \N \p a > b : \C \p \such  \D (a,b) = \such \D (b,a - b)$. \s
\s
\s \s
\s*
\emph{Dem.} \s Hyp. P28 $: \C \ppp x \D a \p x \D b : \C : x \D b \p x \D (a-b)$ \s (1)
\s
\s \s
\s*
\s Hyp. P27 $: \C \ppp x \D b \p x \D (a-b) : \C : x \D b \p x \D (b+(a-b)) : \C : x \D b \p x \D a$. \s (2)
\s
\s \s
\s*
\s $(1)(2) \C :$ Hyp. $\C \ppp x \D a \p x \D b : = : x \D b \p x D(a-b)$. \s (Th.)
\s
\s \s
\s*
35. \s $a,b \smallIn \N \p \C : M \such \D (a,b) \smallIn N$. \s
\s
\s \s
\s*
\emph{Dem.} \s $1 \D a \p 1 \D b : \C : \such \D (a,b) \no = \abs$. \s (1)
\s
\s \s
\s*
\s $\such \D (a,b) \p \such \larger a : = \abs$. \s (2)
\s
\s \s
\s*
\s $(1)(2) \p \S3$ P3 $: \C \p$ Th. \s
\s
\s \s
\s*
36. \s $a,b \smallIn \N \p \C \p \such \D (a,b) = \such \D M \such \D (a,b).$ \s (Eucl. VII, 2)
\s
\s \s
\s*
\emph{Dem.} \s $k = \N [c \smallIn]$ (Hp. $a \smaller c \p b \smaller c : \C \p$ Ts.). \s (1)
\s
\s \s
\s*
\s $a \smallIn \N \p b \smallIn \N \p a \smaller 1 \p b \smaller 1 : = \abs$. \s (2)
\s
\s \s
\s*
\s $ (1)(2) \p \C \p 1 \smallIn \K$. \s (3)
\s
\s \s
\s*
\s $a,b \smallIn \N \p a \smaller c + 1 \p b \smaller c + 1 : \C \ppp a \smaller c \p b \smaller c : \cup : a = c \p b \smaller c : \cup : a \smaller c \p b = c : \cup : a = c \p b = c$. \s (4)
\s
\s \s
\s*
\s $c \smallIn k \p a,b \smallIn \N \p a < c \p b < c : \C :$ Ts. \s (5)
\s
\s \s
\s*
\s $c \smallIn k \p a,b \smallIn \N \p a = c \p b \smaller c : \C : c \smallIn k \p b \smaller c /p a - b \smaller c \p \such \D (a,b) = \such \D (b,a-b) : \C : \such \D (b, a-b) = \such \D m \such \D (b,a-b) : \C : \such \D (a,b) = \such \D M \such \D (a,b) : \C :$ Ts. \s (6)
\s
\s \s
\s*
\s $(a,b) [b,a] (6) \C \p c \smallIn k \p a,b \smallIn \N \p a \smaller c \p b = c : \C :$ Ts. \s (7)
\s
\s \s
\s*
\s $c \smallIn k \p a,b \smallIn \N \p a = c \p b = c : \C : \such \D (a,b) = \such \D c = \such \D M \such \D c = \such \D M \such \D (a,b) : \C :$ Ts. \s (8)
\s
\s \s
\s*
\s $(4)(5)(6)(7)(8) \p \C \p c \smallIn k \p a,b \smallIn \N \p a \smaller c + 1 \p b \smaller c + 1 : \C :$ Ts. \s (9)
\s
\s \s
\s*
\s $(9) \C \p c \smallIn k \p \C \p (c + 1) \smallIn k$. \s (10)
\s
\s \s
\s*
\s $(1)(10) \p \C \ppp c \smallIn \N \p$ Hp. $a \smaller c \p b \smaller c : \C :$ Ts. \s (11)
\s
\s \s
\s*
\s $(a + b) [c] (11) \p \C :$ Hp. $ \C \p$ Ts. \s (Th.)
\s
\s \s
\s*
37. \s $a,b,m \smallIn \N \p \C \p M \such \D (am, bm) = m \times M \such \D (a,b)$. \s
\s
\s \s
\end{translateSixCol}

\peanoPage{12} % page-number 12

\begin{translateTwoCol}
\centering
\phantomsection
\addcontentsline{toc}{section}{\S7. Various theorems}
\peanoHeadingMedium{\S7. Theoremata varia.}
\s
\peanoHeadingMedium{\S7. Various theorems.}
\end{translateTwoCol}

\begin{translateSixCol}{0.1}{0.3}{0.1}{0.1}{0.3}{0.1}
\raggedright
1. \s $a,b \smallIn \N \p a^2 + b^2 \mult 7 : \C : a \mult 7 \p b \mult 7$. \s
\s
1. \s \todo... \s
\s*
2. \s $x \smallIn \N \p \C \p x (x + 1) \mult 2$. \s
\s
\s \s
\s*
3. \s $x \smallIn \N \p \C \p x (x + 1) (x + 2) \mult 6$. \s
\s
\s \s
\s*
4. \s $x \smallIn \N \p \C \p x (x + 1) (2x + 1) \mult 6$. \s
\s
\s \s
\s*
5. \s $x \smallIn \N \p \C : x \p \primeWith \p x + 1$. \s
\s
\s \s
\s*
6. \s $x \smallIn \N \p \C : 2x - 1 \p \primeWith \p 2x + 1$. \s
\s
\s \s
\s*
7. \s $x \smallIn \N \p \C \p (2x + 1)^2 - 1 \mult 8$. \s
\s
\s \s
\s*
8. \s $a \smallIn \N \p a \larger 1 : \C \ppp \Np \p \such \larger 1 \p \such \D a : \no = \abs$. \s (Eucl. VII, 31)
\s
\s \s
\s*
9. \s $a,b \smallIn \N \ppp b^2 \larger a \ppp \such \D a \p \such \larger 1 \p \such \smaller b : = \abs :: \C \p a \smallIn \Np$. \s
\s
\s \s
\s*
10. \s $a,b \smallIn \N \p a \smallIn \Np \p a \no \D b : \C : a \primeWith b$. \s (Eucl. VII, 29)
\s
\s \s
\s*
11. \s $a,b,c \smallIn \N \p a \D bc \p a \primeWith b : \C \p a \D c$. \s
\s
\s \s
\s*
12. \s $a,b \smallIn \N \p m = M \such \D (a,b) : \C : a / m \primeWith \p b / m$. \s
\s
\s \s
\s*
13. \s $ a \smallIn \Np \p b,c \smallIn \N \p a \D b c : \C : a \D b \p \cup \p a \D c$. \s (Eucl. VII, 30)
\s
\s \s
\s*
14. \s $a \smallIn \Np \p b,n \smallIn \N : \C : a \D b^n \p = \p a \D b$. \s (Eucl. IX, 12)
\s
\s \s
\s*
15. \s $a,b,c \smallIn \N \p a \primeWith b \p c \D a : \C : c \primeWith b$. \s (Eucl. VII, 23)
\s
\s \s
\s*
16. \s $ a,b,c \smallIn \N \p \C \ppp a \primeWith b \p a \primeWith c : = : a \primeWith bc$. \s (Eucl. VII, 24)
\s
\s \s
\s*
17. \s $a,b,c \smallIn \N \p b \primeWith c \p b \D a \p c \D a : \C \p bc \D a$. \s
\s
\s \s
\s*
18. \s $a,b,c \smallIn \N \ p a \primeWith b : \C : \such \D (ac, b) = \such \D (c,b)$. \s
\s
\s \s
\s*
19. \s $a,b \smallIn \N \p \C \p \mini \such \mult (a,b) \smallIn N$. \s
\s
\s \s
\s*
20. \s $a,b \smallIn \N \p \C \p \mini \such \mult (a,b) = ab / M \such \D (a,b)$. \s (Eucl. VII, 34)
\s
\s \s
\s*
21. \s $ a,b,c \smallIn \N \p c \mult a \p c \mult b : \C : c \mult \mini \such \D (a,b)$. \s (Eucl. VII, 35)
\s
\s \s
\s*
22. \s $x \smallIn \N \p x < 41 : \C \p 41 - x + x^2 \smallIn \Np$. \s
\s
\s \s
\s*
23. \s $M \p \Np : = \abs$. \s (Eucl. IX, 20)
\s
\s \s
\s*
23. \s $n \smallIn \Np \p a \smallIn \N \p a \no \mult n : \C \p a^{n-1} - 1 \mult n$. \s (Fermat)
\end{translateSixCol}

\begin{translateTwoCol}
\centering
\phantomsection
\addcontentsline{toc}{section}{\S8. Rational numbers}
\peanoHeadingMedium{\S8. Numerorum rationes.}
\s
\peanoHeadingMedium{\S8.Rational numbers.}
\s*
\phantomsection
\addcontentsline{toc}{subsection}{Explications}
\peanoHeadingSmall{Explicationes.}
\s
\peanoHeadingSmall{Explications.}
\end{translateTwoCol}

\begin{translateTwoCol}
\raggedright
Si $p,q \smallIn \N$, tunc $\frac{p}{q}$ legitur \emph{ratio numeri} $p$ \emph{numero} $q$.
\s
If $p,q \smallIn \mathbb{N}$, then $\frac{p}{q}$ is read \emph{the ratio of the number} $p$ {to the number} $q$.
\s*
Signum $\R$ legitur \emph{duorum numerorum ratio}, et indicat numeros rationales positivos.
\s
The symbol $\mathbb{Q}^+$ is read \emph{ratio of two numbers}, and indicates the positive rational numbers.
\end{translateTwoCol}

\peanoPage{13} % page-number 13

\begin{translateTwoCol}
\centering
\phantomsection
\addcontentsline{toc}{subsection}{Definitions}
\peanoHeadingSmall{Definitiones.}
\s
\peanoHeadingSmall{Definitions.}
\end{translateTwoCol}

\begin{translateSixCol}{0.1}{0.3}{0.1}{0.1}{0.3}{0.1}
\raggedright
1. \s $m,p,q \smallIn \N \p \C \p m \frac{p}{q} = m p / q$. \s
\s
1. \s $m,p,q \in \mathbb{N} \rightarrow m \frac{p}{q} = m p / q$. \s
\s*
2. \s $p,q,p',q' \smallIn \N \p \C :: \frac{p}{q} = \frac{p'}{q'} \p = \ppp x \smallIn \N \p x \frac{p}{q}, x \frac{p'}{q'} \smallIn \N : \C_x \p x \frac{p}{q} = x \frac{p'}{q'}$. \s
\s
2. \s $p,q,p',q' \in \mathbb{N} \rightarrow (\frac{p}{q} = \frac{p'}{q'}) = [ x \in \mathbb{N} \wedge (x \frac{p}{q}, x \frac{p'}{q'} \in \mathbb{N}) \xrightarrow[\forall x]{} ( x \frac{p}{q} = x \frac{p'}{q'})]$. \s
\s*
3. \s $R = :: [x \smallIn] \ppp p, q \smallIn \N \p \frac{p}{q} = x : \no = \abs$. \s
\s
3. \s $\mathbb{Q}^+ = \{x | [(p, q \in \mathbb{N}) \wedge \frac{p}{q} = x] \neq \bot \}$. \s
\s*
4. \s $p \smallIn \N \p \C \p \frac{p}{1} = p$. \s
\s
4. \s $p \in \mathbb{N} \rightarrow \frac{p}{1} = p$. \s
\end{translateSixCol}

\begin{translateTwoCol}
\centering
\phantomsection
\addcontentsline{toc}{subsection}{Theorems}
\peanoHeadingSmall{Theoremata.}
\s
\peanoHeadingSmall{Theorems.}
\end{translateTwoCol}

\begin{translateSixCol}{0.1}{0.3}{0.1}{0.1}{0.3}{0.1}
\raggedright
5. \s $p,q,p',q' \smallIn \N \p \C :: \frac{p}{q} = \frac{p'}{q'} \p = \p pq' = p'q$. \s (Eucl. VII, 19)
\s
5. \s \todo... \s
\s*
\emph{Dem.} \s Hp. $\frac{p}{q} = \frac{p'}{q'} : \C \ppp qq', qq' \frac{p}{q}, qq' \frac{p'}{q'} \smallIn \N \p$ P2 $\ppp \C \ppp qq' \frac{p}{q} = qq' \frac{p'}{q'} \p qq' \frac{p}{q} = pq' \p qq' \frac{p'}{q'} = p'q \ppp \C \ppp pq' = p'q$. \s (1)
\s
\s \s
\s*
\s Hp. $pq' = p'q \ppp \C \ppp x \smallIn \N \p x \frac{p}{q}, x \frac{p'}{q'} \smallIn \N : \C_x : x p'q' = x p'q : \C : (x \frac{p}{q}) qq' = (x \frac{p'}{q'}) qq' : \C : x \frac{p}{q} = x \frac{p'}{q'}$. \s (2)
\s
\s \s
\s*
\s $(1)(2) \p \C \p$ Th. \s
\s
\s \s
\s*
6. \s $m,p,q \smallIn \N \p \C \p \frac{p}{q} = \frac{mp}{mq}$. \s (Eucl. VII, 17)
\s
\s \s
\s*
7. \s $p,q \smallIn \N \p m \smallIn \N \p m \D p \p m \D q : \C \p \frac{p}{q} = \frac{p/m}{q/m}$. \s
\s
\s \s
\s*
8. \s $p,q,p',q' \smallIn \N \p p \primeWith q \p p' \primeWith q' \p \frac{p}{q} = \frac{p'}{q'} : \C : p = p' \p q =q'$. \s
\s
\s \s
\s*
9. \s $p,q,p',q' \smallIn \N \p p' \primeWith q' \p \frac{p}{q} = \frac{p'}{q'} : \C : p' / p = q' / q = M \such \D (p,q)$. \s
\s
\s \s
\s*
10. \s $p,q,p',q' \smallIn \N \p \frac{p}{q} = \frac{p'}{q'} \p p \primeWith q \p q' \smaller q : = \abs$. \s (Eucl. VII, 21)
\s
\s \s
\s*
11. \s $p,q,p',q' \smallIn \N : \C : \frac{p}{q} = \frac{p'}{q'} \p = \p \frac{p}{p'} = \frac{q}{q'} \p = \p \frac{q}{p} = \frac{q'}{p'}$. \s (Eucl. VII, 13)
\s
\s \s
\s*
12. \s $p,q \smallIn \N \p \C :: [m \smallIn] : m \smallIn \N \p m \frac{p}{q} \smallIn \N \ppp \no = \abs$. \s
\s
\s \s
\s*
12'. \s $a \smallIn \R \p \C :: [m \smallIn] : m \smallIn \N \p ma \smallIn \N \ppp \no = \abs$. \s
\s
\s \s
\end{translateSixCol}

\peanoPage{14} % page-number 14

\begin{translateSixCol}{0.1}{0.3}{0.1}{0.1}{0.3}{0.1}
\raggedright
13. \s $p,q,p',q' \smallIn \N \p \C :: [(r,s,l) \smallIn] : r,s,t \smallIn \N \p \frac{p}{q} = \frac{r}{t} \p \frac{p'}{q'} = \frac{s}{t} \ppp \no = \abs$. \s
\s
\s \s
\s*
13'. \s $a,b \smallIn \R \p \C :: [(r,s,t) \smallIn] : r,s,t \smallIn \N \p a = \frac{r}{t} \p b = \frac{s}{t} \ppp \no = \abs$. \s
\s
\s \s
\s*
14. \s $a,b,c \smallIn \R \p \C :: [(m,n,p,q) \smallIn] : m,n,p,q \smallIn \N \p a = \frac{m}{q} \p b = \frac{n}{q} \p c = \frac{p}{q} \ppp \no = \abs$. \s
\s
\s \s
\s*
15. \s $p,q,r \smallIn \N \p a = \frac{p}{r} \p b = \frac{q}{r} : \C : a = b \p = \p p = q$. \s
\s
\s \s
\s*
16. \s $m \smallIn \N \p a,b \smallIn \R \p a = b \p ma \smallIn \N : \C \p mb \smallIn N$. \s
\s
\s \s
\s*
17. \s $a,b,c \smallIn \R \p \C \ppp a = a$. \s
\s
\s \s
\s*
\s $\hspace{1.72cm} \C \ppp a = b \p = \p b = a$. \s
\s
\s \s
\s*
\s $\hspace{1.72cm} \C \ppp a = b \p b = c : \C \p a = c$. \s
\s
\s \s
\s*
18. \s $N \C \R$. \s
\s
\s \s
\end{translateSixCol}

\begin{translateTwoCol}
\centering
\phantomsection
\addcontentsline{toc}{subsection}{Definitions}
\peanoHeadingSmall{Definitiones.}
\s
\peanoHeadingSmall{Definitions.}
\end{translateTwoCol}

\begin{translateSixCol}{0.1}{0.35}{0.05}{0.1}{0.35}{0.05}  %wider than usual
\raggedright
19. \s $a,b \smallIn \R \p \C :: a \smaller b \p = \ppp x \smallIn \N \p xa, xb \smallIn \N : \C \p xa \smaller xb$. \s
\s
19. \s $a,b \in \mathbb{Q}^+ \rightarrow a < b = ([x \in \mathbb{N} \wedge (xa, xb \in \mathbb{N})] \rightarrow xa < xb)$.\s
\s*
20. \s $a,b \smallIn \R \p \C : b \larger a \p = \p a \smaller b$. \s
\s
20. \s $a,b \in \mathbb{Q}^+ \rightarrow b \larger a = a \smaller b$. \s
\end{translateSixCol}

\begin{translateTwoCol}
\centering
\phantomsection
\addcontentsline{toc}{subsection}{Theorems}
\peanoHeadingSmall{Theoremata.}
\s
\peanoHeadingSmall{Theorems.}
\end{translateTwoCol}

\begin{translateSixCol}{0.1}{0.3}{0.1}{0.1}{0.3}{0.1}
\raggedright
21. \s $p,q,r \smallIn \N \p a = \frac{p}{r} \p b = \frac{q}{r} : \C : a \smaller b \p = \p p \smaller q$. \s
\s
21. \s \todo... \s
\s*
22. \s $p,q,p',q' \smallIn \N \p \C : \frac{p}{q} \smaller \frac{p'}{q'} \p = \p pq' \smaller p'q$. \s
\s
\s \s
\s*
23. \s $p,q,r \smallIn \N \p a = \frac{r}{p} \p b = \frac{r}{q} : \C : a \smaller b \p = \p p > q$. \s
\s
\s \s
\s*
24. \s $p,q,p',q' \smallIn \N \p \frac{p}{q} \smaller \frac{p'}{q'} : \C \p \frac{p}{q} \smaller \frac{p + p'}{q + q'} \smaller \frac{p'}{q'}$. \s
\s
\s \s
\s*
25. \s $a \smallIn \R \p \C \ppp ? \p \such \larger a : \no = \abs$. \s
\s
\s \s
\s*
26. \s $a \smallIn \R \p \C \ppp \R \p \such \smaller a : \no = \abs$. \s
\s
\s \s
\s*
27. \s $a,b \smallIn \R \p a \smaller b : \C \ppp \R \p \such \larger a \p \such \smaller b : \no = \abs$. \s
\s
\s \s
\s*
28. \s $a,b \smallIn \R : \C \ppp a \smaller b \p a = b : = \abs$. \s
\s
\s \s
\s*
\s $\hspace{1.5cm} \C \ppp a \larger b \p a = b : = \abs$. \s
\s
\s \s
\s*
\s $\hspace{1.5cm} \C \ppp a \smaller b \p a \larger b : = \abs$. \s
\s
\s \s
\s*
\s $\hspace{1.5cm} \C \ppp a \no \smaller b \p a \no = b \p a \no \larger b : = \abs$. \s
\s
\s \s
\s*
29. \s $a,b,c \smallIn \R : \C \ppp a \smaller \cup = b \p b \smaller c : \C : a \smaller c$. \s
\s
\s \s
\s*
\s $\hspace{1.7cm} \C \ppp a \smaller b \p b \smaller \cup = c : \C : a \smaller c$. \s
\end{translateSixCol}

\peanoPage{15} % page-number 15

\begin{translateTwoCol}
\centering
\phantomsection
\addcontentsline{toc}{subsection}{Definitions}
\peanoHeadingSmall{Definitiones.}
\s
\peanoHeadingSmall{Definitions.}
\end{translateTwoCol}

\begin{translateSixCol}{0.1}{0.35}{0.05}{0.1}{0.35}{0.05}  % wider than usual
\raggedright
30. \s $a,b \smallIn \R \p \C \p a + b = [c \smallIn] (c \smallIn  \R \ppp x \smallIn \N \p x a, x b, x c \smallIn \N : \C_x \p xa + xb = xc)$. \s
\s
30. \s $a,b \in \mathbb{Q}^+ \rightarrow a + b = \big\{c \big| c \in  \mathbb{Q}^+ \wedge \big( [ x \in \mathbb{N}  \wedge (x a, x b, x c \in \mathbb{N})] \xrightarrow[\forall x]{}  xa + xb = xc \big) \big\}$. \s
\s*
31. \s $a,b \smallIn \R \p \C :: b - a = \ppp [x \smallIn] (x \smallIn \R \p a + x = b)$. \s
\s
31. \s $a,b \in \mathbb{Q}^+ \rightarrow b - a = \{ x | x \in \mathbb{Q}^+ \wedge a + x = b \}$. \s
\s*
32. \s $a,b \smallIn \R \p \C \p ab = [c \smallIn] (c \smallIn \R \ppp x \smallIn \N \p xa, (xa)b, xc \smallIn \N : \C_x \p (xa) b = xc)$. \s
\s
32. \s $a,b \in \mathbb{Q}^+ \rightarrow ab = \big\{ c \big| c \in \mathbb{Q}^+ \wedge \big[ \big( x \in \mathbb{N} \wedge [xa, (xa)b, xc \in \mathbb{N}] \big) \xrightarrow[\forall x]{} (xa) b = xc \big] \big\}$. \s
\s*
33. \s $a,b \smallIn \R \p \C \p b / a = [x \smallIn] (x \smallIn \R \p ax = b)$. \s
\s
33. \s $a,b \in \mathbb{Q}^+ \rightarrow b / a = \{ x | x \in \mathbb{Q}^+ \wedge ax = b \}$. \s
\end{translateSixCol}

\begin{translateTwoCol}
\centering
\phantomsection
\addcontentsline{toc}{subsection}{Theorems}
\peanoHeadingSmall{Theoremata.}
\s
\peanoHeadingSmall{Theorems.}
\end{translateTwoCol}

\begin{translateSixCol}{0.1}{0.3}{0.1}{0.1}{0.3}{0.1}
\raggedright
34. \s $p,q,r \smallIn \N \p \C \frac{p}{r} + \frac{q}{r} = \frac{p + q}{r}$. \s
\s
34. \s \todo... \s
\s*
35. \s $a,b \smallIn \R \p \C \p a + b \smallIn \R$. \s
\s
\s \s
\s*
36. \s $p,q,r \smallIn \N \p p \smaller q : \C \p \frac{q}{r} - \frac{p}{r} = \frac{q-p}{r}$. \s
\s
\s \s
\s*
37. \s $a,b \smallIn \R \p a \smaller b : \C \p b - a \smallIn \R$. \s
\s
\s \s
\s*
38. \s $p,q,p',q' \smallIn \N \p \C \p \frac{p}{q} \frac{p'}{q'} = \frac{pp'}{qq'}$. \s
\s
\s \s
\s*
39. \s $a,b \smallIn \R \p \C \p ab \smallIn \R$. \s
\s
\s \s
\s*
40. \s $p,q,p',q' \smallIn \N \p \C \p \frac{p}{q} / \frac{p'}{q'} = \frac{pq'}{p'q}$. \s
\s
\s \s
\s*
41. \s $a,b \smallIn \R \p \C \p b / a \smallIn \R$. \s
\s
\s \s
\s*
42. \s $p,q \smallIn \N \p \C \p \frac{p}{q} = \frac{p}{q}$. \s
\s
\s \s
\end{translateSixCol}

\begin{translateTwoCol}
\centering
\phantomsection
\addcontentsline{toc}{section}{\S9. The system of rationals. Irrationals}
\peanoHeadingMedium{\S9. Rationalum systemata. Irrationales.}
\s
\peanoHeadingMedium{\S9. The system of rationals. Irrationals.}
\s*
\phantomsection
\addcontentsline{toc}{subsection}{Explanation}
\peanoHeadingSmall{Explicatio.}
\s
\peanoHeadingSmall{Explanation.}
\end{translateTwoCol}

\begin{translateTwoCol}
\raggedright
Si $a \smallIn \K \R$, signum $\T a$ legitur \emph{terminus summus}, vel \emph{limes summus classis} $a$. Supra hoc novum ens relationes ac operationes tantum definimus.
\s
If $A \in \setOfSets \mathbb{Q}^+$, the symbol $\sup(A)$ is read \emph{upper boundary}, or \emph{upper limit of the set} $A$. We shall define only a few relations and operations on this new entity.
\end{translateTwoCol}

\begin{translateTwoCol}
\centering
\phantomsection
\addcontentsline{toc}{subsection}{Definitions}
\peanoHeadingSmall{Definitiones.}
\s
\peanoHeadingSmall{Definitions.}
\end{translateTwoCol}

\begin{translateSixCol}{0.1}{0.35}{0.05}{0.1}{0.35}{0.05} % wider than usual
\raggedright
1. \s $a \smallIn \K \R \p x \smallIn \R : \C :: x \smaller \T a \p = \ppp a \p \such \larger x : \no = \abs$. \s
\s
1. \s $(A \in \setOfSets \mathbb{Q}^+ \wedge x \in \mathbb{Q}^+) \rightarrow x < \sup(A) = [(A \cap \{ z | z > x \}) \neq \varnothing]$. \s
\s*
2. \s $a \smallIn \K \R \p x \smallIn \R : \C ::: x = \T a \p = : \p : a \p \such \larger x : = \abs :: u \smallIn \R \p u \smaller x : \C_x \ppp a \p \such \larger u : \no = \abs$. \s
\s
2. \s $(A \in \setOfSets \mathbb{Q}^+ \wedge x \in \mathbb{Q}^+) \rightarrow ( x = \sup(A) ) = \big[ [(A \cap \{ z | z > x \}) = \varnothing] \wedge \big( (u \in \mathbb{Q}^+ \wedge u < x) \xrightarrow[\forall u]{} [(A \cap \{ z | z \larger u \}) \neq \varnothing] \big) \big]$. \s
\s*
3. \s $a \smallIn \K \R \p x \smallIn \R : \C \ppp x \larger \T a \p = : x \no \smaller \T a \p x \no = \T a$. \s
\s
3. \s $( A \in \setOfSets \mathbb{Q}^+ \wedge x \in \mathbb{Q}^+ ) \rightarrow (x > \sup(A)) = [(x \nless \sup(A)) \wedge (x \neq \sup(A))] $. \s
\end{translateSixCol}

\peanoPage{16} % page-number 16

\begin{translateTwoCol}
\centering
\phantomsection
\addcontentsline{toc}{subsection}{Theorem}
\peanoHeadingSmall{Theorema.}
\s
\peanoHeadingSmall{Theorem.}
\end{translateTwoCol}

\begin{translateSixCol}{0.1}{0.3}{0.1}{0.1}{0.3}{0.1}
\raggedright
4. \s $x \smallIn \R \p \C :: x = \ppp \T : \R \p \such \smaller x$. \s
\s
4. \s $x \in \mathbb{Q}^+ \rightarrow  x = \sup(\mathbb{Q}^+ \cap \{z | z < x \})$. \s
\end{translateSixCol}

\begin{translateTwoCol}
\centering
\phantomsection
\addcontentsline{toc}{subsection}{Explanation}
\peanoHeadingSmall{Explicatio.}
\s
\peanoHeadingSmall{Explanation.}
\end{translateTwoCol}

\begin{translateTwoCol}
\raggedright
Signum $\Q$ legitur \emph{quantitas}, numerosque indicat reales positivos, rationales aut irrationales, $0$ et $\infty$ exceptis.
\s
The symbol $\mathbb{R}^+$ is read \emph{quantity}, and indicates the positive real numbers, rational or irrational, with the exception of $0$ and $\infty$.
\end{translateTwoCol}

\begin{translateTwoCol}
\centering
\phantomsection
\addcontentsline{toc}{subsection}{Definitions}
\peanoHeadingSmall{Definitiones.}
\s
\peanoHeadingSmall{Definitions.}
\end{translateTwoCol}

\begin{translateSixCol}{0.1}{0.3}{0.1}{0.1}{0.3}{0.1}
\raggedright
5. \s $\Q = [x \smallIn] (a \smallIn \K \R : a \no = \abs : \R \smallIn \larger \T a \p \no = \abs : \T a = x \ppp \no = \abs)$. \s
\s
5. \s $\mathbb{R}^+ = \{x | [A \in \setOfSets \mathbb{Q}^+ \wedge A \neq \varnothing \wedge (\mathbb{Q}^+ \cap \{ z | z > \sup(A) \}) = \varnothing \wedge \sup(A) = x] \neq \bot \}$. \s
\s*
6. \s $a,b \smallIn \Q \p \C :: a = b \p = \ppp \R \p \such \smaller a : = : \R \p \such \smaller b$. \s
\s
6. \s $a,b \in \mathbb{R}^+ \rightarrow (a = b) = [( \mathbb{Q}^+ \cap \{ z | z < a \}) = (\mathbb{Q}^+ \cap \{ z | z < b \})]$ \s
\s*
7. \s $a,b \smallIn \Q \p \C :: a \smaller b \p = \ppp \R \p \such \larger a \p \such \smaller b : \no = \abs$. \s
\s
7. \s $a,b \in \mathbb{R}^+ \rightarrow a < b = [(\mathbb{Q}^+ \cap \{z | z > a \} \cap \{ z | z < b \} \neq \varnothing]$. \s
\s*
8. \s $a,b \smallIn \Q \p \C : b \larger a \p = \p a \smaller b$. \s
\s
8. \s $a,b \in \mathbb{R}^+ \rightarrow b > a = a < b$. \s
\end{translateSixCol}

\begin{translateTwoCol}
\centering
\phantomsection
\addcontentsline{toc}{subsection}{Theorems}
\peanoHeadingSmall{Theoremata.}
\s
\peanoHeadingSmall{Theorems.}
\end{translateTwoCol}

\begin{translateSixCol}{0.1}{0.3}{0.1}{0.1}{0.3}{0.1}
\raggedright
9. \s $a \smallIn \Q \p \C \ppp \R \p \such \smaller a : \no = \abs$. \s
\s
9. \s $a \in \mathbb{R}^+ \rightarrow (\mathbb{Q}^+ \cap \{ z | z < a \}) \neq \varnothing$. \s
\s*
10. \s $a \smallIn \Q \p \C \ppp \R \p \such \larger a : \no = \abs$. \s
\s
10. \s $a \in \mathbb{R}^+ \rightarrow (\mathbb{Q}^+ \cap \{z | z > a \}) \neq \varnothing$. \s
\s*
11. \s $\R \C \Q$. \s
\s
11. \s $\mathbb{Q}^+ \subset \mathbb{R}^+$. \s
\s*
\vspace{1em}
\s Subsistunt quoque propositiones quae a P17, 28, 29 in \S8 obtinentur, si loco $\R$ legatur $\Q$. \s
\s
\vspace{1em}
\s The propositions obtained from P17, 28, 29 in \S8 also hold, by reading $\mathbb{R}^+$ for $\mathbb{Q}^+$. \s
\end{translateSixCol}

\begin{translateTwoCol}
\centering
\phantomsection
\addcontentsline{toc}{subsection}{Definitions}
\peanoHeadingSmall{Definitiones.}
\s
\peanoHeadingSmall{Definitions.}
\end{translateTwoCol}

\begin{translateSixCol}{0.1}{0.3}{0.1}{0.1}{0.3}{0.1}
\raggedright
12. \s $a,b \smallIn Q \p \C \p a + b = \T [z \smallIn] ([(x,y) \smallIn] : x,y \smallIn \R \p x \smaller a \p y \smaller b \p x + y = z \ppp \no = \abs)$. \s
\s
12. \s $a,b \in \mathbb{R}^+ \rightarrow  a + b = \sup\big( \big\{z \big| \{(x,y) | (x,y \in \mathbb{Q}^+) \wedge x < a \wedge y < b \wedge x + y = z \} \neq \varnothing \big\} \big)$. \s
\s*
13. \s $a,b \smallIn Q \p \C \p ab = \T [z \smallIn] ([(x,y) \smallIn] : x,y \smallIn \R \p x \smaller a \p y \smaller b \p xy = z \ppp \no = \abs)$. \s
\s
13. \s $a,b \in \mathbb{R}^+ \rightarrow ab = \sup\big( \big\{ z \big| \{(x,y) |  (x,y \in \mathbb{Q}^+) \wedge x < a \wedge y < b \wedge xy = z \} \neq \varnothing \big\} \big)$. \s
\s*
\vspace{1em}
\s Ut valeant hae definitiones, demonstrandum est subsistere propositiones 12 et 13, si $a,b \smallIn \R$. \s
\s
\vspace{1em}
\s In order for these definitions to have meaning, it must be proved that propositions 12 and 13 hold, if $a,b \smallIn \mathbb{Q}^+$. \s
\s*
\vspace{1em}
\s Substractionem et divisionem ut operationes inversas additiones et multiplicationis definire licet, illarumque proprietas demonstrare. \s
\s
\vspace{1em}
\s Subtraction and division could be defined as the inverse operations to addition and multiplication, and their properties could be proved. \s
\end{translateSixCol}

\begin{translateTwoCol}
\centering
\phantomsection
\addcontentsline{toc}{section}{\S10. System of quantities}
\peanoHeadingMedium{\S10. Quantitatum systemata.}
\s
\peanoHeadingMedium{\S10. System of quantities.}
\s*
\phantomsection
\addcontentsline{toc}{subsection}{Explanations}
\peanoHeadingSmall{Explicationes.}
\s
\peanoHeadingSmall{Explanations.}
\end{translateTwoCol}

\begin{translateTwoCol}
\raggedright
Si $a \smallIn \K \Q$, signa $\I a$, $\E a$, $\Lfat a$ leguntur: \emph{interior, exterior, limes classis} $a$.
\s
If $A \in \setOfSets \mathbb{R}^+$, the symbols $\interior(A)$, $\exterior(A)$, $\boundary(A)$ are read: \emph{interior, exterior, limit of the set} $A$.
\end{translateTwoCol}

\peanoPage{17} % page-number 17

\begin{translateTwoCol}
\centering
\phantomsection
\addcontentsline{toc}{subsection}{Definitions}
\peanoHeadingSmall{Definitiones.}
\s
\peanoHeadingSmall{Definitions.}
\end{translateTwoCol}

\begin{translateSixCol}{0.1}{0.3}{0.1}{0.1}{0.3}{0.1}
\raggedright
1. \s $a \smallIn \K \Q \p \C \I a = \Q [x \smallIn] ([(u,v) \smallIn] :: u,v \smallIn \Q \ppp u \smaller x \smaller v \ppp \such \larger u \p \such \smaller v : \C : a : \p : \no = \abs)$. \s
\s
1. \s $A \in \setOfSets \mathbb{R}^+ \rightarrow \interior(A) = \mathbb{R}^+ \cap \big\{ x \big| \{ (u,v) | (u,v \in \mathbb{R}^+) \wedge ( u < x < v ) \wedge (\{ z | z > u \} \cap \{ z | z < v \}) \subset A \} \neq \varnothing \big\}$. \s
\s*
2. \s $a \smallIn \K \Q \p \C \p \E a = \I (\no a)$. \s
\s
2. \s $A \in \setOfSets \mathbb{R}^+ \rightarrow \exterior(A) = \interior\big( \overline{A} \big)$. \s
\s*
3. \s $a \smallIn \K \Q \p \C \p \Lfat a = (\no \I a) (\no \E a)$. \s
\s
3. \s $A \in \setOfSets \mathbb{R}^+ \rightarrow \boundary(A) = \overline{\interior(A)} \cap \overline{\exterior(A)}$. \s
\end{translateSixCol}

\begin{translateTwoCol}
\centering
\phantomsection
\addcontentsline{toc}{subsection}{Theorems}
\peanoHeadingSmall{Theoremata.}
\s
\peanoHeadingSmall{Theorems.}
\end{translateTwoCol}

\begin{translateSixCol}{0.1}{0.3}{0.1}{0.1}{0.3}{0.1}
\raggedright
4. \s $a \smallIn \K \Q \p x,u,v \smallIn Q \p u \smaller x \smaller v \p (\such \larger u \p \such \smaller v : \C a) : \C \p x \smallIn \I a$. \s
\s
4. \s \todo... \s
\s*
5. \s $a \smallIn \K \Q \p x \smallIn \I a : \C : [(u,v) \smallIn] (u,v \smallIn \Q \ppp u \smaller x \smaller v \ppp \such \larger u \p \such \smaller v : \C : a) \no = \abs$. \s
\s
\s \s
\s*
\emph{Dem.} \s P1 $=$ (P4)(P5). \s
\s
\s \s
\s*
6. \s $a \smallIn \K \Q \p u,v \smallIn \Q \p (\such \larger u \p \such \smaller v : \C a) \ppp \C \ppp \such \larger u \p \such \smaller v : \C \I a$. \s
\s
\s \s
\s*
\emph{Dem.} \s P6 $=$ P4. \s
\s
\s \s
\s*
7. \s $a \smallIn \K \Q \p \C \p \I a \C a$. \s
\s
\s \s
\s*
8. \s $ a \smallIn \K \Q \p \C \p \II a = \I a$. \s
\s
\s \s
\s*
\emph{Dem.} \s Hp. $(\I a) [a]$ P7 $: \C \p \II a \C \I a$ \s (1)
\s
\s \s
\s*
\s Hp. $x,u,v \smallIn \Q \p u \smaller x \smaller v \p (\such \larger u \p \such \smaller v : \C a) \p$ P6 $: \C : u,v \smallIn \Q \p u \smaller x \smaller v \p (\such \larger u \p \such \smaller v : \C \I a)$ \s (2)
\s
\s \s
\s*
\s Hp. $x \smallIn \I a \p (2) : \C : x \smallIn \II a$ \s (3)
\s
\s \s
\s*
\s Hp. $(3) : \C : \I a \II a$ \s (4)
\s
\s \s
\s*
\s Hp. $(1) \p (4) : \C :$ Ts. \s (Theor.)
\s
\s \s
\s*
9. \s $a,b \smallIn \K \Q \p a \C b : \C \p \I a \C \I b$ \s
\s
\s \s
\s*
\emph{Dem.} \s Hp. $x,u,v \smallIn \Q \p u \smaller x \smaller v \p (\such \larger u \p \such \smaller v : \C a) : \C \ppp \such \larger u \p \such \smaller v : \C b$ \s (1)
\s
\s \s
\s*
\s Hp. $x \smallIn \I a : \C : x \smallIn \I b$ \s (Theor.)
\s
\s \s
\s*
10. \s $a,b \smallIn \K \Q : \C : \I (ab) \C \I a$ \s
\s
\s \s
\s*
\emph{Dem.} \s $(ab, a) [a,b]$ P9 $\p = \p$ P10 \s
\s
\s \s
\s*
11. \s $a,b \smallIn \K \Q \p \C \p \I (ab) \C (\I a) (\I b)$ \s
\s
\s \s
\s*
\emph{Dem.} \s P11 $= :$ P10 $\p \cap \p (b,a) [a,b]$ P10 \s
\s
\s \s
\s*
12. \s $a,b \smallIn \K \Q \p \C \p \I a \C \I (a \cup b)$ \s
\s
\s \s
\s*
13. \s $a,b \smallIn \K \Q \p \C \p \I a \cup \I b \C \I (a \cup b)$ \s
\s
\s \s
\s*
14. \s $a,b \smallIn \K \Q \p \C \p \I (ab) = (\I a)(\I b)$ \s
\s
\s \s
\s*
\emph{Dem.} \s Hp. P11 $: \C \p \I (ab) \C (\I a)(\I b)$ \s (1)
\s
\s \s
\end{translateSixCol}

\peanoPage{18} % page-number 18

\begin{translateSixCol}{0.1}{0.3}{0.1}{0.1}{0.3}{0.1}
\raggedright
\s Hp. $x \smallIn \Q \p u,v \smallIn \Q \p u \smaller x \smaller v \p (\such \larger u \p \such \smaller v : \C a) \p u',v' \smallIn \Q \p u' \smaller x \smaller v' \p (\such \larger u' \p \such \smaller v' : \C b) \p u'' = M (u \cup u') \p v'' = \mini (v,v') : \C : u'', v'' \smallIn Q \p u '' \smaller x \smaller v'' \p (\such \larger u'' \p \such \larger v'' : \C : ab)$ \s (2)
\s
\s \s
\s*
\s Hp. $x \smallIn \I a \p x \smallIn \I b \p (2) : \C \p x \smallIn \I (ab)$ \s (3)
\s
\s \s
\s*
\s Hp. $(3) : \C : (\I a)(\I b) \C \I (ab)$ \s (4)
\s
\s \s
\s*
\s Hp. $(1) \p (4) : \C \p$ Ts. \s
\s
\s \s
\s*
15. \s $a \smallIn \K \Q \p \C \p \E a \C - a$ \s
\s
\s \s
\s*
\emph{Dem.} \s P15 $= (\no a) [a]$ P7 \s
\s
\s \s
\s*
16. \s $a \smallIn \K \Q \p \C \ppp \I a \p \E a : = \abs$ \s
\s
\s \s
\s*
\emph{Dem.} \s Hp. P7 $.$ P15 $: \C \ppp \I a \p \E a : \C : a \no a : = \abs$ \s
\s
\s \s
\s*
17. \s $a \smallIn \K \Q \p \C \p \I \E a = \E a$ \s
\s
\s \s
\s*
\emph{Dem.} \s P17 $= (\no a) [a]$ P8 \s
\s
\s \s
\s*
18. \s $a,b \smallIn \K \Q \p b \C a : \C \p \E a \C \E b$ \s
\s
\s \s
\s*
\emph{Dem.} \s P18 $= (\no a, \no b) [a,b]$ P9 \s
\s
\s \s
\s*
19. \s $a,b \smallIn \K \Q \p \C : \E a \cup \E b \p \C \E (ab)$ \s
\s
\s \s
\s*
20. \s $a,b \smallIn \K \Q \p \C \p \E (a \cup b) = (\E a) (\E b)$ \s
\s
\s \s
\s*
\emph{Dem.} \s P20 $= (\no a, \no b) [a,b]$ P14 \s
\s
\s \s
\s*
21. \s $a \smallIn \K \Q \p \C \p \Lfat (\no a) = \Lfat a$ \s
\s
\s \s
\s*
22. \s $a \smallIn \K \Q \p \C \ppp \I a \p \Lfat a : = \abs$ \s
\s
\s \s
\s*
\s $\hspace{1.68cm} \C \ppp \E a \p \Lfat a : = \abs$ \s
\s
\s \s
\s* 
\s $\hspace{1.68cm} \C \ppp \no \I a \p \no \E a \p \no \Lfat a : = \abs$ \s
\s
\s \s
\s*
\emph{Dem.} \s P22 $=$ P3 \s
\s
\s \s
\s*
23. \s $a \smallIn \K \Q \p \C : a \C \p \I a \cup \Lfat a$ \s
\s
\s \s
\s*
24. \s $a \smallIn \K \Q \p \C \p \I (a \Lfat a) = \abs$ \s
\s
\s \s
\s*
\emph{Dem.} \s Hp. P14 $\p$ P7 $\p$ P22 $: \C : \I (a \Lfat a) \p = \p \I a \I \Lfat a \p \C \p \I a \Lfat a \p = \p \abs$ \s
\s
\s \s
\s*
25. \s $a,b \smallIn \K \Q \p a \C b : \C : \Lfat a \p \C \p \I b \cup \Lfat b$ \s
\s
\s \s
\s*
\emph{Dem.} \s Hp. P18 $: \C : \E b \C \E a : \C : \I a \cup \Lfat a \p \C \p \I b \cup \Lfat b : \C \p$ Ts. \s
\s
\s \s
\s*
26. \s $a,b \smallIn \K \Q \p \C : \Lfat (ab) \C \p \I a \Lfat b \cup \I b \Lfat a \cup \Lfat a \Lfat b$ \s
\s
\s \s
\s*
\emph{Dem.} \s Hp. $\C : ab \C a \p ab \C b \p$ P25 $: \C : \Lfat (ab) \C \I a \cup \Lfat a \p \Lfat (ab) \C \I b \cup \Lfat b : \C : \Lfat (ab) \C (\I a \cup \Lfat a) (\I b \cup \Lfat b) \p \Lfat (ab) (\I a) (\I b) = \Lfat (ab) \I (ab) = \abs : \C :$ Ts. \s
\s
\s \s
\s*
26' \s $a,b \smallIn \K \Q \p \C \p \Lfat (ab) \C \Lfat a \cup \Lfat b$ \s
\s
\s \s
\s*
27. \s $a,b \smallIn \K \Q \p \C : \Lfat (a \cup b) = \Lfat a \E b \cup \Lfat b \E a \cup \Lfat a \Lfat b $ \s
\s
\s \s
\s*
\emph{Dem.} \s P27 $= (\no a, \no b) [a,b]$ P26 \s
\s
\s \s
\end{translateSixCol}

\peanoPage{19} % page-number 19

\begin{translateSixCol}{0.1}{0.3}{0.1}{0.1}{0.3}{0.1}
\raggedright
27'. \s $a,b \smallIn \K \Q \p \C : \Lfat (a \cup b) \C \Lfat a \cup \Lfat b$ \s
\s
\s \s
\s*
28. \s $a \smallIn \K \Q \p \C \p \Lfat \I a \C \Lfat a$ \s
\s
\s \s
\s*
\emph{Dem.} \s Hp. P7 $: \C : \I a \C a \p$ P25 $: \C : \Lfat \I a \C \I a \cup \Lfat a$ \s (1)
\s
\s \s
\s*
\s Hp. P8 $\p$ P22 $: \C \p \Lfat \I a \I a = \Lfat \I a \II a = \abs$ \s (2)
\s
\s \s
\s*
\s $(1)(2) \p \C \p$ Theor. \s
\s
\s \s
\s*
28'. \s $a \smallIn \K \Q \p \C \Lfat \E a \C \Lfat a$ \s
\s
\s \s
\s*
29. \s $a \smallIn \K \Q \p \C \p \LfatLfat a \C \Lfat \I a \cup \Lfat \E a$ \s
\s
\s \s
\s*
\emph{Dem.} \s Hp. $\C : \LfatLfat a = \Lfat (\I a \cup \E a) \p$ P27' $: \C \p$ Ts. \s
\s
\s \s
\s*
29'. \s $a \smallIn \K \Q \p \C \p \LfatLfat a \C \Lfat a$ \s
\s
\s \s
\s*
\emph{Dem.} \s P29 $\p$ P28 $\p$ P28' $: \C \p$ Theor. \s
\s
\s \s
\s*
30. \s $a \smallIn \K \Q \p \C \p \Lfat a = \I\Lfat a \cup \LfatLfat a$ \s
\s
\s \s
\s*
\emph{Dem.} \s Hp. P23 $: \C \p \Lfat a \C \I\Lfat a \cup \LfatLfat a$ \s (1)
\s
\s \s
\s*
\s Hp. P7 $: \C \p \I \Lfat a \C \Lfat a$ \s (2)
\s
\s \s
\s*
\s Hp. P29' $: \C \p \LfatLfat a \C \Lfat a$ \s (3)
\s
\s \s
\s*
\s $(1)(2)(3) \p \C \p$ Theor. \s
\s
\s \s
\s*
31. \s $a \smallIn \K \Q \p \C \p \Lfat \I \Lfat a \C \LfatLfat a$ \s
\s
\s \s
\s*
\emph{Dem.} \s P31 $= (\Lfat a) [a]$ P28 \s
\s
\s \s
\s*
32. \s $a \smallIn \K \Q \p \C \p \I \LfatLfat a = \abs$ \s
\s
\s \s
\s*
\emph{Dem.} \s Hp. P29' $: \C : \LfatLfat a = \Lfat a \LfatLfat a \p (\Lfat a) [a]$ P24 $: \C$ Ts. \s
\s
\s \s
\s*
33. \s $a \smallIn \K \Q \p \C : \I\Lfat\I\Lfat a = \abs$ \s
\s
\s \s
\s*
\emph{Dem.} \s P31 $\p$ P32 $: \C \p P33$ \s
\s
\s \s
\s*
34. \s $a \smallIn \K \Q \p \C \p \LfatLfat\Lfat a = \LfatLfat a$ \s
\s
\s \s
\s*
\emph{Dem.} \s $(\Lfat a) [a]$ P30 $/p$ P32 $: \C \p$ Theor. \s
\s
\s \s
\s*
35. \s $a,b \smallIn \K \Q \p \C \p \I a \Lfat b \C \Lfat (ab)$ \s
\s
\s \s
\s*
\emph{Dem.} \s Hp. P14 $: \C \p \I a \Lfat b \I (ab) = \I a \I b \Lfat b = \abs$ \s (1)
\s
\s \s (1)
\s*
\s Hp. P2 $\p$ P14 $: \C \p \I a \Lfat b \E (ab) = \I a \Lfat b \I (\no a \cup \no b) = \I (a \no b) \Lfat b = \I a \E b \Lfat b = \abs$ \s (2)
\s
\s \s (2)
\s*
\s $(1)(2) \C$ Theor. \s
\s
\s \s
\s*
36. \s $a,b \smallIn \K \Q \p \C \p \I a \Lfat b \cup \I b \Lfat a \C \Lfat ab$. \s (Vide P26)
\s
\s \s (Cf. p. 26)
\s*
\emph{Dem.} \s P36 $= :$ P35 $\p (b,a) [a,b]$ P35 \s
\s
\s \s
\s*
37. \s $a,b \smallIn \K \Q \p \C \p \E a \Lfat b \cup \E b \Lfat a \cup \Lfat (a \cup b)$. \s (Vide P27)
\s
\s \s (Cf. p. 27)
\s*
\emph{Dem.} \s P37 $= (\no a, \no b)[a,b]$ P36 \s
\s
\s \s
\s*
38. \s $a,b \smallIn \K \Q \p \C \p \I (a \cup b) \C \I a \cup \I b \cup \Lfat a \Lfat b$ \s (Vide P13)
\s
\s \s (Cf. p. 13)
\s*
\emph{Dem.} \s Hp. $\C \p \I (a \cup b) \C (\I a \cup \Lfat a \cup \E a) (\I b \cup \Lfat b \cup \E b)$ \s (1)
\s
\s \s  (1)
\s*
\s Hp. P20 $\p$ P16 $: \C \p \I (a \cup b) \E a \E b = \I (a \cup b) \E (a \cup b) = \abs$ \s (2)
\s
\s \s (2)
\end{translateSixCol}

\peanoPage{20} % page-number 20

\begin{translateSixCol}{0.1}{0.3}{0.1}{0.1}{0.3}{0.1}
\raggedright
\s Hp. P37$: \C : \I (a \cup b)(\E a \Lfat b \cup \E b \Lfat a) \p \C \p \I (a \cup b) \Lfat (a \cup b) \p = \abs$ \s (3)
\s
\s \s (3)
\s*
\s $(1)(2)(3) \p \C \p$ Theor. \s
\s
\s \s
\s*
38'. \s $a,b \smallIn \K \Q \p \C \p \E (ab) \C \E a \cup \E b \cup \Lfat a \Lfat b$ \s (Vide P19)
\s
\s \s (Cf. p. 19)
\s*
39. \s $a \smallIn \K \Q \p \C \p \I\Lfat a \Lfat\I a = \abs$ \s
\s
\s \s
\s*
\emph{Dem.} \s Hp. P36 $: \C : \I\Lfat a \Lfat\I a \C \Lfat (\Lfat a \I a) = \abs$ \s
\s
\s \s
\s*
40. \s $a \smallIn \K \Q \p \C \p \Lfat \I a \C \LfatLfat a$ \s
\s
\s \s
\s*
\emph{Dem.} \s Hp. P28 $\p$ P30 $\p$ P39 $: \C$ Theor. \s
\s
\s \s
\s*
40'. \s $a \smallIn \K \Q \p \C \p \Lfat \E a \C \LfatLfat a$ \s
\s
\s \s
\s*
41. \s $a \smallIn \K \Q \p \C \LfatLfat a = \Lfat\I a \cup \Lfat\E a$ \s
\s
\s \s
\s*
\emph{Dem.} \s P29 $\p$ P40 $\p$ P40' $: \C \p$ Theor. \s
\s
\s \s
\s*
42. \s $a \smallIn \K \Q \p \C \p \I\Lfat\I a = \abs$ \s
\s
\s \s
\s*
\s $\hspace{1.68cm} \C \p \I\Lfat\E a = \abs$ \s
\s
\s \s
\s*
\s $\hspace{1.68cm} \C \p \LfatLfat\I a = \Lfat\I a$ \s
\s
\s \s
\s*
\s $\hspace{1.68cm} \C \p \LfatLfat\E a = \Lfat\E a$ \s
\s
\s \s
\s*
43. \s $a,b \smallIn \K \Q \p \C \p \I (\I a \cup \I b) = \I a \cup \I b$ \s
\s
\s \s
\s*
\emph{Dem.} \s Hp. P7 $: \C \p \I (\I a \cup \I b) \C \I a \cup \I b$ \s (1)
\s
\s \s (1)
\s*
\s Hp. P8 $\p$ P13 $: \C : \I a \cup \I b \p = \p \II a \cup \II b \p \C \p \I (\I a \cup \I b)$ \s (2)
\s
\s \s (2)
\s*
\s $(1)(2) \C$ Theor. \s
\s
\s \s
\s*
44. \s $a,b \smallIn \K \Q \p \C \p \I (\LfatLfat a \cup \LfatLfat b) = \abs$ \s
\s
\s \s
\s*
\emph{Dem.} \s Hp. P38 $\p$ P32 $\p$ P34 $: \C \p \I (\LfatLfat a \cup \LfatLfat b) \C \LfatLfat a \LfatLfat b \C \LfatLfat a$ \s (1)
\s
\s \s (1)
\s*
\s Hp. $(1) \p$ P8 $: \C \p \I (\LfatLfat a \cup \LfatLfat b) \cup \I \LfatLfat a = \abs$ \s
\s
\s \s
\s*
45. \s $a \smallIn \K \Q \p \C \p \I (\I a \cup \E a) = \I a \cup \E a$ \s
\s
\s \s
\s*
\emph{Dem.} \s P8 $\p$ P17 $\p (\no a) [b]$ P43 $: \C \p$ Theor. \s
\s
\s \s
\s*
45'. \s $a \smallIn \K \Q \p \C \p \E\Lfat a = \I a \cup \E a$ \s
\s
\s \s
\s*
46. \s $a \smallIn \K \Q \p \C \p \E\I a = \no (\I a \cup \Lfat\I a)$ \s
\s
\s \s
\s*
46'. \s $a \smallIn \K \Q \p \C \p \EE a = \no (\E a \cup \Lfat\E a)$ \s
\s
46'. \s \s
\end{translateSixCol}

\peanoPage{END} 


%\vspace{1em}
%\columnratio{1} \begin{paracol}{1} \centering \hdashrule{\columnwidth}{0.1mm}{0.1mm 1mm} \end{paracol} % end of page-numbering

\phantomsection
\addcontentsline{toc}{chapter}{Endnotes}
\vspace{1em}
\eendnoteHeading{\textbf{Endnotes}}
\theendnotes
\end{document}
